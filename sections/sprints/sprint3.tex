\newpage
\section{Sprint 3}

Siguiendo la planificación inicial, el tercer sprint se centra en la implementación del procesado multimedia en el servidor (generación de miniaturas, compresión de imágenes y vídeos, etiquetado basado en metadatos, etc.) y en el cliente móvil la subida de archivos multimedia al servidor y la visualización de los archivos subidos.

El objetivo es entregar un incremento funcional que permita al usuario subir fotos y vídeos desde el móvil, que el servidor procese estos archivos (compresión, miniaturas) y que puedan visualizarse en una galería online básica. Se priorizan historias que permitan una experiencia de usuario completa de subida y visualización, así como la robustez y eficiencia del proceso.

\subsection{Historias de usuario}
A continuación se presentan las historias de usuario y técnicas seleccionadas para este sprint, siguiendo el mismo formato que en los sprints anteriores. El desglose en tareas se realizará posteriormente.

Las historias seleccionadas son las siguientes:
\begin{itemize}
    % Servidor (prioridad máxima)
    \item HU01: Subida de fotos -- 5 PH
    \item HU04: Subida de vídeos -- 5 PH
    \item HT09: Compresión de imágenes -- 5 PH
    \item HT10: Subida concurrente -- 8 PH
    \item HU09: Galería visual -- 5 PH
    \item HT17: Notificaciones de progreso -- 3 PH
    % Cliente móvil (imprescindible para probar subida y visualización)
    \item HU16: Seleccionar fotos -- 3 PH
    \item HU27: Subir vídeos -- 5 PH
    \item HU18: Ver progreso de subida -- 8 PH
\end{itemize}

La suma total de las historias seleccionadas es de \textbf{47 puntos de historia (PH)}. Esta selección se ha ajustado para priorizar el procesado multimedia en el servidor y solo las funcionalidades imprescindibles del cliente móvil, manteniendo una carga realista y coherente con la capacidad demostrada en los sprints anteriores (34–56 PH). Se han dejado fuera historias menos críticas para el objetivo de este sprint, como la cancelación de subida, sincronización manual, galería online avanzada, manejo de errores de red y estadísticas de copia, que se abordarán en sprints posteriores.

\subsection{Descomposición en tareas de desarrollo}

% HU01: Subida de fotos
\begin{table}[H]
    \begin{center}
        \begin{tabularx}{\textwidth}{|l|X|l|}
            \hline
            \textbf{Identificador HU01} &
            \textbf{Como usuario, quiero subir varias fotos desde mi móvil para tener una copia de seguridad en mi servidor} &
            \textbf{Estimación: 5 PH}\\
            \hline
            \multicolumn{3}{|p{\textwidth}|}{
                \begin{minipage}{\textwidth}
                    \centering
                    \vspace{0.5em}
                    \begin{tabular}{|l|p{8cm}|r|}
                        \hline
                        \textbf{Identificador} & \textbf{Título de la tarea de desarrollo} & \makecell{\textbf{Estimación}\\\textbf{(h)}} \\
                        \hline
                        Tarea 01-1 & Implementar endpoint de subida de fotos (backend) & 2 \\
                        \hline
                        Tarea 01-2 & Validar y almacenar archivos recibidos & 1.5 \\
                        \hline
                        Tarea 01-3 & Integrar con sistema de almacenamiento (local o cloud) & 1 \\
                        \hline
                        Tarea 01-4 & Pruebas unitarias y de integración & 1 \\
                        \hline
                        Tarea 01-5 & Documentar endpoint & 0.5 \\
                        \hline
                    \end{tabular}
                    \vspace{0.5em}
                \end{minipage}
            } \\
            \hline
            \multicolumn{3}{|p{\textwidth}|}{
                \textbf{Pruebas de aceptación:}
                \begin{itemize}
                    \item El usuario puede subir una o varias fotos desde el móvil.
                    \item Los archivos se almacenan correctamente en el servidor.
                    \item El endpoint rechaza archivos no válidos.
                \end{itemize}
            }\\
            \hline
            \multicolumn{3}{|p{\textwidth}|}{
                \textbf{Observaciones:}
                \begin{itemize}
                    \item El endpoint debe ser seguro y validar el tipo de archivo.
                    \item Es necesario implementar un límite de tamaño máximo de archivo.
                \end{itemize}
            }\\
            \hline
        \end{tabularx}
    \end{center}
\end{table}

% HU04: Subida de vídeos
\begin{table}[H]
    \begin{center}
        \begin{tabularx}{\textwidth}{|l|X|l|}
            \hline
            \textbf{Identificador HU04} &
            \textbf{Como usuario, quiero subir vídeos desde mi móvil para tener una copia de seguridad en mi servidor} &
            \textbf{Estimación: 5 PH}\\
            \hline
            \multicolumn{3}{|p{\textwidth}|}{
                \begin{minipage}{\textwidth}
                    \centering
                    \vspace{0.5em}
                    \begin{tabular}{|l|p{8cm}|r|}
                        \hline
                        \textbf{Identificador} & \textbf{Título de la tarea de desarrollo} & \makecell{\textbf{Estimación}\\\textbf{(h)}} \\
                        \hline
                        Tarea 04-1 & Implementar endpoint de subida de vídeos & 2 \\
                        \hline
                        Tarea 04-2 & Validar y almacenar vídeos recibidos & 1.5 \\
                        \hline
                        Tarea 04-3 & Integrar con sistema de almacenamiento & 1 \\
                        \hline
                        Tarea 04-4 & Pruebas unitarias y de integración & 1 \\
                        \hline
                        Tarea 04-5 & Documentar endpoint & 0.5 \\
                        \hline
                    \end{tabular}
                    \vspace{0.5em}
                \end{minipage}
            } \\
            \hline
            \multicolumn{3}{|p{\textwidth}|}{
                \textbf{Pruebas de aceptación:}
                \begin{itemize}
                    \item El usuario puede subir uno o varios vídeos desde el móvil.
                    \item Los archivos se almacenan correctamente en el servidor.
                    \item El endpoint rechaza archivos no válidos.
                \end{itemize}
            }\\
            \hline
            \multicolumn{3}{|p{\textwidth}|}{
                \textbf{Observaciones:}
                \begin{itemize}
                    \item El endpoint debe ser seguro y validar el tipo de archivo.
                    \item Es necesario implementar un límite de tamaño máximo de archivo.
                \end{itemize}
            }\\
            \hline
        \end{tabularx}
    \end{center}
\end{table}

% HT09: Compresión de imágenes
\begin{table}[H]
    \begin{center}
        \begin{tabularx}{\textwidth}{|l|X|l|}
            \hline
            \textbf{Identificador HT09} &
            \textbf{Comprimir imágenes tras la subida para optimizar almacenamiento y ancho de banda} &
            \textbf{Estimación: 5 PH}\\
            \hline
            \multicolumn{3}{|p{\textwidth}|}{
                \begin{minipage}{\textwidth}
                    \centering
                    \vspace{0.5em}
                    \begin{tabular}{|l|p{8cm}|r|}
                        \hline
                        \textbf{Identificador} & \textbf{Título de la tarea de desarrollo} & \makecell{\textbf{Estimación}\\\textbf{(h)}} \\
                        \hline
                        Tarea 09-1T & Investigar y seleccionar librería de compresión & 1 \\
                        \hline
                        Tarea 09-2T & Implementar lógica de compresión tras subida & 2 \\
                        \hline
                        Tarea 09-3T & Pruebas de compresión y calidad & 1 \\
                        \hline
                        Tarea 09-4T & Manejo de errores y logs & 0.5 \\
                        \hline
                        Tarea 09-5T & Documentar proceso & 0.5 \\
                        \hline
                    \end{tabular}
                    \vspace{0.5em}
                \end{minipage}
            } \\
            \hline
            \multicolumn{3}{|p{\textwidth}|}{
                \textbf{Pruebas de aceptación:}
                \begin{itemize}
                    \item Las imágenes subidas se comprimen automáticamente.
                    \item La calidad de las imágenes comprimidas es aceptable.
                    \item El proceso de compresión no bloquea la subida.
                \end{itemize}
            }\\
            \hline
            \multicolumn{3}{|p{\textwidth}|}{
                \textbf{Observaciones:}
                \begin{itemize}
                    \item Seleccionar un balance adecuado entre compresión y calidad.
                \end{itemize}
            }\\
            \hline
        \end{tabularx}
    \end{center}
\end{table}

% HT10: Subida concurrente
\begin{table}[H]
    \begin{center}
        \begin{tabularx}{\textwidth}{|l|X|l|}
            \hline
            \textbf{Identificador HT10} &
            \textbf{Permitir la subida concurrente de archivos para mejorar la eficiencia} &
            \textbf{Estimación: 8 PH}\\
            \hline
            \multicolumn{3}{|p{\textwidth}|}{
                \begin{minipage}{\textwidth}
                    \centering
                    \vspace{0.5em}
                    \begin{tabular}{|l|p{8cm}|r|}
                        \hline
                        \textbf{Identificador} & \textbf{Título de la tarea de desarrollo} & \makecell{\textbf{Estimación}\\\textbf{(h)}} \\
                        \hline
                        Tarea 10-1 & Diseñar arquitectura para subida concurrente & 1.5 \\
                        \hline
                        Tarea 10-2 & Implementar manejo de múltiples subidas simultáneas & 2 \\
                        \hline
                        Tarea 10-3 & Control de concurrencia y límites & 1.5 \\
                        \hline
                        Tarea 10-4 & Pruebas de estrés y concurrencia & 1.5 \\
                        \hline
                        Tarea 10-5 & Logs y métricas & 0.5 \\
                        \hline
                        Tarea 10-6 & Documentar solución & 0.5 \\
                        \hline
                    \end{tabular}
                    \vspace{0.5em}
                \end{minipage}
            } \\
            \hline
            \multicolumn{3}{|p{\textwidth}|}{
                \textbf{Pruebas de aceptación:}
                \begin{itemize}
                    \item El servidor acepta varias subidas simultáneas sin errores.
                    \item Se limita el número de subidas concurrentes según configuración.
                    \item El rendimiento mejora respecto a la subida secuencial.
                \end{itemize}
            }\\
            \hline
            \multicolumn{3}{|p{\textwidth}|}{
                \textbf{Observaciones:}
                \begin{itemize}
                    \item Es importante evitar bloqueos y condiciones de carrera.
                \end{itemize}
            }\\
            \hline
        \end{tabularx}
    \end{center}
\end{table}

% HU09: Galería visual
\begin{table}[H]
    \begin{center}
        \begin{tabularx}{\textwidth}{|l|X|l|}
            \hline
            \textbf{Identificador HU09} &
            \textbf{Como usuario, quiero ver una galería online de mis archivos subidos} &
            \textbf{Estimación: 5 PH}\\
            \hline
            \multicolumn{3}{|p{\textwidth}|}{
                \begin{minipage}{\textwidth}
                    \centering
                    \vspace{0.5em}
                    \begin{tabular}{|l|p{8cm}|r|}
                        \hline
                        \textbf{Identificador} & \textbf{Título de la tarea de desarrollo} & \makecell{\textbf{Estimación}\\\textbf{(h)}} \\
                        \hline
                        Tarea 09-1U & Implementar endpoint para listar archivos multimedia & 1.5 \\
                        \hline
                        Tarea 09-2U & Generar miniaturas para galería & 1.5 \\
                        \hline
                        Tarea 09-3U & Implementar paginación/búsqueda básica & 1 \\
                        \hline
                        Tarea 09-4U & Pruebas de visualización & 0.5 \\
                        \hline
                        Tarea 09-5U & Documentar endpoint & 0.5 \\
                        \hline
                    \end{tabular}
                    \vspace{0.5em}
                \end{minipage}
            } \\
            \hline
            \multicolumn{3}{|p{\textwidth}|}{
                \textbf{Pruebas de aceptación:}
                \begin{itemize}
                    \item El usuario puede ver una galería online de sus archivos subidos.
                    \item Las miniaturas se generan correctamente.
                    \item La galería soporta paginación o búsqueda básica.
                \end{itemize}
            }\\
            \hline
            \multicolumn{3}{|p{\textwidth}|}{
                \textbf{Observaciones:}
                \begin{itemize}
                    \item La galería debe ser eficiente y escalable.
                \end{itemize}
            }\\
            \hline
        \end{tabularx}
    \end{center}
\end{table}

% HT17: Notificaciones de progreso
\begin{table}[H]
    \begin{center}
        \begin{tabularx}{\textwidth}{|l|X|l|}
            \hline
            \textbf{Identificador HT17} &
            \textbf{Notificar al cliente el progreso de la subida de archivos} &
            \textbf{Estimación: 3 PH}\\
            \hline
            \multicolumn{3}{|p{\textwidth}|}{
                \begin{minipage}{\textwidth}
                    \centering
                    \vspace{0.5em}
                    \begin{tabular}{|l|p{8cm}|r|}
                        \hline
                        \textbf{Identificador} & \textbf{Título de la tarea de desarrollo} & \makecell{\textbf{Estimación}\\\textbf{(h)}} \\
                        \hline
                        Tarea 17-1 & Implementar sistema de notificaciones de progreso (API) & 1 \\
                        \hline
                        Tarea 17-2 & Integrar con endpoints de subida & 0.5 \\
                        \hline
                        Tarea 17-3 & Pruebas y logs & 0.5 \\
                        \hline
                        Tarea 17-4 & Documentar & 0.5 \\
                        \hline
                    \end{tabular}
                    \vspace{0.5em}
                \end{minipage}
            } \\
            \hline
            \multicolumn{3}{|p{\textwidth}|}{
                \textbf{Pruebas de aceptación:}
                \begin{itemize}
                    \item El cliente recibe notificaciones de progreso durante la subida.
                    \item El sistema informa correctamente de errores o finalización.
                \end{itemize}
            }\\
            \hline
            \multicolumn{3}{|p{\textwidth}|}{
                \textbf{Observaciones:}
                \begin{itemize}
                    \item Puede usarse WebSocket, SSE o polling según la arquitectura.
                \end{itemize}
            }\\
            \hline
        \end{tabularx}
    \end{center}
\end{table}

% HU16: Seleccionar fotos (móvil)
\begin{table}[H]
    \begin{center}
        \begin{tabularx}{\textwidth}{|l|X|l|}
            \hline
            \textbf{Identificador HU16} &
            \textbf{Como usuario, quiero seleccionar fotos desde la galería del móvil para subirlas al servidor} &
            \textbf{Estimación: 3 PH}\\
            \hline
            \multicolumn{3}{|p{\textwidth}|}{
                \begin{minipage}{\textwidth}
                    \centering
                    \vspace{0.5em}
                    \begin{tabular}{|l|p{8cm}|r|}
                        \hline
                        \textbf{Identificador} & \textbf{Título de la tarea de desarrollo} & \makecell{\textbf{Estimación}\\\textbf{(h)}} \\
                        \hline
                        Tarea 16-1 & Implementar selector de fotos en el cliente móvil & 1.5 \\
                        \hline
                        Tarea 16-2 & Integrar con permisos del sistema & 1 \\
                        \hline
                        Tarea 16-3 & Pruebas en dispositivos reales & 0.5 \\
                        \hline
                    \end{tabular}
                    \vspace{0.5em}
                \end{minipage}
            } \\
            \hline
            \multicolumn{3}{|p{\textwidth}|}{
                \textbf{Pruebas de aceptación:}
                \begin{itemize}
                    \item El usuario puede seleccionar una o varias fotos desde la galería.
                    \item Se solicitan los permisos necesarios solo cuando es necesario.
                \end{itemize}
            }\\
            \hline
            \multicolumn{3}{|p{\textwidth}|}{
                \textbf{Observaciones:}
            }\\
            \hline
        \end{tabularx}
    \end{center}
\end{table}

% HU27: Subir vídeos (móvil)
\begin{table}[H]
    \begin{center}
        \begin{tabularx}{\textwidth}{|l|X|l|}
            \hline
            \textbf{Identificador HU27} &
            \textbf{Como usuario, quiero seleccionar y subir vídeos desde el móvil al servidor} &
            \textbf{Estimación: 5 PH}\\
            \hline
            \multicolumn{3}{|p{\textwidth}|}{
                \begin{minipage}{\textwidth}
                    \centering
                    \vspace{0.5em}
                    \begin{tabular}{|l|p{8cm}|r|}
                        \hline
                        \textbf{Identificador} & \textbf{Título de la tarea de desarrollo} & \makecell{\textbf{Estimación}\\\textbf{(h)}} \\
                        \hline
                        Tarea 27-1 & Implementar selector de vídeos en el cliente móvil & 1 \\
                        \hline
                        Tarea 27-2 & Lógica de subida de vídeos (cliente) & 1.5 \\
                        \hline
                        Tarea 27-3 & Manejo de errores y reintentos & 1 \\
                        \hline
                        Tarea 27-4 & Pruebas en dispositivos reales & 0.5 \\
                        \hline
                        Tarea 27-5 & Documentar flujo & 0.5 \\
                        \hline
                    \end{tabular}
                    \vspace{0.5em}
                \end{minipage}
            } \\
            \hline
            \multicolumn{3}{|p{\textwidth}|}{
                \textbf{Pruebas de aceptación:}
                \begin{itemize}
                    \item El usuario puede seleccionar y subir uno o varios vídeos.
                    \item El sistema maneja correctamente errores de red y reintentos.
                \end{itemize}
            }\\
            \hline
            \multicolumn{3}{|p{\textwidth}|}{
                \textbf{Observaciones:}
                \begin{itemize}
                    \item Probar con vídeos de diferentes tamaños y formatos.
                \end{itemize}
            }\\
            \hline
        \end{tabularx}
    \end{center}
\end{table}

% HU18: Ver progreso de subida (móvil)
\begin{table}[H]
    \begin{center}
        \begin{tabularx}{\textwidth}{|l|X|l|}
            \hline
            \textbf{Identificador HU18} &
            \textbf{Como usuario, quiero ver el progreso de la subida de archivos en la app móvil} &
            \textbf{Estimación: 8 PH}\\
            \hline
            \multicolumn{3}{|p{\textwidth}|}{
                \begin{minipage}{\textwidth}
                    \centering
                    \vspace{0.5em}
                    \begin{tabular}{|l|p{8cm}|r|}
                        \hline
                        \textbf{Identificador} & \textbf{Título de la tarea de desarrollo} & \makecell{\textbf{Estimación}\\\textbf{(h)}} \\
                        \hline
                        Tarea 18-1 & Implementar barra/indicador de progreso en UI & 1.5 \\
                        \hline
                        Tarea 18-2 & Integrar con notificaciones de backend & 1.5 \\
                        \hline
                        Tarea 18-3 & Actualización en tiempo real del progreso & 1.5 \\
                        \hline
                        Tarea 18-4 & Pruebas de UX & 1 \\
                        \hline
                        Tarea 18-5 & Manejo de errores y estados & 1 \\
                        \hline
                        Tarea 18-6 & Documentar & 0.5 \\
                        \hline
                    \end{tabular}
                    \vspace{0.5em}
                \end{minipage}
            } \\
            \hline
            \multicolumn{3}{|p{\textwidth}|}{
                \textbf{Pruebas de aceptación:}
                \begin{itemize}
                    \item El usuario ve el progreso de cada archivo subido en tiempo real.
                    \item El sistema informa correctamente de errores o subidas completadas.
                \end{itemize}
            }\\
            \hline
            \multicolumn{3}{|p{\textwidth}|}{
                \textbf{Observaciones:}
                \begin{itemize}
                    \item Probar en diferentes dispositivos y condiciones de red.
                \end{itemize}
            }\\
            \hline
        \end{tabularx}
    \end{center}
\end{table}

\subsection{Diagrama de Gantt}
Una vez definidas las tareas para cada historia de usuario, se ha elaborado un diagrama de Gantt para visualizar la planificación del sprint. Este diagrama muestra el orden de las tareas y su duración estimada:

\begin{figure}[H]
    \begin{center}
        \includegraphics[width=0.8\textwidth]{assets/sprint3/week1-gantt.png}
    \end{center}
    \caption{Diagrama de Gantt de las tareas de la primera semana del sprint 3}\label{fig:gantt-sprint3-week1}
\end{figure}


\begin{figure}[H]
    \begin{center}
        \includegraphics[width=0.8\textwidth]{assets/sprint3/week2-gantt.png}
    \end{center}
    \caption{Diagrama de Gantt de las tareas de la segunda semana del sprint 3}\label{fig:gantt-sprint3-week2}
\end{figure}

En este sprint se han priorizado las tareas relacionadas principalmente con el procesado multimedia en el servidor, dado que en el anterior sprint el enfoque estuvo en el desarrollo de la aplicación móvil.

Las tareas relacionadas con la aplicación móvil de este sprint se centran principalmente en integrar los cambios implementados en el servidor.
Se realiza de esta manera para que al finalizar el sprint 3 tengamos un producto con más valor, puesto que el usuario tendrá la posibilidad de subir fotos y vídeos desde su móvil, que serán procesados en el servidor y podrán visualizarse en una galería online.

\subsection{Diseño detallado e implementación}
\paragraph{Subida de archivos multimedia}
\subparagraph{}

Se ha realizado la implementación de la subida de archivos multimedia (fotos y vídeos) desde la aplicación móvil al servidor, así como el procesado de estos archivos en el servidor (compresión de imágenes, generación de miniaturas) y la visualización en una galería online.

El flujo que se ha seguido en la implementación de la subida de los archivos multimedia ha sido el siguiente:
\begin{figure}[H]
    \begin{center}
        \includegraphics[width=0.95\textwidth]{assets/sprint3/diagrama-subida-archivos.png}
    \end{center}
    \caption{Diagrama de flujo de subida de un archivo multimedia}\label{fig:diagrama-flujo-subida-archivos}
\end{figure}

Como se muestra en la Figura \ref{fig:diagrama-flujo-subida-archivos}, el usuario selecciona la foto o video que desea subir desde la aplicación móvil. La aplicación móvil envía los archivos al servidor a través de un endpoint de subida. El servidor recibe el archivo y lo procesa sin alojar todo el contenido del mismo en memoria, consiguiendo así un procesamiento eficiente. A continuación, el servidor procesa el archivo y genera una miniatura. Finalmente, los archivos procesados se almacenan en el sistema de almacenamiento definitivo (MinIO) y el usuario puede ver sus archivos en una galería online, es decir, no necesita tener los archivos en su dispositivo para poder visualizarlos.

La generación de miniaturas se realiza de manera paralela, por lo que no bloquea la subida de archivos.

\paragraph{Manejo de subidas concurrentes}
\subparagraph{}

Durante el desarrollo se realizaron varias pruebas de estrés de subida de diferentes números de archivos (100, 1000, 20000) para asegurar que el servidor podía manejar múltiples subidas concurrentes sin errores ni bloqueos. Se implementó un sistema de control de concurrencia para limitar el número de subidas simultáneas, evitando así la sobrecarga del servidor.
Para la gestión de subidas concurrentes se ha definido un número máximo de peticiones que pueden ser atendidas de forma simultánea. Este ajuste es configurable a través de una variable de entorno, permitiendo así adaptar el comportamiento del servidor según la capacidad del entorno de despliegue.

El principal problema que se encontró durante las pruebas de estrés fue la saturación del servidor al recibir demasiadas peticiones simultáneas de subida de archivos. El problema que se encontró fue en la memoria, pues para cada archivo que se quería subir se tenía que alojar una cantidad de memoria para cada petición, bloqueando el servidor para muchas peticiones concurrentes. Para solucionar este problema, se implementó un sistema de semáforos que gestiona las peticiones entrantes y establece un límite de peticiones que pueden ejecutarse en paralelo, asegurando así que el servidor no se sobrecargue.

\subsection{Extensión del sprint y retos técnicos}

Durante el desarrollo del Sprint 3 se encontraron dificultades técnicas significativas que requirieron una semana adicional para completar las funcionalidades planificadas. Los principales retos surgieron por las limitaciones de la tecnología LynxJS utilizada en el desarrollo de la aplicación móvil, especialmente en el manejo de subidas de archivos.

\paragraph{Limitaciones críticas de LynxJS}
\subparagraph{}

La principal limitación técnica encontrada fue la **ausencia de soporte nativo para subidas multipart** en LynxJS. Esta carencia impactó severamente en las siguientes áreas:

\begin{itemize}
    \item \textbf{Subidas multipart nativas}: LynxJS no proporciona APIs para manejar subidas de archivos multipart, obligando a implementar toda la funcionalidad de transferencia de archivos desde cero utilizando módulos nativos del sistema operativo.
    
    \item \textbf{Tracking de progreso de subida}: La falta de soporte multipart imposibilita el seguimiento granular del progreso de subida, ya que las implementaciones nativas requeridas no exponen interfaces compatibles con el sistema de eventos de LynxJS.
    
    \item \textbf{Gestión de archivos grandes}: El manejo de vídeos y archivos multimedia de gran tamaño requiere implementaciones nativas complejas que van más allá del scope actual del framework.
\end{itemize}

\paragraph{Historia de usuario no completada}
\subparagraph{}

Debido a estas limitaciones técnicas, la historia HU18 (Ver progreso de subida) no pudo ser completada en este sprint. Los motivos específicos son:

\begin{itemize}
    \item La implementación nativa de subidas multipart consume todo el tiempo estimado sin permitir la integración del sistema de progreso.
    \item El desarrollo de un puente de comunicación entre los módulos nativos y LynxJS para el tracking de progreso requiere un análisis arquitectónico más profundo.
    \item Las pruebas de integración entre componentes nativos y el framework presentan complejidades adicionales no contempladas inicialmente.
\end{itemize}

Esta funcionalidad se ha pospuesto para el Sprint 4, donde se dedicará tiempo específico a desarrollar una arquitectura robusta que permita la comunicación bidireccional entre módulos nativos y LynxJS.

\paragraph{Impacto en el cronograma}
\subparagraph{}

El tiempo adicional de una semana se destinó a:
\begin{enumerate}
    \item Investigación de limitaciones de LynxJS para subidas multipart (8 horas)
    \item Implementación completa de módulos nativos para subida de archivos (16 horas)
    \item Refactorización de HU16 y HU27 para usar implementaciones nativas (12 horas)
    \item Pruebas de integración y compatibilidad (8 horas)
\end{enumerate}

A pesar de la extensión, se ha logrado entregar un incremento funcional que permite la subida de fotos y vídeos, cumpliendo con los objetivos principales del sprint. La funcionalidad de progreso de subida se abordará en el siguiente sprint con una planificación específica para superar las limitaciones arquitectónicas identificadas.
