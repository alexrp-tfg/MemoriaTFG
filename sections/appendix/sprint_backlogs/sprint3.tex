\section{Sprint 3}
\label{appendix:sprint3-backlog}
% HU01: Subida de fotos
\begin{table}[H]
    \begin{center}
        \begin{tabularx}{\textwidth}{|l|X|l|}
            \hline
            \textbf{Identificador HU01} &
            \textbf{Como usuario, quiero subir varias fotos desde mi móvil para tener una copia de seguridad en mi servidor} &
            \textbf{Estimación: 5 PH}\\
            \hline
            \multicolumn{3}{|p{\textwidth}|}{
                \begin{minipage}{\textwidth}
                    \centering
                    \vspace{0.5em}
                    \begin{tabular}{|l|p{8cm}|r|}
                        \hline
                        \textbf{Identificador} & \textbf{Título de la tarea de desarrollo} & \makecell{\textbf{Estimación}\\\textbf{(h)}} \\
                        \hline
                        Tarea 01-1 & Implementar endpoint de subida de fotos (backend) & 2 \\
                        \hline
                        Tarea 01-2 & Validar y almacenar archivos recibidos & 1.5 \\
                        \hline
                        Tarea 01-3 & Integrar con sistema de almacenamiento (local o cloud) & 1 \\
                        \hline
                        Tarea 01-4 & Pruebas unitarias y de integración & 1 \\
                        \hline
                        Tarea 01-5 & Documentar endpoint & 0.5 \\
                        \hline
                    \end{tabular}
                    \vspace{0.5em}
                \end{minipage}
            } \\
            \hline
            \multicolumn{3}{|p{\textwidth}|}{
                \textbf{Pruebas de aceptación:}
                \begin{itemize}
                    \item El usuario puede subir una o varias fotos desde el móvil.
                    \item Los archivos se almacenan correctamente en el servidor.
                    \item El endpoint rechaza archivos no válidos.
                \end{itemize}
            }\\
            \hline
            \multicolumn{3}{|p{\textwidth}|}{
                \textbf{Observaciones:}
                \begin{itemize}
                    \item El endpoint debe ser seguro y validar el tipo de archivo.
                    \item Es necesario implementar un límite de tamaño máximo de archivo.
                \end{itemize}
            }\\
            \hline
        \end{tabularx}
    \end{center}
\end{table}

% HU04: Subida de vídeos
\begin{table}[H]
    \begin{center}
        \begin{tabularx}{\textwidth}{|l|X|l|}
            \hline
            \textbf{Identificador HU04} &
            \textbf{Como usuario, quiero subir vídeos desde mi móvil para tener una copia de seguridad en mi servidor} &
            \textbf{Estimación: 5 PH}\\
            \hline
            \multicolumn{3}{|p{\textwidth}|}{
                \begin{minipage}{\textwidth}
                    \centering
                    \vspace{0.5em}
                    \begin{tabular}{|l|p{8cm}|r|}
                        \hline
                        \textbf{Identificador} & \textbf{Título de la tarea de desarrollo} & \makecell{\textbf{Estimación}\\\textbf{(h)}} \\
                        \hline
                        Tarea 04-1 & Implementar endpoint de subida de vídeos & 2 \\
                        \hline
                        Tarea 04-2 & Validar y almacenar vídeos recibidos & 1.5 \\
                        \hline
                        Tarea 04-3 & Integrar con sistema de almacenamiento & 1 \\
                        \hline
                        Tarea 04-4 & Pruebas unitarias y de integración & 1 \\
                        \hline
                        Tarea 04-5 & Documentar endpoint & 0.5 \\
                        \hline
                    \end{tabular}
                    \vspace{0.5em}
                \end{minipage}
            } \\
            \hline
            \multicolumn{3}{|p{\textwidth}|}{
                \textbf{Pruebas de aceptación:}
                \begin{itemize}
                    \item El usuario puede subir uno o varios vídeos desde el móvil.
                    \item Los archivos se almacenan correctamente en el servidor.
                    \item El endpoint rechaza archivos no válidos.
                \end{itemize}
            }\\
            \hline
            \multicolumn{3}{|p{\textwidth}|}{
                \textbf{Observaciones:}
                \begin{itemize}
                    \item El endpoint debe ser seguro y validar el tipo de archivo.
                    \item Es necesario implementar un límite de tamaño máximo de archivo.
                \end{itemize}
            }\\
            \hline
        \end{tabularx}
    \end{center}
\end{table}

% HT09: Compresión de imágenes
\begin{table}[H]
    \begin{center}
        \begin{tabularx}{\textwidth}{|l|X|l|}
            \hline
            \textbf{Identificador HT09} &
            \textbf{Comprimir imágenes tras la subida para optimizar almacenamiento y ancho de banda} &
            \textbf{Estimación: 5 PH}\\
            \hline
            \multicolumn{3}{|p{\textwidth}|}{
                \begin{minipage}{\textwidth}
                    \centering
                    \vspace{0.5em}
                    \begin{tabular}{|l|p{8cm}|r|}
                        \hline
                        \textbf{Identificador} & \textbf{Título de la tarea de desarrollo} & \makecell{\textbf{Estimación}\\\textbf{(h)}} \\
                        \hline
                        Tarea 09-1T & Investigar y seleccionar librería de compresión & 1 \\
                        \hline
                        Tarea 09-2T & Implementar lógica de compresión tras subida & 2 \\
                        \hline
                        Tarea 09-3T & Pruebas de compresión y calidad & 1 \\
                        \hline
                        Tarea 09-4T & Manejo de errores y logs & 0.5 \\
                        \hline
                        Tarea 09-5T & Documentar proceso & 0.5 \\
                        \hline
                    \end{tabular}
                    \vspace{0.5em}
                \end{minipage}
            } \\
            \hline
            \multicolumn{3}{|p{\textwidth}|}{
                \textbf{Pruebas de aceptación:}
                \begin{itemize}
                    \item Las imágenes subidas se comprimen automáticamente.
                    \item La calidad de las imágenes comprimidas es aceptable.
                    \item El proceso de compresión no bloquea la subida.
                \end{itemize}
            }\\
            \hline
            \multicolumn{3}{|p{\textwidth}|}{
                \textbf{Observaciones:}
                \begin{itemize}
                    \item Seleccionar un balance adecuado entre compresión y calidad.
                \end{itemize}
            }\\
            \hline
        \end{tabularx}
    \end{center}
\end{table}

% HT10: Subida concurrente
\begin{table}[H]
    \begin{center}
        \begin{tabularx}{\textwidth}{|l|X|l|}
            \hline
            \textbf{Identificador HT10} &
            \textbf{Permitir la subida concurrente de archivos para mejorar la eficiencia} &
            \textbf{Estimación: 8 PH}\\
            \hline
            \multicolumn{3}{|p{\textwidth}|}{
                \begin{minipage}{\textwidth}
                    \centering
                    \vspace{0.5em}
                    \begin{tabular}{|l|p{8cm}|r|}
                        \hline
                        \textbf{Identificador} & \textbf{Título de la tarea de desarrollo} & \makecell{\textbf{Estimación}\\\textbf{(h)}} \\
                        \hline
                        Tarea 10-1 & Diseñar arquitectura para subida concurrente & 1.5 \\
                        \hline
                        Tarea 10-2 & Implementar manejo de múltiples subidas simultáneas & 2 \\
                        \hline
                        Tarea 10-3 & Control de concurrencia y límites & 1.5 \\
                        \hline
                        Tarea 10-4 & Pruebas de estrés y concurrencia & 1.5 \\
                        \hline
                        Tarea 10-5 & Logs y métricas & 0.5 \\
                        \hline
                        Tarea 10-6 & Documentar solución & 0.5 \\
                        \hline
                    \end{tabular}
                    \vspace{0.5em}
                \end{minipage}
            } \\
            \hline
            \multicolumn{3}{|p{\textwidth}|}{
                \textbf{Pruebas de aceptación:}
                \begin{itemize}
                    \item El servidor acepta varias subidas simultáneas sin errores.
                    \item Se limita el número de subidas concurrentes según configuración.
                    \item El rendimiento mejora respecto a la subida secuencial.
                \end{itemize}
            }\\
            \hline
            \multicolumn{3}{|p{\textwidth}|}{
                \textbf{Observaciones:}
                \begin{itemize}
                    \item Es importante evitar bloqueos y condiciones de carrera.
                \end{itemize}
            }\\
            \hline
        \end{tabularx}
    \end{center}
\end{table}

% HU09: Galería visual
\begin{table}[H]
    \begin{center}
        \begin{tabularx}{\textwidth}{|l|X|l|}
            \hline
            \textbf{Identificador HU09} &
            \textbf{Como usuario, quiero ver una galería online de mis archivos subidos} &
            \textbf{Estimación: 5 PH}\\
            \hline
            \multicolumn{3}{|p{\textwidth}|}{
                \begin{minipage}{\textwidth}
                    \centering
                    \vspace{0.5em}
                    \begin{tabular}{|l|p{8cm}|r|}
                        \hline
                        \textbf{Identificador} & \textbf{Título de la tarea de desarrollo} & \makecell{\textbf{Estimación}\\\textbf{(h)}} \\
                        \hline
                        Tarea 09-1U & Implementar endpoint para listar archivos multimedia & 1.5 \\
                        \hline
                        Tarea 09-2U & Generar miniaturas para galería & 1.5 \\
                        \hline
                        Tarea 09-3U & Implementar paginación/búsqueda básica & 1 \\
                        \hline
                        Tarea 09-4U & Pruebas de visualización & 0.5 \\
                        \hline
                        Tarea 09-5U & Documentar endpoint & 0.5 \\
                        \hline
                    \end{tabular}
                    \vspace{0.5em}
                \end{minipage}
            } \\
            \hline
            \multicolumn{3}{|p{\textwidth}|}{
                \textbf{Pruebas de aceptación:}
                \begin{itemize}
                    \item El usuario puede ver una galería online de sus archivos subidos.
                    \item Las miniaturas se generan correctamente.
                    \item La galería soporta paginación o búsqueda básica.
                \end{itemize}
            }\\
            \hline
            \multicolumn{3}{|p{\textwidth}|}{
                \textbf{Observaciones:}
                \begin{itemize}
                    \item La galería debe ser eficiente y escalable.
                \end{itemize}
            }\\
            \hline
        \end{tabularx}
    \end{center}
\end{table}

% HT17: Notificaciones de progreso
\begin{table}[H]
    \begin{center}
        \begin{tabularx}{\textwidth}{|l|X|l|}
            \hline
            \textbf{Identificador HT17} &
            \textbf{Notificar al cliente el progreso de la subida de archivos} &
            \textbf{Estimación: 3 PH}\\
            \hline
            \multicolumn{3}{|p{\textwidth}|}{
                \begin{minipage}{\textwidth}
                    \centering
                    \vspace{0.5em}
                    \begin{tabular}{|l|p{8cm}|r|}
                        \hline
                        \textbf{Identificador} & \textbf{Título de la tarea de desarrollo} & \makecell{\textbf{Estimación}\\\textbf{(h)}} \\
                        \hline
                        Tarea 17-1 & Implementar sistema de notificaciones de progreso (API) & 1 \\
                        \hline
                        Tarea 17-2 & Integrar con endpoints de subida & 0.5 \\
                        \hline
                        Tarea 17-3 & Pruebas y logs & 0.5 \\
                        \hline
                        Tarea 17-4 & Documentar & 0.5 \\
                        \hline
                    \end{tabular}
                    \vspace{0.5em}
                \end{minipage}
            } \\
            \hline
            \multicolumn{3}{|p{\textwidth}|}{
                \textbf{Pruebas de aceptación:}
                \begin{itemize}
                    \item El cliente recibe notificaciones de progreso durante la subida.
                    \item El sistema informa correctamente de errores o finalización.
                \end{itemize}
            }\\
            \hline
            \multicolumn{3}{|p{\textwidth}|}{
                \textbf{Observaciones:}
                \begin{itemize}
                    \item Puede usarse WebSocket, SSE o polling según la arquitectura.
                \end{itemize}
            }\\
            \hline
        \end{tabularx}
    \end{center}
\end{table}

% HU16: Seleccionar fotos (móvil)
\begin{table}[H]
    \begin{center}
        \begin{tabularx}{\textwidth}{|l|X|l|}
            \hline
            \textbf{Identificador HU16} &
            \textbf{Como usuario, quiero seleccionar fotos desde la galería del móvil para subirlas al servidor} &
            \textbf{Estimación: 3 PH}\\
            \hline
            \multicolumn{3}{|p{\textwidth}|}{
                \begin{minipage}{\textwidth}
                    \centering
                    \vspace{0.5em}
                    \begin{tabular}{|l|p{8cm}|r|}
                        \hline
                        \textbf{Identificador} & \textbf{Título de la tarea de desarrollo} & \makecell{\textbf{Estimación}\\\textbf{(h)}} \\
                        \hline
                        Tarea 16-1 & Implementar selector de fotos en el cliente móvil & 1.5 \\
                        \hline
                        Tarea 16-2 & Integrar con permisos del sistema & 1 \\
                        \hline
                        Tarea 16-3 & Pruebas en dispositivos reales & 0.5 \\
                        \hline
                    \end{tabular}
                    \vspace{0.5em}
                \end{minipage}
            } \\
            \hline
            \multicolumn{3}{|p{\textwidth}|}{
                \textbf{Pruebas de aceptación:}
                \begin{itemize}
                    \item El usuario puede seleccionar una o varias fotos desde la galería.
                    \item Se solicitan los permisos necesarios solo cuando es necesario.
                \end{itemize}
            }\\
            \hline
            \multicolumn{3}{|p{\textwidth}|}{
                \textbf{Observaciones:}
            }\\
            \hline
        \end{tabularx}
    \end{center}
\end{table}

% HU27: Subir vídeos (móvil)
\begin{table}[H]
    \begin{center}
        \begin{tabularx}{\textwidth}{|l|X|l|}
            \hline
            \textbf{Identificador HU27} &
            \textbf{Como usuario, quiero seleccionar y subir vídeos desde el móvil al servidor} &
            \textbf{Estimación: 5 PH}\\
            \hline
            \multicolumn{3}{|p{\textwidth}|}{
                \begin{minipage}{\textwidth}
                    \centering
                    \vspace{0.5em}
                    \begin{tabular}{|l|p{8cm}|r|}
                        \hline
                        \textbf{Identificador} & \textbf{Título de la tarea de desarrollo} & \makecell{\textbf{Estimación}\\\textbf{(h)}} \\
                        \hline
                        Tarea 27-1 & Implementar selector de vídeos en el cliente móvil & 1 \\
                        \hline
                        Tarea 27-2 & Lógica de subida de vídeos (cliente) & 1.5 \\
                        \hline
                        Tarea 27-3 & Manejo de errores y reintentos & 1 \\
                        \hline
                        Tarea 27-4 & Pruebas en dispositivos reales & 0.5 \\
                        \hline
                        Tarea 27-5 & Documentar flujo & 0.5 \\
                        \hline
                    \end{tabular}
                    \vspace{0.5em}
                \end{minipage}
            } \\
            \hline
            \multicolumn{3}{|p{\textwidth}|}{
                \textbf{Pruebas de aceptación:}
                \begin{itemize}
                    \item El usuario puede seleccionar y subir uno o varios vídeos.
                    \item El sistema maneja correctamente errores de red y reintentos.
                \end{itemize}
            }\\
            \hline
            \multicolumn{3}{|p{\textwidth}|}{
                \textbf{Observaciones:}
                \begin{itemize}
                    \item Probar con vídeos de diferentes tamaños y formatos.
                \end{itemize}
            }\\
            \hline
        \end{tabularx}
    \end{center}
\end{table}

% HU18: Ver progreso de subida (móvil)
\begin{table}[H]
    \begin{center}
        \begin{tabularx}{\textwidth}{|l|X|l|}
            \hline
            \textbf{Identificador HU18} &
            \textbf{Como usuario, quiero ver el progreso de la subida de archivos en la app móvil} &
            \textbf{Estimación: 8 PH}\\
            \hline
            \multicolumn{3}{|p{\textwidth}|}{
                \begin{minipage}{\textwidth}
                    \centering
                    \vspace{0.5em}
                    \begin{tabular}{|l|p{8cm}|r|}
                        \hline
                        \textbf{Identificador} & \textbf{Título de la tarea de desarrollo} & \makecell{\textbf{Estimación}\\\textbf{(h)}} \\
                        \hline
                        Tarea 18-1 & Implementar barra/indicador de progreso en UI & 1.5 \\
                        \hline
                        Tarea 18-2 & Integrar con notificaciones de backend & 1.5 \\
                        \hline
                        Tarea 18-3 & Actualización en tiempo real del progreso & 1.5 \\
                        \hline
                        Tarea 18-4 & Pruebas de UX & 1 \\
                        \hline
                        Tarea 18-5 & Manejo de errores y estados & 1 \\
                        \hline
                        Tarea 18-6 & Documentar & 0.5 \\
                        \hline
                    \end{tabular}
                    \vspace{0.5em}
                \end{minipage}
            } \\
            \hline
            \multicolumn{3}{|p{\textwidth}|}{
                \textbf{Pruebas de aceptación:}
                \begin{itemize}
                    \item El usuario ve el progreso de cada archivo subido en tiempo real.
                    \item El sistema informa correctamente de errores o subidas completadas.
                \end{itemize}
            }\\
            \hline
            \multicolumn{3}{|p{\textwidth}|}{
                \textbf{Observaciones:}
                \begin{itemize}
                    \item Probar en diferentes dispositivos y condiciones de red.
                \end{itemize}
            }\\
            \hline
        \end{tabularx}
    \end{center}
\end{table}
