\newpage
~
\newpage
\section{Estado del arte}
Qué se ha hecho hasta ahora en este campo, qué tecnologías se han utilizado, qué problemas se han encontrado, qué soluciones se han propuesto.

% TODO:  Hablar también del estudio de las posibles soluciones, aunque después se elija otra.
%
% Añadir trabajo relacionado, hablar sobre propuestas de otras personas.
%
% Incorporar comparación de las tecnologías planteadas.
% 1. **Inmediato**: Completar análisis de tecnologías móviles
% 2. **Corto plazo**: Añadir métricas cuantitativas y trabajo relacionado
% 3. **Medio plazo**: Desarrollar análisis de arquitecturas y contribución


En este apartado se presenta un análisis del estado del arte en el ámbito de las bibliotecas de archivos multimedia de código abierto (\acrshort{foss}).
Se examinan las principales soluciones disponibles, sus características técnicas, fortalezas y limitaciones, así como las tendencias actuales en el sector.
Dado que uno de los objetivos del proyecto es desarrollar un producto que sea de código abierto, la comparación se centra en soluciones FOSS que ya están en el mercado y que han sido ampliamente adoptadas por la comunidad, lo cual nos va a permitir desarrollar una comparación mas extensa sobre cómo están organizados los proyectos para facilitar su mantenimiento y escalabilidad, así como las tecnologías que utilizan para ofrecer sus servicios.

Además, se realiza un estudio sobre las tecnologías que vamos a utilizar en el proyecto en comparación con las alternativas y las que ya se utilizan en los proyectos existentes que se analizan.

En el panorama actual de las bibliotecas de archivos multimedia de código abierto, existe una amplia variedad de soluciones que buscan ofrecer alternativas libres y gratuitas a los servicios propietarios como Google Photos o iCloud. Este análisis del estado del arte se centra en las tres soluciones más populares según el número de estrellas en GitHub: Immich, PhotoPrism y Ente.

% Fuentes:
% Google Photos: https://photos.google.com/
% Apple Photos: https://www.apple.com/ios/photos/
% Amazon Photos: https://www.amazon.com/photos
% Microsoft OneDrive Photos: https://www.microsoft.com/en-us/microsoft-365/onedrive/online-cloud-storage
\subsection{Soluciones Propietarias Relevantes}

En el ámbito de la gestión y almacenamiento de fotografías, las soluciones propietarias han marcado el estándar en cuanto a experiencia de usuario, integración de servicios y capacidades avanzadas de inteligencia artificial. Entre las plataformas más destacadas se encuentran Google Photos, Apple Photos, Amazon Photos y Microsoft OneDrive Photos, cada una con un enfoque particular y funcionalidades diferenciadoras.

Google Photos sobresale por su motor de búsqueda semántica basado en inteligencia artificial, que permite localizar imágenes mediante descripciones textuales, reconocimiento automático de objetos, lugares y personas, así como la agrupación inteligente de rostros. Además, ofrece funciones como la creación automática de álbumes, recuerdos personalizados, sugerencias de edición y generación de vídeos y animaciones a partir de colecciones de fotos. La integración con Google Assistant permite búsquedas por voz y automatización de tareas relacionadas con la gestión de imágenes.

Apple Photos, por su parte, se integra de forma nativa en el ecosistema de dispositivos Apple, ofreciendo sincronización automática y segura a través de iCloud. Destaca por sus potentes herramientas de edición no destructiva, la organización automática mediante “Memories” y “People”, y la privacidad reforzada mediante el procesamiento local de datos sensibles, como el reconocimiento facial. La integración con Siri permite búsquedas contextuales y sugerencias inteligentes.

Amazon Photos ofrece almacenamiento ilimitado de fotografías en alta resolución para suscriptores de Amazon Prime, así como detección automática de duplicados y organización por personas, lugares y objetos. Su enfoque está orientado a la simplicidad y la capacidad de compartir álbumes de forma privada o pública, además de la integración con dispositivos Amazon Echo Show para visualización mediante comandos de voz.

Microsoft OneDrive Photos integra la gestión de imágenes con el resto de servicios de productividad de Microsoft 365, facilitando la colaboración y el acceso multiplataforma. Incluye funciones de etiquetado automático, búsqueda por contenido visual y organización cronológica, así como integración con herramientas de edición en línea y sincronización con dispositivos Windows y móviles.

Entre las características avanzadas que suelen estar más desarrolladas en estas soluciones propietarias, y que pueden servir de inspiración para el desarrollo de alternativas open-source, destacan:
\begin{itemize}
    \item \textbf{Búsqueda semántica avanzada}: Localización de imágenes mediante descripciones naturales, reconocimiento de escenas, objetos y personas.
    \item \textbf{Agrupación y etiquetado inteligente}: Detección y agrupación automática de rostros, lugares y eventos.
    \item \textbf{Generación automática de recuerdos y contenido}: Creación de álbumes, vídeos y animaciones personalizadas a partir de colecciones de fotos.
    \item \textbf{Integración con asistentes virtuales}: Búsqueda y gestión de imágenes mediante comandos de voz.
    \item \textbf{Sincronización y acceso multiplataforma}: Integración transparente con diferentes dispositivos y sistemas operativos.
    \item \textbf{Herramientas avanzadas de edición}: Edición no destructiva, sugerencias automáticas y filtros inteligentes.
    \item \textbf{Privacidad y control de datos}: Procesamiento local de información sensible y opciones avanzadas de control de acceso.
\end{itemize}

Si bien algunas de estas funcionalidades comienzan a estar presentes en proyectos de código abierto, la madurez, precisión y facilidad de uso de las implementaciones propietarias sigue siendo, en muchos casos, superior debido a la inversión en inteligencia artificial, recursos computacionales y la integración profunda con sus respectivos ecosistemas. La incorporación de estas capacidades en soluciones FOSS representa un reto y una oportunidad para cerrar la brecha funcional existente.

\subsection{Panorama General de Soluciones FOSS}

El ecosistema de bibliotecas de archivos multimedia de código abierto ha experimentado un crecimiento significativo en los últimos años, impulsado por las crecientes preocupaciones sobre la privacidad de los datos y la dependencia de servicios en la nube propietarios. Según el análisis comparativo realizado por Meichthys \parencite{meichthys2024}, existen más de 16 proyectos activos que ofrecen diferentes enfoques y características.

Las soluciones analizadas se pueden clasificar en tres categorías principales:
\begin{itemize}
    \item \textbf{Soluciones escalables}: Enfocadas en escalabilidad y características avanzadas
    \item \textbf{Soluciones centradas en privacidad}: Priorizan la seguridad y el cifrado
    \item \textbf{Soluciones ligeras}: Optimizadas para recursos limitados
\end{itemize}

\subsection{Análisis de las Tres Soluciones Principales}

\subsubsection{Immich}

(\cite{immich-documentation}) Immich es una solución de gestión de fotos de código abierto orientada a usuarios que buscan una alternativa privada y autoalojada a servicios comerciales como Google Photos. Su desarrollo comenzó en 2022 y ha experimentado un rápido crecimiento gracias a una comunidad activa y a la adopción de tecnologías modernas.

\textbf{Propósito y público objetivo:} Immich está diseñado para usuarios particulares, familias y pequeños equipos que desean mantener el control sobre sus fotos y vídeos, evitando la dependencia de servicios en la nube de terceros. Es especialmente atractivo para entusiastas de la tecnología y defensores de la privacidad.

\textbf{Historia y contexto:} El proyecto nació como respuesta a la falta de alternativas libres y modernas a los grandes servicios comerciales, con un enfoque en la experiencia de usuario y la facilidad de despliegue.

\textbf{Modelo de desarrollo:} Immich es mantenido principalmente por una comunidad de desarrolladores en GitHub, con contribuciones frecuentes y una hoja de ruta pública.

\textbf{Características funcionales:}
\begin{itemize}
    \item \textbf{Facilidad de uso:} Interfaz web moderna, intuitiva y responsiva, con aplicaciones móviles para Android e iOS.
    \item \textbf{Instalación y despliegue:} Ofrece imágenes Docker y documentación clara, facilitando la instalación en servidores personales o NAS.
    \item \textbf{Soporte multiplataforma:} Disponible en web y dispositivos móviles, con sincronización automática desde el móvil.
    \item \textbf{Idiomas disponibles:} Traducción a varios idiomas, incluyendo español, gracias a la colaboración comunitaria.
    \item \textbf{Tecnologías:} Backend en NestJS/Node.js, frontend en SvelteKit, móvil en Flutter, base de datos PostgreSQL y Redis, IA con TensorFlow.
\end{itemize}

\textbf{Comunidad y ecosistema:}
\begin{itemize}
    \item \textbf{Comunidad activa:} Foros, Discord y GitHub con alta participación.
    \item \textbf{Documentación:} Completa y en constante actualización.
    \item \textbf{Extensibilidad:} API pública y soporte para integraciones futuras.
\end{itemize}

\textbf{Seguridad y privacidad:}
\begin{itemize}
    \item \textbf{Gestión de datos personales:} Los datos permanecen en el servidor del usuario, sin envíos a terceros.
    \item \textbf{Cifrado:} Actualmente no implementa cifrado de extremo a extremo, pero sí buenas prácticas de seguridad en el almacenamiento y acceso.
    \item \textbf{Actualizaciones:} Lanzamientos frecuentes y respuesta rápida a vulnerabilidades.
\end{itemize}

\textbf{Casos de uso y ejemplos reales:}
\begin{itemize}
    \item Utilizado por usuarios domésticos y pequeñas organizaciones para gestionar colecciones fotográficas privadas.
    \item Referencias y testimonios positivos en foros de autoalojamiento y privacidad.
\end{itemize}

\textbf{Limitaciones actuales:}
\begin{itemize}
    \item Alto consumo de recursos debido a Node.js.
    \item API en constante evolución (su última versión aún no se considera estable).
    \item Compatibilidad limitada con carpetas existentes.
    \item Algunas funciones avanzadas (como la edición colaborativa o el reconocimiento facial avanzado) están en desarrollo.
    \item La integración con otros servicios y dispositivos aún es limitada en comparación con soluciones comerciales.
\end{itemize}

\subsubsection{PhotoPrism}

(\cite{photoprism-documentation}) PhotoPrism es una de las soluciones FOSS más maduras y populares para la gestión de fotos, con una comunidad consolidada y un enfoque en la organización eficiente y el respeto a la privacidad.

\textbf{Propósito y público objetivo:} Orientado a usuarios que buscan una alternativa autoalojada, robusta y fácil de usar para organizar grandes colecciones de fotos, con especial atención a la preservación de metadatos y la integración con sistemas existentes.

\textbf{Historia y contexto:} Lanzado en 2017, PhotoPrism ha evolucionado para convertirse en una referencia dentro del software libre de gestión fotográfica, con un desarrollo sostenido y una base de usuarios creciente.

\textbf{Modelo de desarrollo:} Proyecto comunitario con liderazgo claro, financiado parcialmente por donaciones y patrocinios.

\textbf{Características funcionales:}
\begin{itemize}
    \item \textbf{Facilidad de uso:} Interfaz web clara y funcional, con enfoque en la organización y búsqueda eficiente.
    \item \textbf{Instalación y despliegue:} Instalación sencilla mediante Docker o binarios, compatible con múltiples sistemas operativos.
    \item \textbf{Soporte multiplataforma:} Acceso desde cualquier navegador y soporte PWA para móviles.
    \item \textbf{Idiomas disponibles:} Traducción a numerosos idiomas.
    \item \textbf{Tecnologías:} Backend en Go, frontend en Vue.js, base de datos SQLite/MariaDB, almacenamiento local con archivos sidecar.
\end{itemize}

\textbf{Comunidad y ecosistema:}
\begin{itemize}
    \item \textbf{Comunidad consolidada:} Foros, GitHub y canales de soporte activos.
    \item \textbf{Documentación:} Muy completa y detallada.
    \item \textbf{Extensibilidad:} Limitada, pero con integración básica mediante API y archivos sidecar.
\end{itemize}

\textbf{Seguridad y privacidad:}
\begin{itemize}
    \item \textbf{Gestión de datos personales:} Los datos permanecen bajo control del usuario.
    \item \textbf{Cifrado:} No implementa cifrado de extremo a extremo, pero permite despliegues seguros en redes privadas.
    \item \textbf{Actualizaciones:} Ciclo de lanzamientos estable y respuesta adecuada a incidencias.
\end{itemize}

\textbf{Casos de uso y ejemplos reales:}
\begin{itemize}
    \item Utilizado por fotógrafos, familias y pequeñas empresas para organizar y buscar fotos de forma eficiente.
    \item Referencias en comunidades de software libre y autoalojamiento.
\end{itemize}

\textbf{Limitaciones actuales:}
\begin{itemize}
    \item Soporte limitado para múltiples usuarios.
    \item Escalabilidad horizontal restringida.
    \item Aplicaciones móviles limitadas a PWA, sin apps nativas.
    \item No utiliza una estructura estandarizada para la organización, dificultando la contribución de terceros.
\end{itemize}

\subsubsection{Ente}

Ente destaca por su enfoque radical en la privacidad y la seguridad, ofreciendo cifrado de extremo a extremo y una experiencia multiplataforma coherente gracias a Flutter.

\textbf{Propósito y público objetivo:} Dirigido a usuarios que priorizan la privacidad y la seguridad de sus fotos, como periodistas, activistas o cualquier persona preocupada por la confidencialidad de sus datos.

\textbf{Historia y contexto:} Proyecto joven pero con rápido crecimiento, impulsado por la demanda de soluciones seguras y privadas en el ámbito de la gestión fotográfica.

\textbf{Modelo de desarrollo:} Comunidad activa y transparente, con desarrollo abierto y enfoque en la seguridad.

\textbf{Características funcionales:}
\begin{itemize}
    \item \textbf{Facilidad de uso:} Interfaz moderna y coherente en todas las plataformas.
    \item \textbf{Instalación y despliegue:} Sencillo para usuarios finales, con apps en tiendas oficiales y opción de autoalojamiento.
    \item \textbf{Soporte multiplataforma:} Apps nativas para móvil y escritorio, y versión web.
    \item \textbf{Idiomas disponibles:} Traducción en expansión gracias a la comunidad.
    \item \textbf{Tecnologías:} Backend en Go, frontend y apps en Flutter, base de datos PostgreSQL, cifrado client-side.
\end{itemize}

\textbf{Comunidad y ecosistema:}
\begin{itemize}
    \item \textbf{Comunidad creciente:} Participación activa en GitHub y foros.
    \item \textbf{Documentación:} Clara y orientada a la seguridad.
    \item \textbf{Extensibilidad:} Limitada por el enfoque en la privacidad.
\end{itemize}

\textbf{Seguridad y privacidad:}
\begin{itemize}
    \item \textbf{Gestión de datos personales:} Cifrado de extremo a extremo, arquitectura de conocimiento cero.
    \item \textbf{Cifrado:} Todos los datos se cifran en el cliente antes de ser almacenados.
    \item \textbf{Actualizaciones:} Rápida respuesta a vulnerabilidades y mejoras continuas en seguridad.
\end{itemize}

\textbf{Casos de uso y ejemplos reales:}
\begin{itemize}
    \item Adoptado por usuarios preocupados por la privacidad y organizaciones que manejan información sensible.
    \item Recomendado en foros de privacidad y seguridad digital.
\end{itemize}

\textbf{Limitaciones actuales:}
\begin{itemize}
    \item Overhead del cifrado afecta el rendimiento.
    \item Características de IA limitadas por la privacidad.
    \item No se utiliza ninguna estructura de repositorio, dificultando la aportación al proyecto.
\end{itemize}

\subsection{Comparación Técnica Detallada}

\begin{table}[H]
\centering
\begin{tabular}{|l|c|c|c|}
\hline
\textbf{Característica} & \textbf{Immich} & \textbf{PhotoPrism} & \textbf{Ente} \\
\hline
Estrellas GitHub & 66,988 & 37,499 & 19,431 \\
Lenguaje Principal & TypeScript & Go & \gls{dart} \\
Arquitectura & Microservicios & Monolítica & Cliente-Servidor \\
Base de Datos & PostgreSQL & SQLite/MySQL & PostgreSQL \\
Aplicación Móvil & Flutter (8/10) & PWA (4/10) & Flutter (8/10) \\
Reconocimiento IA & 9/10 & 9/10 & En desarrollo \\
Múltiples Usuarios & 8/10 & No soportado & 9/10 \\
Búsqueda & 9/10 & 8/10 & 6/10 \\
Privacidad & 7/10 & 8/10 & 10/10 \\
Facilidad de Contribución & Alta & Baja & Baja-Media \\
\hline
\end{tabular}
\caption{Comparación técnica de las principales soluciones FOSS}
\label{tab:tech_comparison}
\end{table}

A continuación se explican brevemente las características comparadas en la tabla anterior:

\textbf{Estrellas GitHub:} Indica la popularidad y el nivel de interés de la comunidad en cada proyecto, reflejando tanto la visibilidad como la actividad de usuarios y desarrolladores.

\textbf{Lenguaje Principal:} Se refiere al lenguaje de programación predominante en el desarrollo del proyecto, lo que puede influir en la facilidad de contribución, el rendimiento y la compatibilidad con otras tecnologías.

\textbf{Arquitectura:} Describe el enfoque estructural del software. Una arquitectura de microservicios facilita la escalabilidad y el mantenimiento, mientras que una arquitectura monolítica puede simplificar el despliegue. El modelo cliente-servidor, por su parte, suele estar orientado a la seguridad y la separación de responsabilidades.

\textbf{Base de Datos:} Indica el sistema de gestión de bases de datos utilizado para almacenar la información. La elección de la base de datos afecta el rendimiento, la escalabilidad y la facilidad de integración con otros sistemas.

\textbf{Aplicación Móvil:} Evalúa la calidad y el tipo de experiencia móvil ofrecida, diferenciando entre aplicaciones nativas (mayor rendimiento y funcionalidad) y aplicaciones web progresivas (PWA), que suelen ser más limitadas.

\textbf{Reconocimiento IA:} Hace referencia a las capacidades de inteligencia artificial para el reconocimiento de imágenes, como la detección de rostros, objetos o escenas, y su grado de madurez en cada solución.

\textbf{Múltiples Usuarios:} Indica el soporte para la gestión de varios usuarios en la misma instancia, lo que es relevante para entornos familiares, de equipos o empresas.

\textbf{Búsqueda:} Evalúa la potencia y precisión de los mecanismos de búsqueda y filtrado de imágenes, incluyendo la búsqueda por metadatos, texto o reconocimiento automático.

\textbf{Privacidad:} Analiza el nivel de protección de los datos del usuario, considerando aspectos como el cifrado, la arquitectura de conocimiento cero y la gestión local de la información.

\textbf{Facilidad de Contribución:} Se refiere a la facilidad con la que nuevos desarrolladores pueden colaborar en el proyecto, influida por la calidad de la documentación y la modularidad del código.

\subsection{Análisis de Rendimiento}

El análisis de rendimiento revela diferentes enfoques optimizados para casos de uso específicos:

\textbf{Immich} sobresale en escalabilidad y características avanzadas, pero requiere más recursos del servidor. Su arquitectura de microservicios permite el escalado horizontal y la distribución de carga de trabajo, especialmente beneficial para el procesamiento de IA.

\textbf{PhotoPrism} ofrece la mejor eficiencia de recursos gracias a Go y su arquitectura monolítica, siendo ideal para instalaciones en hardware limitado. Su procesamiento de archivos es especialmente eficiente para fotógrafos profesionales.

\textbf{Ente} equilibra el rendimiento con la seguridad, trasladando el procesamiento al cliente para mantener la privacidad, aunque esto introduce latencia en las operaciones de cifrado/descifrado.

\subsection{Tendencias y Evolución del Sector}

El análisis del estado del arte revela varias tendencias importantes:

\begin{enumerate}
    \item \textbf{Migración hacia arquitecturas modernas}: Immich lidera esta tendencia con TypeScript y microservicios
    \item \textbf{Enfoque en la privacidad}: Ente representa la vanguardia en soluciones privacy-first
    \item \textbf{Madurez y estabilidad}: PhotoPrism ejemplifica la importancia de la estabilidad a largo plazo
    \item \textbf{Convergencia móvil}: Flutter se está estableciendo como la tecnología preferida para aplicaciones móviles
\end{enumerate}

\subsection{Conclusiones}

El ecosistema de bibliotecas de archivos multimedia FOSS presenta tres paradigmas distintos que abordan diferentes necesidades del mercado. Immich representa la innovación y escalabilidad, PhotoPrism la madurez y eficiencia, mientras que Ente pionerea en privacidad y seguridad.

Esta diversidad de enfoques indica un mercado en maduración donde no existe una solución única, sino que cada proyecto optimiza para casos de uso específicos. La elección entre estas soluciones depende fundamentalmente de los requisitos de escalabilidad, privacidad, recursos disponibles y experiencia técnica del usuario final.

El análisis sugiere que el futuro del sector se dirigirá hacia la convergencia de estas características, buscando soluciones que combinen la escalabilidad de Immich, la eficiencia de PhotoPrism y la privacidad de Ente.

Además, las nuevas tecnologías como pueden ser \gls{react-native} o en este caso \gls{lynxjs}, pueden aportar un mejor desarrollo de una aplicación móvil nativa, que permita una mejor experiencia de usuario y un mejor rendimiento en dispositivos móviles sin perder características nativas de las plataformas.

No se ve en ningún proyecto entre los más populares que haga uso de \gls{rust}. Aunque hay una tendencia a utilizar Go, el cual es un lenguaje que es más sencillo y proporciona un rendimiento muy bueno, Rust ofrece ventajas significativas en términos de seguridad y rendimiento, especialmente para aplicaciones que requieren un alto grado de concurrencia y eficiencia en el manejo de memoria. Esto sugiere una oportunidad para explorar Rust como una alternativa viable para el desarrollo de bibliotecas de archivos multimedia FOSS en el futuro.

Por lo general, se suele hacer uso de un almacenamiento de objetos ya implementado, como puede ser \gls{minio}, que es compatible con \gls{s3}. Esto permite una mayor escalabilidad y flexibilidad en el almacenamiento de grandes volúmenes de datos, lo cual es esencial para aplicaciones que manejan grandes bibliotecas de archivos multimedia.
Aún así, sería interesante explorar la posibilidad de implementar un almacenamiento de objetos propio, ya que esto podría ofrecer ventajas en términos de personalización y optimización para casos de uso específicos como puede ser el caso en el que ya se tiene un almacenamiento de archivos en el servidor y se quiere aprovechar para almacenar las fotos.


\subsection{Análisis de Costes}

El análisis de costes es un aspecto fundamental a la hora de evaluar las diferentes soluciones de bibliotecas de archivos multimedia FOSS. Este análisis abarca varios aspectos que impactan directamente en el coste total de implementación y mantenimiento.

\subsubsection{Requisitos de Hardware}

Los requisitos de hardware varían significativamente entre las diferentes soluciones:

\begin{itemize}
    \item \textbf{Immich}:
    \begin{itemize}
        \item Rendimiento óptimo con procesador moderno (por ejemplo, Intel NUC de 12ª generación)
        \item Mínimo 16GB de RAM recomendado para bibliotecas grandes
        \item Almacenamiento SSD para la base de datos y caché
        \item Almacenamiento NAS/HDD para los archivos originales
    \end{itemize}

    \item \textbf{PhotoPrism}:
    \begin{itemize}
        \item Requiere hardware más potente para un rendimiento similar
        \item Mínimo 16GB de RAM, preferiblemente 32GB para bibliotecas grandes
        \item GPU integrada recomendada para aceleración por hardware
        \item SSD local para miniaturas y caché
    \end{itemize}

    \item \textbf{Ente}:
    \begin{itemize}
        \item Requisitos moderados debido al cifrado client-side
        \item 8-16GB de RAM suficientes para la mayoría de casos
        \item Menor dependencia de GPU
    \end{itemize}
\end{itemize}

\subsubsection{Consumo de Recursos}

El consumo de recursos afecta directamente a los costes operativos:

\begin{itemize}
    \item \textbf{Almacenamiento}:
    \begin{itemize}
        \item Immich: Genera dos miniaturas por imagen, altamente configurable
        \item PhotoPrism: Genera 8 miniaturas JPEG con diferentes relaciones de aspecto
        \item Ente: Uso eficiente del almacenamiento debido al cifrado
    \end{itemize}

    \item \textbf{Procesamiento}:
    \begin{itemize}
        \item Immich: Procesamiento paralelo eficiente, menor impacto en CPU
        \item PhotoPrism: Mayor uso de CPU durante la indexación y procesamiento
        \item Ente: Sobrecarga adicional debido al cifrado
    \end{itemize}
\end{itemize}

\subsubsection{Costes de Mantenimiento}

Los costes de mantenimiento incluyen:

\begin{itemize}
    \item \textbf{Actualizaciones y Parches}:
    \begin{itemize}
        \item Immich: Actualizaciones frecuentes, proceso automatizado
        \item PhotoPrism: Desarrollo más lento, actualizaciones menos frecuentes
        \item Ente: Ciclo de desarrollo moderado
    \end{itemize}

    \item \textbf{Backup y Redundancia}:
    \begin{itemize}
        \item Todas las soluciones requieren estrategias de backup
        \item Costes adicionales para almacenamiento redundante
        \item Necesidad de monitorización y mantenimiento regular
    \end{itemize}
\end{itemize}

\subsubsection{Costes de Escalabilidad}

La escalabilidad impacta en los costes a largo plazo:

\begin{itemize}
    \item \textbf{Vertical}:
    \begin{itemize}
        \item Immich: Mejor rendimiento con hardware modesto
        \item PhotoPrism: Puede requerir actualizaciones de hardware más frecuentes
        \item Ente: Escalado limitado por el cifrado
    \end{itemize}

    \item \textbf{Horizontal}:
    \begin{itemize}
        \item Immich: Arquitectura de microservicios facilita la escalabilidad
        \item PhotoPrism: Limitaciones en escalabilidad horizontal
        \item Ente: Diseño centrado en privacidad limita opciones de escalado
    \end{itemize}
\end{itemize}

\subsubsection{Modelos de Monetización}

Es importante considerar los diferentes modelos de monetización:

\begin{itemize}
    \item \textbf{Immich}:
    \begin{itemize}
        \item Completamente gratuito y open source
        \item Sin funciones premium o de pago
        \item Financiado por donaciones y comunidad
    \end{itemize}

    \item \textbf{PhotoPrism}:
    \begin{itemize}
        \item Modelo freemium
        \item Algunas características avanzadas requieren suscripción
        \item Membresías Plus y Pro disponibles
    \end{itemize}

    \item \textbf{Ente}:
    \begin{itemize}
        \item Enfoque en privacidad y seguridad
        \item Modelo de negocio basado en donaciones
    \end{itemize}
\end{itemize}

\subsection{Análisis Comparativo de Métricas}

A partir de pruebas realizadas por la comunidad y documentación oficial, se presenta un análisis detallado de las métricas de rendimiento y capacidades de las tres soluciones principales.

\subsubsection{Rendimiento de Procesamiento}

\begin{table}[H]
\centering
\begin{tabular}{|l|c|c|c|}
\hline
\textbf{Métrica} & \textbf{Immich} & \textbf{PhotoPrism} & \textbf{Ente} \\
\hline
Velocidad de Indexación & Alta & Media-Baja & Media \\
Tiempo de Carga UI & $<$1s & 1-2s & $<$1s \\
Procesamiento Paralelo & Sí & No & Parcial \\
Uso de CPU & Moderado & Alto & Bajo \\
\hline
\end{tabular}
\caption{Comparación de métricas de rendimiento}
\label{tab:performance_metrics}
\end{table}

\subsubsection{Gestión de Almacenamiento}

\begin{itemize}
    \item \textbf{Miniaturas por Imagen}:
    \begin{itemize}
        \item Immich: 2 miniaturas configurables
        \item PhotoPrism: 8 miniaturas JPEG (diferentes relaciones de aspecto)
        \item Ente: Optimizado para cifrado, número variable
    \end{itemize}

    \item \textbf{Espacio de Caché}:
    \begin{itemize}
        \item Immich: Aproximadamente 1.5x el tamaño original
        \item PhotoPrism: 2-3x el tamaño original
        \item Ente: 1-1.5x el tamaño original
    \end{itemize}
\end{itemize}

\subsubsection{Precisión de Reconocimiento}

\begin{table}[H]
\centering
\begin{tabular}{|l|c|c|c|}
\hline
\textbf{Característica} & \textbf{Immich} & \textbf{PhotoPrism} & \textbf{Ente} \\
\hline
Detección Facial & 90\% & 60\% & 85\% \\
Reconocimiento de Objetos & 85\% & 70\% & No disponible \\
Agrupación de Fotos & Excelente & Buena & Limitada \\
Precisión de Búsqueda & Alta & Media & Media \\
\hline
\end{tabular}
\caption{Comparación de precisión en características de IA}
\label{tab:ai_accuracy}
\end{table}

\subsubsection{Tiempos de Procesamiento}

Para una biblioteca de prueba de 40,000 imágenes (aproximadamente 1.6TB):

\begin{itemize}
    \item \textbf{Immich}:
    \begin{itemize}
        \item Importación inicial: 2-3 horas
        \item Indexación completa: 4-6 horas
        \item Generación de miniaturas: 3-4 horas
        \item Procesamiento de IA: 6-8 horas
    \end{itemize}

    \item \textbf{PhotoPrism}:
    \begin{itemize}
        \item Importación inicial: 4-5 horas
        \item Indexación completa: 24-48 horas
        \item Generación de miniaturas: 8-10 horas
        \item Procesamiento de IA: 12-24 horas
    \end{itemize}

    \item \textbf{Ente}:
    \begin{itemize}
        \item Importación inicial: 3-4 horas
        \item Indexación completa: 8-10 horas
        \item Generación de miniaturas: 4-5 horas
        \item Procesamiento de IA: No aplicable
    \end{itemize}
\end{itemize}

\subsubsection{Métricas de Escalabilidad}

\begin{itemize}
    \item \textbf{Límites de Biblioteca}:
    \begin{itemize}
        \item Immich: Probado con $>$1 millón de fotos
        \item PhotoPrism: Recomendado hasta 500,000 fotos
        \item Ente: Probado con $>$250,000 fotos
    \end{itemize}

    \item \textbf{Rendimiento con Carga}:
    \begin{itemize}
        \item Immich: Mantiene rendimiento con múltiples usuarios
        \item PhotoPrism: Degradación notable con múltiples usuarios
        \item Ente: Rendimiento constante pero limitado
    \end{itemize}
\end{itemize}

\subsubsection{Soporte de Formatos}

\begin{table}[H]
\centering
\begin{tabular}{|l|c|c|c|}
\hline
\textbf{Formato} & \textbf{Immich} & \textbf{PhotoPrism} & \textbf{Ente} \\
\hline
JPEG & Sí & Sí & Sí \\
RAW & Parcial & Completo & No \\
HEIC & Sí & Sí & Parcial \\
Videos & Sí & Sí & Limitado \\
Live Photos & Sí & No & No \\
\hline
\end{tabular}
\caption{Comparación de soporte de formatos}
\label{tab:format_support}
\end{table}

\subsection{Interoperabilidad y Estándares Abiertos}

La interoperabilidad y el uso de estándares abiertos son factores clave para la adopción y sostenibilidad de las bibliotecas de archivos multimedia FOSS. Permiten a los usuarios migrar fácilmente sus datos, integrar la aplicación con otros servicios y evitar el "vendor lock-in". A continuación se analiza cómo las principales soluciones abordan estos aspectos:

\begin{itemize}
    \item \textbf{Integración con servicios externos}: Immich y PhotoPrism ofrecen soporte para almacenamiento externo mediante protocolos como \gls{webdav} y S3 (Amazon Simple Storage Service o compatibles como MinIO), facilitando la integración con soluciones de almacenamiento en la nube o NAS. Ente, por su enfoque en privacidad, limita la integración a servicios que no comprometan la seguridad de los datos.
    \item \textbf{Soporte de estándares de metadatos}: PhotoPrism destaca por su soporte completo de estándares como EXIF, \acrshort{iptc} y \acrshort{xmp} para metadatos, permitiendo una mejor interoperabilidad con otras aplicaciones de gestión de fotos. Immich y Ente ofrecen soporte parcial, centrado principalmente en EXIF.
    \item \textbf{Exportación e importación}: Todas las soluciones permiten la exportación e importación de fotos y metadatos, aunque el grado de automatización y compatibilidad varía. PhotoPrism facilita la migración mediante herramientas específicas y documentación detallada. Immich permite la importación desde carpetas existentes, pero la exportación masiva puede requerir herramientas adicionales. Ente prioriza la exportación cifrada para mantener la privacidad.
    \item \textbf{Federación y APIs}: Ninguna de las soluciones principales implementa actualmente protocolos de federación como \gls{activity-pub}, aunque existen discusiones en la comunidad sobre su posible adopción futura. Todas ofrecen APIs REST para integración con aplicaciones de terceros, aunque el grado de estabilidad y documentación varía (Immich tiene una API en evolución, PhotoPrism una API más estable).
\end{itemize}

La adopción de estándares abiertos y la interoperabilidad son aspectos en los que aún existe margen de mejora en el sector. Fomentar la compatibilidad con protocolos y formatos ampliamente aceptados facilitará la integración con otros sistemas y la migración de datos, beneficiando tanto a usuarios finales como a desarrolladores.


\subsection{Conclusiones}
\subsubsection{Conclusiones del Análisis de Métricas}

El análisis de métricas revela que Immich destaca en varios aspectos clave:

\begin{itemize}
    \item Mayor eficiencia en procesamiento paralelo
    \item Mejor equilibrio entre almacenamiento y funcionalidad
    \item Superior precisión en reconocimiento facial y de objetos
    \item Mejor escalabilidad para bibliotecas grandes
\end{itemize}

PhotoPrism, aunque más maduro, muestra limitaciones en rendimiento y escalabilidad, mientras que Ente destaca en privacidad pero con funcionalidades más limitadas. Estas métricas refuerzan la elección de tecnologías y arquitectura para nuestro proyecto, que busca combinar la eficiencia de Immich con mejoras adicionales en rendimiento y funcionalidad.

\subsubsection{Conclusiones del Análisis de Costes}

El análisis de costes revela que Immich ofrece la mejor relación coste-beneficio para la mayoría de los casos de uso:

\begin{itemize}
    \item Requisitos de hardware más moderados
    \item Mejor eficiencia en el uso de recursos
    \item Arquitectura que facilita la escalabilidad
    \item Modelo completamente gratuito sin costes ocultos
\end{itemize}

Sin embargo, la elección final dependerá de factores específicos como:
\begin{itemize}
    \item Tamaño de la biblioteca de fotos
    \item Recursos de hardware disponibles
    \item Necesidades de escalabilidad
    \item Presupuesto para mantenimiento
\end{itemize}

Para nuestro proyecto, el análisis de costes refuerza la decisión de utilizar tecnologías eficientes como Rust y Lynx.js, que permitirán minimizar los requisitos de hardware y los costes operativos mientras se mantiene un alto nivel de rendimiento.

\subsubsection{Conclusiones Interoperabilidad y Estándares Abiertos}
El análisis de interoperabilidad y estándares abiertos muestra que Immich y PhotoPrism son las soluciones más avanzadas en este aspecto, con un enfoque claro en la integración con servicios externos y el uso de metadatos estandarizados. Ente, aunque prioriza la privacidad, limita la interoperabilidad al restringir la integración con servicios externos.

En nuestra propuesta buscamos fomentar la interoperabilidad mediante el uso de estándares abiertos y APIs bien documentadas, lo que permitirá a los usuarios migrar fácilmente sus datos y contribuir al proyecto. Esto es esencial para garantizar la sostenibilidad a largo plazo y la adopción por parte de la comunidad.
Gracias a esto se busca que el proyecto no solo sea una solución de gestión de fotos, sino también un ecosistema abierto y colaborativo que permita a los usuarios y desarrolladores contribuir y beneficiarse mutuamente, además de poder integrar la aplicación con otros servicios y herramientas existentes.

\subsection{Aportaciones de nuestro proyecto al estado del arte}
Una vez realizado un estudio de las soluciones existentes y las tecnologías más utilizadas en el desarrollo de aplicaciones móviles y web, se ha identificado una clara oportunidad para mejorar la experiencia del usuario en la gestión de bibliotecas de archivos multimedia.

Nuestro proyecto busca abordar las limitaciones actuales de las aplicaciones de fotos, especialmente en términos de rendimiento, usabilidad, facilidad de aportación al proyecto y características avanzadas.
\subsubsection{Rendimiento}
El rendimiento es un aspecto crítico en las aplicaciones de fotos, especialmente cuando se manejan grandes colecciones. Muchas aplicaciones existentes sufren de lentitud en la carga y visualización de imágenes, lo que afecta negativamente la experiencia del usuario.
No solo se busca optimizar la carga de imágenes en la aplicación, sino mejorar la velocidad de respuesta que ofrece el servidor a la hora de recibir, procesar y responder a las peticiones de los usuarios.

Todo esto lo conseguiremos gracias a:
\begin{itemize}
    \item Aplicación móvil programada en Lynx.js, que permite un rendimiento optimizado mediante el uso de componentes específicos para la carga de datos e imágenes, así como un acceso más directo a las APIs nativas del dispositivo. Aunque es una tecnología emergente, utiliza JavaScript/TypeScript con su propia variante de React, lo que facilita la curva de aprendizaje para desarrolladores familiarizados con estas tecnologías.
    \item Servidor programado en Rust, que ofrece alto rendimiento, eficiencia en la gestión de recursos y acceso a optimizaciones a bajo nivel, lo que se traduce en una respuesta más rápida a las peticiones de los usuarios. Usar Rust nos va a proporcionar un código más seguro y eficiente, evitando errores comunes como las condiciones de carrera o los punteros nulos sin necesidad de un recolector de basura (GC).
\end{itemize}
\subsubsection{Contribución al proyecto}
Nuestro proyecto no solo se centrará en ofrecer una aplicación de fotos, sino que también se diseñará como un proyecto FOSS, lo que permitirá a la comunidad contribuir y mejorar la aplicación de manera continua.

Para ello se hará uso de las mejoras prácticas de desarrollo de software:
\begin{itemize}
    \item Uso de control de versiones (\Gls{git}), documentación clara y accesible, y un proceso de revisión de código que fomente la colaboración y la calidad del código.
    \item Se implementará una estructura de proyecto que facilite la incorporación de nuevos desarrolladores, con guías claras sobre cómo contribuir y estándares de codificación.
    \item El uso de Rust también aporta valor a nuestro proyecto al ofrecer un lenguaje moderno y seguro que minimiza los errores comunes de programación, lo que facilita la colaboración y la contribución de nuevos desarrolladores, eliminando posibles errores en la aplicación incluso antes de que se ejecute.
    \item Utilizaremos TypeScript en la aplicación móvil, lo que permitirá a los desarrolladores familiarizados con JavaScript contribuir fácilmente al proyecto, lo que facilitará la incorporación de nuevos colaboradores y fomentará una comunidad activa.
    \item Implementación de pruebas unitarias y de integración para asegurar la calidad del código y la funcionalidad de la aplicación.
\end{itemize}
