\newglossaryentry{microservicios}{
    name=Microservicios,
    text=microservicio,
    plural=microservicios,
    description={
        Un estilo arquitectónico que estructura una aplicación como un conjunto de servicios pequeños, independientes y desplegables de forma autónoma. Cada microservicio es responsable de una funcionalidad específica y se comunica con otros servicios a través de APIs
    },
}

\newglossaryentry{nestjs}{
    name=NestJS,
    description={
        Un framework para construir aplicaciones del lado del servidor con Node.js, que utiliza TypeScript y se basa en el patrón de diseño de microservicios. Proporciona una estructura modular y escalable para desarrollar aplicaciones web y APIs
    },
}

\newglossaryentry{nodejs}{
    name=Node.js,
    description={
        Un entorno de ejecución para JavaScript del lado del servidor, que permite ejecutar código JavaScript fuera de un navegador web. Es conocido por su modelo de I/O no bloqueante y su capacidad para manejar aplicaciones en tiempo real
    },
}

\newglossaryentry{sveltekit}{
    name=SvelteKit,
    description={
        Un framework para construir aplicaciones web modernas utilizando Svelte. Proporciona una estructura para el desarrollo de aplicaciones del lado del cliente y del servidor, con características como enrutamiento, generación de sitios estáticos y manejo de datos
    },
}

\newglossaryentry{flutter}{
    name=Flutter,
    description={
        Un framework de código abierto para construir aplicaciones nativas multiplataforma utilizando un único código base. Permite desarrollar aplicaciones para iOS, Android, web y escritorio con una experiencia de usuario consistente
    },
}

\newglossaryentry{postgresql}{
    name=PostgreSQL,
    description={
        Un sistema de gestión de bases de datos relacional y objeto-relacional de código abierto, conocido por su robustez, escalabilidad y cumplimiento de estándares SQL. Es ampliamente utilizado para aplicaciones empresariales y web
    },
}

\newglossaryentry{redis}{
    name=Redis,
    description={
        Un sistema de almacenamiento de datos en memoria, clave-valor, que se utiliza como base de datos, caché y broker de mensajes. Es conocido por su alta velocidad y eficiencia en el manejo de datos temporales y en tiempo real
    },
}

\newglossaryentry{tensorflow}{
    name=TensorFlow,
    description={
        Una biblioteca de código abierto para el aprendizaje automático y la inteligencia artificial, desarrollada por Google. Proporciona herramientas para construir y entrenar modelos de aprendizaje profundo y es ampliamente utilizada en aplicaciones de IA
    },
}

\newglossaryentry{live-photos}{
    name=Live Photos,
    description={
        Una característica de dispositivos modernos tanto Apple como Android que captura una imagen estática junto con un breve video, permitiendo revivir momentos con movimiento y sonido
    },
}

\newglossaryentry{go}{
    name=Go,
    description={
        Un lenguaje de programación de código abierto desarrollado por Google, conocido por su simplicidad, eficiencia y concurrencia. Es ampliamente utilizado para desarrollar aplicaciones de red y sistemas distribuidos
    },
}

\newglossaryentry{vuejs}{
    name=Vue.js,
    description={
        Un framework progresivo para construir interfaces de usuario. Es conocido por su facilidad de integración, flexibilidad y capacidad para crear aplicaciones web interactivas y dinámicas
    },
}

\newglossaryentry{sqlite}{
    name=SQLite,
    description={
        Un sistema de gestión de bases de datos relacional ligero y autónomo, que se almacena en un solo archivo. Es ampliamente utilizado en aplicaciones móviles y de escritorio debido a su simplicidad y eficiencia
    },
}

\newglossaryentry{mariadb}{
    name=MariaDB,
    description={
        Un sistema de gestión de bases de datos relacional de código abierto, que es un fork de MySQL. Es conocido por su rendimiento, escalabilidad y compatibilidad con MySQL, y se utiliza en aplicaciones web y empresariales
    },
}

\newglossaryentry{sidecar}{
    name=Sidecar,
    text=sidecar,
    description={
        Un patrón arquitectónico en el que un servicio auxiliar se ejecuta junto a un servicio principal para proporcionar funcionalidades adicionales, como monitoreo, logging o proxy. Es comúnmente utilizado en arquitecturas de microservicios
    },
}

\newglossaryentry{dart}{
    name=Dart,
    description={
        Un lenguaje de programación desarrollado por Google, diseñado para construir aplicaciones web, móviles y de escritorio. Es el lenguaje principal utilizado en el framework Flutter y se caracteriza por su sintaxis clara y su enfoque en la productividad del desarrollador
    },
}

\newglossaryentry{react-native}{
    name=React Native,
    description={
        Un framework de código abierto para construir aplicaciones móviles nativas utilizando JavaScript y React. Permite desarrollar aplicaciones para iOS y Android con un único código base, aprovechando componentes nativos para una experiencia de usuario fluida
    },
}

\newglossaryentry{lynxjs}{
    name=Lynx.js,
    description={
        Framework que permite a los desarrolladores web crear aplicaciones multiplataforma. Permite renderizar de forma nativa en Android, iOS y la web. Destaca por su enfoque en el procesado multi-hilo, que hace que la experiencia de usuario sea fluida y rápida en todo momento
    },
}

\newglossaryentry{rust}{
    name=Rust,
    description={
        Un lenguaje de programación de sistemas enfocado en la seguridad, el rendimiento y la concurrencia. Es conocido por su sistema de tipos y su enfoque en evitar errores comunes de memoria, lo que lo hace ideal para aplicaciones de alto rendimiento y sistemas críticos
    },
}

\newglossaryentry{minio}{
    name=MinIO,
    description={
        Un sistema de almacenamiento de objetos de código abierto, compatible con la API de Amazon S3. Es conocido por su alto rendimiento, escalabilidad y facilidad de uso, y se utiliza para almacenar grandes volúmenes de datos no estructurados en aplicaciones modernas
    },
}

\newglossaryentry{s3}{
    name=Amazon S3,
    description={
        Un servicio de almacenamiento de objetos en la nube proporcionado por Amazon Web Services (AWS). Permite almacenar y recuperar cualquier cantidad de datos desde cualquier lugar en la web, y es ampliamente utilizado para aplicaciones web, móviles y de big data
    },
}

\newglossaryentry{framework}{
    name=Framework,
    description={
        Un conjunto de herramientas, bibliotecas y convenciones que facilitan el desarrollo de software al proporcionar una estructura predefinida. Los frameworks pueden ser específicos para un lenguaje de programación o para un tipo de aplicación, como aplicaciones web o móviles
    },
}

\newglossaryentry{tipado-dinamico}{
    name=Tipado dinámico,
    text=tipado dinámico,
    description={
        Un sistema de tipos en el que las variables pueden cambiar de tipo en tiempo de ejecución. Esto permite mayor flexibilidad en el código, pero también puede llevar a errores difíciles de detectar si no se maneja adecuadamente
    },
}

\newglossaryentry{condicion-carrera}{
    name=Condición de carrera,
    text=condición de carrera,
    plural=condiciones de carrera,
    description={
        Una situación en la que dos o más procesos o hilos acceden a recursos compartidos y tratan de cambiar su estado al mismo tiempo, lo que puede llevar a resultados inesperados o inconsistentes. Es un problema común en programación concurrente y paralela
    },
}

\newglossaryentry{rust-ownership}{
    name=Rust Ownership,
    text=ownership,
    description={
        Un sistema de gestión de memoria en el lenguaje de programación Rust que garantiza la seguridad de memoria sin necesidad de un recolector de basura. Se basa en las reglas de propiedad, préstamos y referencias, lo que permite a los desarrolladores escribir código seguro y eficiente
    },
}

\newglossaryentry{rust-borrowing}{
    name=Rust Borrowing,
    text=borrowing,
    description={
        Un mecanismo en Rust que permite a los desarrolladores tomar prestados valores sin transferir su propiedad. Esto permite compartir datos entre diferentes partes del código de manera segura, evitando problemas de concurrencia y garantizando la integridad de los datos
    },
}

\newglossaryentry{rust-lifetimes}{
    name=Rust Lifetimes,
    text=lifetime,
    description={
        Un concepto en Rust que permite a los desarrolladores especificar cuánto tiempo viven las referencias a los datos. Esto ayuda a prevenir errores de uso de memoria, como referencias colgantes, y garantiza que las referencias sean válidas durante el tiempo necesario
    },
}

\newglossaryentry{goroutine}{
    name=Goroutine,
    description={
        Una función que se ejecuta de manera concurrente con otras funciones en el lenguaje de programación Go. Las goroutines son ligeras y permiten a los desarrolladores escribir código concurrente de manera sencilla, facilitando la creación de aplicaciones que manejan múltiples tareas al mismo tiempo
    },
}

\newglossaryentry{fearless-concurrency}{
    name=Fearless Concurrency,
    description={
        Un enfoque en la programación concurrente que busca minimizar los errores comunes asociados con el acceso concurrente a recursos compartidos. En Rust, se logra a través de su sistema de tipos, que garantiza la seguridad de memoria y evita condiciones de carrera
    },
}

\newglossaryentry{mutex}{
    name=Mutex,
    text=mutex,
    plural=mutexes,
    description={
        Un mecanismo de sincronización que permite a múltiples hilos o procesos acceder a un recurso compartido de manera segura, garantizando que solo un hilo pueda acceder al recurso a la vez. Es comúnmente utilizado para evitar condiciones de carrera en programación concurrente
    },
}

\newglossaryentry{semaforo}{
    name=Semáforo,
    text=semáforo,
    plural=semáforos,
    description={
        Un mecanismo de sincronización que controla el acceso a recursos compartidos mediante contadores. Permite que múltiples hilos o procesos accedan a un recurso limitado, asegurando que no se exceda el número máximo de accesos simultáneos
    },
}

\newglossaryentry{lock}{
    name=Lock,
    text=lock,
    plural=locks,
    description={
        Un mecanismo que impide que otros hilos o procesos accedan a un recurso compartido mientras está bloqueado. Se utiliza para garantizar la exclusión mutua y evitar condiciones de carrera en programación concurrente
    },
}

\newglossaryentry{haskell}{
    name=Haskell,
    description={
        Un lenguaje de programación funcional puro, conocido por su fuerte sistema de tipos y su enfoque en la inmutabilidad. Haskell es utilizado en aplicaciones que requieren alta confiabilidad y mantenibilidad, y es popular en el ámbito académico y en la industria
    },
}

\newglossaryentry{rust-traits}{
    name=Rust Traits,
    text=trait,
    plural=traits,
    description={
        Un mecanismo en Rust que permite definir comportamientos compartidos entre diferentes tipos. Los traits son similares a las interfaces en otros lenguajes y permiten la implementación de polimorfismo, facilitando la reutilización de código y la abstracción
    },
}

\newglossaryentry{adt-gls}{
    name=ADT,
    text=ADT,
    plural=ADTs,
    description={
        Un Tipo de Datos Algebraico (ADT) es una estructura de datos definida por sus operaciones y comportamientos, en lugar de su implementación. Los ADTs permiten a los desarrolladores trabajar con abstracciones y encapsular detalles de implementación, facilitando la creación de programas más robustos y mantenibles
    },
}

\newglossaryentry{pattern-matching}{
    name=Pattern Matching,
    text=pattern matching,
    description={
        Una técnica de programación que permite comparar una estructura de datos con un patrón y, si coincide, extraer valores o ejecutar acciones específicas. Es comúnmente utilizada en lenguajes funcionales y proporciona una forma concisa y legible de manejar estructuras complejas
    },
}

\newglossaryentry{metaprogramacion}{
    name=Metaprogramación,
    text=metaprogramación,
    description={
        Un enfoque de programación en el que los programas pueden tratar otros programas como datos, permitiendo la creación de código que genera o manipula código. Esto permite la automatización de tareas repetitivas y la creación de abstracciones más poderosas
    },
}

\newglossaryentry{webassembly}{
    name=WebAssembly,
    text=WebAssembly,
    description={
        Un formato de código binario que permite ejecutar código de alto rendimiento en navegadores web. WebAssembly es un estándar abierto que complementa JavaScript, permitiendo a los desarrolladores escribir aplicaciones web más rápidas y eficientes en lenguajes como C, C++ y Rust
    },
}

\newglossaryentry{devops}{
    name=DevOps,
    description={
        Una práctica de desarrollo de software que combina el desarrollo (Dev) y las operaciones (Ops) para mejorar la colaboración, la automatización y la entrega continua. DevOps busca acortar el ciclo de vida del desarrollo de software y aumentar la calidad y la confiabilidad de las aplicaciones
    },
}

\newglossaryentry{skia}{
    name=Skia,
    description={
        Una biblioteca de gráficos 2D de código abierto utilizada por Google en sus productos, como Chrome y Android. Skia proporciona una API para renderizar gráficos vectoriales, texto y imágenes, y es conocida por su rendimiento y flexibilidad
    },
}

\newglossaryentry{ahead-of-time}{
    name=Ahead-of-Time (AOT),
    text=ahead-of-time,
    plural=Ahead-of-Time,
    description={
        Un enfoque de compilación en el que el código fuente se compila a código máquina antes de su ejecución, en lugar de compilarlo en tiempo de ejecución. AOT puede mejorar el rendimiento y reducir el tiempo de inicio de las aplicaciones, especialmente en entornos móviles y embebidos
    },
}

\newglossaryentry{hot-reload}{
    name=Hot Reload,
    text=hot reload,
    description={
        Una característica que permite a los desarrolladores ver los cambios en el código fuente reflejados en la aplicación en tiempo real, sin necesidad de reiniciar la aplicación. Esto acelera el proceso de desarrollo y mejora la productividad al permitir iteraciones rápidas
    },
}

\newglossaryentry{webgl}{
    name=WebGL,
    description={
        Una API de JavaScript que permite renderizar gráficos 3D en navegadores web sin necesidad de plugins. WebGL utiliza la potencia de la GPU para crear gráficos interactivos y es ampliamente utilizado en juegos, visualizaciones y aplicaciones científicas
    },
}

\newglossaryentry{lazy-loading}{
    name=Lazy Loading,
    text=lazy loading,
    description={
        Una técnica de optimización que retrasa la carga de recursos o componentes hasta que son necesarios, en lugar de cargarlos todos al inicio. Esto mejora el rendimiento y reduce el tiempo de carga inicial de las aplicaciones, especialmente en aplicaciones web y móviles
    },
}

\newglossaryentry{jetpack-compose}{
    name=Jetpack Compose,
    description={
        Un toolkit moderno de Android para construir interfaces de usuario declarativas. Utiliza un enfoque basado en componentes y permite a los desarrolladores crear UI de manera más eficiente y con menos código, aprovechando las capacidades de Kotlin
    },
}

\newglossaryentry{git}{
    name=Git,
    text=git,
    description={
        Un sistema de control de versiones distribuido que permite a los desarrolladores rastrear cambios en el código fuente a lo largo del tiempo. Git es ampliamente utilizado en proyectos de software para colaborar, gestionar versiones y mantener un historial de cambios
    },
}

\newglossaryentry{webdav}{
    name=WebDAV,
    description={
        Un protocolo de red que permite a los usuarios gestionar archivos en servidores web. WebDAV extiende el protocolo HTTP para permitir operaciones como crear, mover y eliminar archivos y directorios, facilitando la colaboración y el intercambio de archivos en línea
    },
}

\newglossaryentry{activity-pub}{
    name=ActivityPub,
    description={
        Un protocolo de comunicación descentralizado para redes sociales y aplicaciones web. Permite a los usuarios interactuar y compartir contenido entre diferentes plataformas y servicios, promoviendo la interoperabilidad y la descentralización en la web
    },
}

\newglossaryentry{bridge}{
    name=Bridge,
    text=bridge,
    description={
        Un componente que conecta dos sistemas o protocolos diferentes, permitiendo la comunicación y el intercambio de datos entre ellos. Los bridges son comunes en aplicaciones distribuidas y redes sociales para integrar diferentes servicios y plataformas
    },
}

\newglossaryentry{service-worker}{
    name=Service Worker,
    description={
        Un script que el navegador ejecuta en segundo plano, separado de una página web, permitiendo funcionalidades como la sincronización en segundo plano, notificaciones push y el manejo de solicitudes de red. Los service workers son fundamentales para crear aplicaciones web progresivas (PWA)
    },
}

\newglossaryentry{mockear}{
    name=Mockear,
    text=mockear,
    description={
        El proceso de crear versiones simuladas de objetos o servicios para probar el comportamiento de un sistema sin depender de sus implementaciones reales. Esto es útil en pruebas unitarias y de integración para aislar componentes y verificar su funcionamiento
    },
}

\newglossaryentry{ifr}{
    name=Instant First-Frame Rendering (IFR),
    text=Instant First-Frame Rendering,
    description={
        Técnica de renderizado que permite mostrar el primer fotograma de una aplicación de manera instantánea, mejorando la experiencia del usuario al reducir el tiempo de espera para ver la interfaz.
    },
}
