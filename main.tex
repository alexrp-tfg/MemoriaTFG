\documentclass[11pt]{book}

\usepackage[spanish]{babel}

\usepackage[a4paper,top=2cm,bottom=2cm,left=3cm,right=3cm,marginparwidth=1.75cm]{geometry}

% Useful packages
\usepackage{amsmath}
\usepackage{graphicx}
\graphicspath{ {./images/} }
\usepackage[colorlinks=true, allcolors=blue]{hyperref}
\usepackage{titling}
\usepackage{dirtree}
\usepackage{eurosym}

\usepackage{listings}
\usepackage{xcolor}
\usepackage{color}
\input{languages_definitions/rust}
\lstdefinelanguage{typescript}
{
    columns=fullflexible,
    keepspaces=true,
    frame=single,
    framesep=0pt,
    framerule=0pt,
    framexleftmargin=4pt,
    framexrightmargin=4pt,
    framextopmargin=5pt,
    framexbottommargin=3pt,
    xleftmargin=4pt,
    xrightmargin=4pt,
    backgroundcolor=\color{GrayCodeBlock},
    basicstyle=\ttfamily\color{BlackText},
    keywords={
        true,false,null,undefined,
        if,else,switch,case,default,
        for,while,do,break,continue,return,yield,
        try,catch,finally,throw,
        function,async,await,
        let,const,var,
        import,export,from,as,
        class,interface,extends,implements,constructor,super,this,new,
        public,private,protected,readonly,static,get,set,
        type,enum,namespace,module,declare,abstract,infer,is,keyof,typeof,instanceof,in,of
    },
    keywordstyle=\color{PurpleKeyword},
    ndkeywords={
        number,string,boolean,any,void,never,unknown,symbol,bigint,object,
        Array,ReadonlyArray,Promise,Record,Partial,Required,Pick,Omit,Exclude,Extract,NonNullable,Parameters,ReturnType,InstanceType
    },
    ndkeywordstyle=\color{RedTypename},
    comment=[l][\color{GrayComment}\slshape]{//},
    morecomment=[s][\color{GrayComment}\slshape]{/*}{*/},
    stringstyle=\color{GreenString},
    string=[b]",
    morestring=[b]'
}

\usepackage{float}
\usepackage{colortbl} % For colored tables
\usepackage{tabularx}
\usepackage{makecell}
\usepackage{ltablex}
\usepackage[acronym]{glossaries}
\makeglossaries

\usepackage{csquotes}
\usepackage[backend=biber,style=apa,sorting=nyt,language=spanish]{biblatex}
\addbibresource{references.bib}

\setlength{\bibitemsep}{0.5em}



\definecolor{codegreen}{rgb}{0,0.6,0}
\definecolor{codegray}{rgb}{0.5,0.5,0.5}
\definecolor{codepurple}{rgb}{0.58,0,0.82}
\definecolor{backcolour}{rgb}{0.95,0.95,0.92}

\lstdefinestyle{mystyle}{
  backgroundcolor=\color{backcolour},
  commentstyle=\color{codegreen},
  keywordstyle=\color{magenta},
  numberstyle=\tiny\color{codegray},
  stringstyle=\color{codepurple},
  basicstyle=\ttfamily\footnotesize,
  breakatwhitespace=false,
  breaklines=true,
  captionpos=b,
  keepspaces=true,
  numbers=left,
  numbersep=5pt,
  showspaces=false,
  showstringspaces=false,
  showtabs=false,
  tabsize=2
}

\lstset{
    inputencoding=utf8,
    extendedchars=true,
    literate={á}{{\'a}}1 {é}{{\'e}}1 {í}{{\'i}}1 {ó}{{\'o}}1 {ú}{{\'u}}1
             {Á}{{\'A}}1 {É}{{\'E}}1 {Í}{{\'I}}1 {Ó}{{\'O}}1 {Ú}{{\'U}}1
             {ñ}{{\~n}}1 {Ñ}{{\~N}}1
}

\lstset{style=mystyle}
\setlength{\parindent}{0pt}
\setlength{\parskip}{0.5em}

\title{Sistema multiplataforma y multiusuario para la compartición de
archivos multimedia}
\author{Alejandro Ramos Peña}


\begin{document}

\begin{titlepage}
 
 
\newlength{\centeroffset}
\setlength{\centeroffset}{-0.5\oddsidemargin}
\addtolength{\centeroffset}{0.5\evensidemargin}
\thispagestyle{empty}

\noindent\hspace*{\centeroffset}\begin{minipage}{\textwidth}

\centering
\includegraphics[width=0.9\textwidth]{images/logo-ugr.jpg}\\[1.4cm]

\textsc{ \Large TRABAJO FIN DE GRADO\\[0.2cm]}
\textsc{ INGENIERÍA EN INFORMÁTICA}\\[1cm]
% Upper part of the page
% 
% Title
{\Huge\bfseries Sistema multiplataforma y multiusuario basado en tecnologías FOSS para la compartición de archivos multimedia\\
}
\noindent\rule[-1ex]{\textwidth}{3pt}\\[3.5ex]
{\large\bfseries Solución FOSS para la compartición de archivos multimedia en entornos domésticos\\[1.5cm]}
\end{minipage}

\vspace{2.5cm}
\noindent\hspace*{\centeroffset}\begin{minipage}{\textwidth}
\centering

\textbf{Autor}\\ {Alejandro Ramos Peña}\\[2.5ex]
\textbf{Directores}\\
{Nombre Apellido1 Apellido2 (tutor1)}\\[2cm]
\includegraphics[width=0.3\textwidth]{images/logo-ugr.jpg}\\[0.1cm]
\textsc{Escuela Técnica Superior de Ingenierías Informática y de Telecomunicación}\\
\textsc{---}\\
Granada, Septiembre 2025
\end{minipage}
%\addtolength{\textwidth}{\centeroffset}
%\vspace{\stretch{2}}
\end{titlepage}




\tableofcontents

\newpage
\listoffigures

\newpage
\renewcommand{\listtablename}{Índice de Tablas}
\listoftables

\newpage
\newglossaryentry{microservicios}{
    name=Microservicios,
    text=microservicio,
    plural=microservicios,
    description={
        Un estilo arquitectónico que estructura una aplicación como un conjunto de servicios pequeños, independientes y desplegables de forma autónoma. Cada microservicio es responsable de una funcionalidad específica y se comunica con otros servicios a través de APIs
    },
}

\newglossaryentry{nestjs}{
    name=NestJS,
    description={
        Un framework para construir aplicaciones del lado del servidor con Node.js, que utiliza TypeScript y se basa en el patrón de diseño de microservicios. Proporciona una estructura modular y escalable para desarrollar aplicaciones web y APIs
    },
}

\newglossaryentry{nodejs}{
    name=Node.js,
    description={
        Un entorno de ejecución para JavaScript del lado del servidor, que permite ejecutar código JavaScript fuera de un navegador web. Es conocido por su modelo de I/O no bloqueante y su capacidad para manejar aplicaciones en tiempo real
    },
}

\newglossaryentry{sveltekit}{
    name=SvelteKit,
    description={
        Un framework para construir aplicaciones web modernas utilizando Svelte. Proporciona una estructura para el desarrollo de aplicaciones del lado del cliente y del servidor, con características como enrutamiento, generación de sitios estáticos y manejo de datos
    },
}

\newglossaryentry{flutter}{
    name=Flutter,
    description={
        Un framework de código abierto para construir aplicaciones nativas multiplataforma utilizando un único código base. Permite desarrollar aplicaciones para iOS, Android, web y escritorio con una experiencia de usuario consistente
    },
}

\newglossaryentry{postgresql}{
    name=PostgreSQL,
    description={
        Un sistema de gestión de bases de datos relacional y objeto-relacional de código abierto, conocido por su robustez, escalabilidad y cumplimiento de estándares SQL. Es ampliamente utilizado para aplicaciones empresariales y web
    },
}

\newglossaryentry{redis}{
    name=Redis,
    description={
        Un sistema de almacenamiento de datos en memoria, clave-valor, que se utiliza como base de datos, caché y broker de mensajes. Es conocido por su alta velocidad y eficiencia en el manejo de datos temporales y en tiempo real
    },
}

\newglossaryentry{tensorflow}{
    name=TensorFlow,
    description={
        Una biblioteca de código abierto para el aprendizaje automático y la inteligencia artificial, desarrollada por Google. Proporciona herramientas para construir y entrenar modelos de aprendizaje profundo y es ampliamente utilizada en aplicaciones de IA
    },
}

\newglossaryentry{live-photos}{
    name=Live Photos,
    description={
        Una característica de dispositivos modernos tanto Apple como Android que captura una imagen estática junto con un breve video, permitiendo revivir momentos con movimiento y sonido
    },
}

\newglossaryentry{go}{
    name=Go,
    description={
        Un lenguaje de programación de código abierto desarrollado por Google, conocido por su simplicidad, eficiencia y concurrencia. Es ampliamente utilizado para desarrollar aplicaciones de red y sistemas distribuidos
    },
}

\newglossaryentry{vuejs}{
    name=Vue.js,
    description={
        Un framework progresivo para construir interfaces de usuario. Es conocido por su facilidad de integración, flexibilidad y capacidad para crear aplicaciones web interactivas y dinámicas
    },
}

\newglossaryentry{sqlite}{
    name=SQLite,
    description={
        Un sistema de gestión de bases de datos relacional ligero y autónomo, que se almacena en un solo archivo. Es ampliamente utilizado en aplicaciones móviles y de escritorio debido a su simplicidad y eficiencia
    },
}

\newglossaryentry{mariadb}{
    name=MariaDB,
    description={
        Un sistema de gestión de bases de datos relacional de código abierto, que es un fork de MySQL. Es conocido por su rendimiento, escalabilidad y compatibilidad con MySQL, y se utiliza en aplicaciones web y empresariales
    },
}

\newglossaryentry{sidecar}{
    name=Sidecar,
    text=sidecar,
    description={
        Un patrón arquitectónico en el que un servicio auxiliar se ejecuta junto a un servicio principal para proporcionar funcionalidades adicionales, como monitoreo, logging o proxy. Es comúnmente utilizado en arquitecturas de microservicios
    },
}

\newglossaryentry{dart}{
    name=Dart,
    description={
        Un lenguaje de programación desarrollado por Google, diseñado para construir aplicaciones web, móviles y de escritorio. Es el lenguaje principal utilizado en el framework Flutter y se caracteriza por su sintaxis clara y su enfoque en la productividad del desarrollador
    },
}

\newglossaryentry{react-native}{
    name=React Native,
    description={
        Un framework de código abierto para construir aplicaciones móviles nativas utilizando JavaScript y React. Permite desarrollar aplicaciones para iOS y Android con un único código base, aprovechando componentes nativos para una experiencia de usuario fluida
    },
}

\newglossaryentry{lynxjs}{
    name=Lynx.js,
    description={
        Framework que permite a los desarrolladores web crear aplicaciones multiplataforma. Permite renderizar de forma nativa en Android, iOS y la web. Destaca por su enfoque en el procesado multi-hilo, que hace que la experiencia de usuario sea fluida y rápida en todo momento
    },
}

\newglossaryentry{rust}{
    name=Rust,
    description={
        Un lenguaje de programación de sistemas enfocado en la seguridad, el rendimiento y la concurrencia. Es conocido por su sistema de tipos y su enfoque en evitar errores comunes de memoria, lo que lo hace ideal para aplicaciones de alto rendimiento y sistemas críticos
    },
}

\newglossaryentry{minio}{
    name=MinIO,
    description={
        Un sistema de almacenamiento de objetos de código abierto, compatible con la API de Amazon S3. Es conocido por su alto rendimiento, escalabilidad y facilidad de uso, y se utiliza para almacenar grandes volúmenes de datos no estructurados en aplicaciones modernas
    },
}

\newglossaryentry{s3}{
    name=Amazon S3,
    description={
        Un servicio de almacenamiento de objetos en la nube proporcionado por Amazon Web Services (AWS). Permite almacenar y recuperar cualquier cantidad de datos desde cualquier lugar en la web, y es ampliamente utilizado para aplicaciones web, móviles y de big data
    },
}

\newglossaryentry{framework}{
    name=Framework,
    description={
        Un conjunto de herramientas, bibliotecas y convenciones que facilitan el desarrollo de software al proporcionar una estructura predefinida. Los frameworks pueden ser específicos para un lenguaje de programación o para un tipo de aplicación, como aplicaciones web o móviles
    },
}

\newglossaryentry{tipado-dinamico}{
    name=Tipado dinámico,
    text=tipado dinámico,
    description={
        Un sistema de tipos en el que las variables pueden cambiar de tipo en tiempo de ejecución. Esto permite mayor flexibilidad en el código, pero también puede llevar a errores difíciles de detectar si no se maneja adecuadamente
    },
}

\newglossaryentry{condicion-carrera}{
    name=Condición de carrera,
    text=condición de carrera,
    plural=condiciones de carrera,
    description={
        Una situación en la que dos o más procesos o hilos acceden a recursos compartidos y tratan de cambiar su estado al mismo tiempo, lo que puede llevar a resultados inesperados o inconsistentes. Es un problema común en programación concurrente y paralela
    },
}

\newglossaryentry{rust-ownership}{
    name=Rust Ownership,
    text=ownership,
    description={
        Un sistema de gestión de memoria en el lenguaje de programación Rust que garantiza la seguridad de memoria sin necesidad de un recolector de basura. Se basa en las reglas de propiedad, préstamos y referencias, lo que permite a los desarrolladores escribir código seguro y eficiente
    },
}

\newglossaryentry{rust-borrowing}{
    name=Rust Borrowing,
    text=borrowing,
    description={
        Un mecanismo en Rust que permite a los desarrolladores tomar prestados valores sin transferir su propiedad. Esto permite compartir datos entre diferentes partes del código de manera segura, evitando problemas de concurrencia y garantizando la integridad de los datos
    },
}

\newglossaryentry{rust-lifetimes}{
    name=Rust Lifetimes,
    text=lifetime,
    description={
        Un concepto en Rust que permite a los desarrolladores especificar cuánto tiempo viven las referencias a los datos. Esto ayuda a prevenir errores de uso de memoria, como referencias colgantes, y garantiza que las referencias sean válidas durante el tiempo necesario
    },
}

\newglossaryentry{goroutine}{
    name=Goroutine,
    description={
        Una función que se ejecuta de manera concurrente con otras funciones en el lenguaje de programación Go. Las goroutines son ligeras y permiten a los desarrolladores escribir código concurrente de manera sencilla, facilitando la creación de aplicaciones que manejan múltiples tareas al mismo tiempo
    },
}

\newglossaryentry{fearless-concurrency}{
    name=Fearless Concurrency,
    description={
        Un enfoque en la programación concurrente que busca minimizar los errores comunes asociados con el acceso concurrente a recursos compartidos. En Rust, se logra a través de su sistema de tipos, que garantiza la seguridad de memoria y evita condiciones de carrera
    },
}

\newglossaryentry{mutex}{
    name=Mutex,
    text=mutex,
    plural=mutexes,
    description={
        Un mecanismo de sincronización que permite a múltiples hilos o procesos acceder a un recurso compartido de manera segura, garantizando que solo un hilo pueda acceder al recurso a la vez. Es comúnmente utilizado para evitar condiciones de carrera en programación concurrente
    },
}

\newglossaryentry{semaforo}{
    name=Semáforo,
    text=semáforo,
    plural=semáforos,
    description={
        Un mecanismo de sincronización que controla el acceso a recursos compartidos mediante contadores. Permite que múltiples hilos o procesos accedan a un recurso limitado, asegurando que no se exceda el número máximo de accesos simultáneos
    },
}

\newglossaryentry{lock}{
    name=Lock,
    text=lock,
    plural=locks,
    description={
        Un mecanismo que impide que otros hilos o procesos accedan a un recurso compartido mientras está bloqueado. Se utiliza para garantizar la exclusión mutua y evitar condiciones de carrera en programación concurrente
    },
}

\newglossaryentry{haskell}{
    name=Haskell,
    description={
        Un lenguaje de programación funcional puro, conocido por su fuerte sistema de tipos y su enfoque en la inmutabilidad. Haskell es utilizado en aplicaciones que requieren alta confiabilidad y mantenibilidad, y es popular en el ámbito académico y en la industria
    },
}

\newglossaryentry{rust-traits}{
    name=Rust Traits,
    text=trait,
    plural=traits,
    description={
        Un mecanismo en Rust que permite definir comportamientos compartidos entre diferentes tipos. Los traits son similares a las interfaces en otros lenguajes y permiten la implementación de polimorfismo, facilitando la reutilización de código y la abstracción
    },
}

\newglossaryentry{adt-gls}{
    name=ADT,
    text=ADT,
    plural=ADTs,
    description={
        Un Tipo de Datos Algebraico (ADT) es una estructura de datos definida por sus operaciones y comportamientos, en lugar de su implementación. Los ADTs permiten a los desarrolladores trabajar con abstracciones y encapsular detalles de implementación, facilitando la creación de programas más robustos y mantenibles
    },
}

\newglossaryentry{pattern-matching}{
    name=Pattern Matching,
    text=pattern matching,
    description={
        Una técnica de programación que permite comparar una estructura de datos con un patrón y, si coincide, extraer valores o ejecutar acciones específicas. Es comúnmente utilizada en lenguajes funcionales y proporciona una forma concisa y legible de manejar estructuras complejas
    },
}

\newglossaryentry{metaprogramacion}{
    name=Metaprogramación,
    text=metaprogramación,
    description={
        Un enfoque de programación en el que los programas pueden tratar otros programas como datos, permitiendo la creación de código que genera o manipula código. Esto permite la automatización de tareas repetitivas y la creación de abstracciones más poderosas
    },
}

\newglossaryentry{webassembly}{
    name=WebAssembly,
    text=WebAssembly,
    description={
        Un formato de código binario que permite ejecutar código de alto rendimiento en navegadores web. WebAssembly es un estándar abierto que complementa JavaScript, permitiendo a los desarrolladores escribir aplicaciones web más rápidas y eficientes en lenguajes como C, C++ y Rust
    },
}

\newglossaryentry{devops}{
    name=DevOps,
    description={
        Una práctica de desarrollo de software que combina el desarrollo (Dev) y las operaciones (Ops) para mejorar la colaboración, la automatización y la entrega continua. DevOps busca acortar el ciclo de vida del desarrollo de software y aumentar la calidad y la confiabilidad de las aplicaciones
    },
}

\newglossaryentry{skia}{
    name=Skia,
    description={
        Una biblioteca de gráficos 2D de código abierto utilizada por Google en sus productos, como Chrome y Android. Skia proporciona una API para renderizar gráficos vectoriales, texto y imágenes, y es conocida por su rendimiento y flexibilidad
    },
}

\newglossaryentry{ahead-of-time}{
    name=Ahead-of-Time (AOT),
    text=ahead-of-time,
    plural=Ahead-of-Time,
    description={
        Un enfoque de compilación en el que el código fuente se compila a código máquina antes de su ejecución, en lugar de compilarlo en tiempo de ejecución. AOT puede mejorar el rendimiento y reducir el tiempo de inicio de las aplicaciones, especialmente en entornos móviles y embebidos
    },
}

\newglossaryentry{hot-reload}{
    name=Hot Reload,
    text=hot reload,
    description={
        Una característica que permite a los desarrolladores ver los cambios en el código fuente reflejados en la aplicación en tiempo real, sin necesidad de reiniciar la aplicación. Esto acelera el proceso de desarrollo y mejora la productividad al permitir iteraciones rápidas
    },
}

\newglossaryentry{webgl}{
    name=WebGL,
    description={
        Una API de JavaScript que permite renderizar gráficos 3D en navegadores web sin necesidad de plugins. WebGL utiliza la potencia de la GPU para crear gráficos interactivos y es ampliamente utilizado en juegos, visualizaciones y aplicaciones científicas
    },
}

\newglossaryentry{lazy-loading}{
    name=Lazy Loading,
    text=lazy loading,
    description={
        Una técnica de optimización que retrasa la carga de recursos o componentes hasta que son necesarios, en lugar de cargarlos todos al inicio. Esto mejora el rendimiento y reduce el tiempo de carga inicial de las aplicaciones, especialmente en aplicaciones web y móviles
    },
}

\newglossaryentry{jetpack-compose}{
    name=Jetpack Compose,
    description={
        Un toolkit moderno de Android para construir interfaces de usuario declarativas. Utiliza un enfoque basado en componentes y permite a los desarrolladores crear UI de manera más eficiente y con menos código, aprovechando las capacidades de Kotlin
    },
}

\newglossaryentry{git}{
    name=Git,
    text=git,
    description={
        Un sistema de control de versiones distribuido que permite a los desarrolladores rastrear cambios en el código fuente a lo largo del tiempo. Git es ampliamente utilizado en proyectos de software para colaborar, gestionar versiones y mantener un historial de cambios
    },
}

\newglossaryentry{webdav}{
    name=WebDAV,
    description={
        Un protocolo de red que permite a los usuarios gestionar archivos en servidores web. WebDAV extiende el protocolo HTTP para permitir operaciones como crear, mover y eliminar archivos y directorios, facilitando la colaboración y el intercambio de archivos en línea
    },
}

\newglossaryentry{activity-pub}{
    name=ActivityPub,
    description={
        Un protocolo de comunicación descentralizado para redes sociales y aplicaciones web. Permite a los usuarios interactuar y compartir contenido entre diferentes plataformas y servicios, promoviendo la interoperabilidad y la descentralización en la web
    },
}

\newglossaryentry{bridge}{
    name=Bridge,
    text=bridge,
    description={
        Un componente que conecta dos sistemas o protocolos diferentes, permitiendo la comunicación y el intercambio de datos entre ellos. Los bridges son comunes en aplicaciones distribuidas y redes sociales para integrar diferentes servicios y plataformas
    },
}

\newglossaryentry{service-worker}{
    name=Service Worker,
    description={
        Un script que el navegador ejecuta en segundo plano, separado de una página web, permitiendo funcionalidades como la sincronización en segundo plano, notificaciones push y el manejo de solicitudes de red. Los service workers son fundamentales para crear aplicaciones web progresivas (PWA)
    },
}

\newglossaryentry{mockear}{
    name=Mockear,
    text=mockear,
    description={
        El proceso de crear versiones simuladas de objetos o servicios para probar el comportamiento de un sistema sin depender de sus implementaciones reales. Esto es útil en pruebas unitarias y de integración para aislar componentes y verificar su funcionamiento
    },
}

\newglossaryentry{ifr}{
    name=Instant First-Frame Rendering (IFR),
    text=Instant First-Frame Rendering,
    description={
        Técnica de renderizado que permite mostrar el primer fotograma de una aplicación de manera instantánea, mejorando la experiencia del usuario al reducir el tiempo de espera para ver la interfaz.
    },
}

\printglossaries

\newpage
\newacronym{foss}{FOSS}{Free and Open Source Software}
\newacronym{ia}{IA}{Inteligencia Artificial}
\newacronym{pwa}{PWA}{Progressive Web App}
\newacronym{exif}{EXIF}{Exchangeable Image File Format}
\newacronym{gc}{GC}{Garbage Collector/Recolector de Basura}
\newacronym{i-o}{I/O}{Input/Output}
\newacronym{csp}{CSP}{Comunicación Secuencial de Procesos}
\newacronym{adt}{ADT}{Algebraic Data Type/Tipo de Datos Algebraico}
\newacronym{cli}{CLI}{Command Line Interface/Herramienta de Línea de Comandos}
\newacronym{fps}{FPS}{Frames Per Second/Imágenes por Segundo}
\newacronym{acr}{ACR}{Automatic Content Recognition/Reconocimiento Automático de Contenido}
\newacronym{iptc}{IPTC}{International Press Telecommunications Council/Consejo Internacional de Telecomunicaciones de Prensa}
\newacronym{xmp}{XMP}{Extensible Metadata Platform/Plataforma de Metadatos Extensible}
\newacronym{ifracr}{IFR}{Instant First-Frame Rendering/Renderizado Instantáneo del Primer Fotograma}
\newacronym{irpf}{IRPF}{Impuesto sobre la Renta de las Personas Físicas}
\newacronym{cqrs}{CQRS}{Command Query Responsibility Segregation/Segregación de Responsabilidad de Consulta de Comando}
\newacronym{orm}{ORM}{Object-Relational Mapping/Mapeo Objeto-Relacional}


\newpage

\section{Resumen}
Tenemos que describir de manera concisa y concentrada el motivo del trabajo, objetivo y conclusiones a las que se llega.
Esta sección tiene que estar en inglés y en español.
\newpage


\newpage
\chapter*{Agradecimientos}
\addcontentsline{toc}{chapter}{Agradecimientos}
Personas que me han apoyado en la realización de este trabajo.


\newpage
~
\newpage
\section{Introducción}
% TODO: Este tiene que ser el primer capítulo del documento y los demás los sucesivos.

% TODO: Hay que añadir 2 secciones al finale de esta: Planificación y análisis de costes y estructura del documento (explicamos lo que se va a encontrar el lector en cada capítulo).

\subsection{Contexto}
% TODO: Aumentar un poco más el contexto para intentar ocupar una página entera.
Hoy en día, en la era de la digitalización, la mayoría de las personas tienen móviles con cámaras de alta resolución y grandes almacenamientos que nos permiten capturar y guardar una gran cantidad de fotografías y vídeos, espacio el cual, aunque parezca enorme, se acaba terminando. Por este motivo se ofrecen distintos tipos de servicios en la nube que permiten almacenar todos esos archivos multimedia guardados en un lugar desconocido para el usuario promedio junto con la facilidad de la sincronización automática.

Aún así, este espacio también se acaba, junto con la desventaja de que tiene un coste por guardar esos archivos en ese lugar desconocido. Este proyecto busca ofrecer una solución a este problema, mediante el desarrollo de un servicio de almacenamiento y sincronización que podrá ser alojado y desplegado en la mayoría de equipos servidores y computadoras personales.

\subsection{Motivación}
% TODO: Aumentar un poco más la motivación para intentar ocupar una página entera.
Durante un periodo vacacional, se presentó un caso práctico en el que un familiar enfrentaba dificultades debido a que su almacenamiento en Google Fotos se había acabado. Este problema planteó la necesidad de implementar un sistema que permitiera la transferencia automática de fotografías desde un dispositivo móvil a un portátil antiguo, aprovechando la conectividad de la red wifi doméstica.

Es por ello que me decidí a solucionar esta problemática mediante el diseño e implementación de una solución tecnológica eficiente y segura que permita a cualquier usuario tener un sistema de almacenamiento y sincronización de archivos multimedia entre dispositivos, haciendo uso de un servicio que se pudiera alojar y desplegar en la mayoría de los equipos.

\subsection{Objetivos}
% En infinitivo y concisos. Siguiendo las siglas SMART (Specific, Measurable, Achievable, Relevant, Time-bound).
% Mejor tener objetivos generales y después específicos.

\textbf{Objetivo general}

Desarrollar una solución multiplataforma, multiusuario y open-source para la compartición de archivos multimedia.

\textbf{Objetivos específicos}

\begin{itemize}
    \item \textbf{OE1: Análisis, diseño e implementación del sistema} \\
    Analizar, diseñar e implementar un sistema para el almacenamiento y sincronización de archivos multimedia, seleccionando los estilos y patrones arquitectónicos más adecuados (como Cliente/Servidor, Modelo-Vista-Controlador, etc.), e incorporando funcionalidades para la gestión y protección de los archivos mediante la gestión de usuarios y permisos.

    \item \textbf{OE2: Desarrollo del cliente y servidor de sincronización} \\
    Implementar tanto el cliente como el servidor encargados de la sincronización automática de archivos multimedia entre dispositivos, asegurando la compatibilidad multiplataforma y la facilidad de despliegue en diferentes entornos.
    Tanto el cliente como el servidor tendrán que ser lo más ligeros y eficientes posibles, para que puedan ser instalados en la mayoría de los equipos sin problemas.

    \item \textbf{OE3: Gestión de usuarios, permisos y seguridad} \\
    Desarrollar un sistema robusto de gestión de usuarios y permisos, incluyendo mecanismos de autenticación, autorización y cifrado de archivos, con el objetivo de garantizar la seguridad y privacidad de los datos almacenados y compartidos.

    \item \textbf{OE4: Publicación y documentación del proyecto} \\
    Publicar el código fuente del proyecto bajo una licencia open-source, asegurando la documentación exhaustiva de todos los componentes y facilitando la comprensión, uso y mantenimiento por parte de la comunidad.

    \item \textbf{OE5: Escalabilidad y mantenimiento} \\
    Diseñar el sistema con una arquitectura desacoplada que permita la escalabilidad horizontal y el mantenimiento sencillo, utilizando tecnologías y soluciones que favorezcan el crecimiento futuro del sistema.

    \item \textbf{OE6: Copias de seguridad y recuperación} \\
    Implementar mecanismos que permitan la realización de copias de seguridad y la recuperación sencilla de los archivos multimedia, garantizando la integridad y disponibilidad de los datos.

    \item \textbf{OE7: Desarrollo de aplicaciones móviles nativas} \\
    Desarrollar aplicaciones móviles nativas para las plataformas más utilizadas, priorizando el rendimiento, la experiencia de usuario y la facilidad de uso mediante interfaces intuitivas y optimizadas.
\end{itemize}


\newpage
~
\newpage
\section{Estado del arte}
Qué se ha hecho hasta ahora en este campo, qué tecnologías se han utilizado, qué problemas se han encontrado, qué soluciones se han propuesto.

% TODO:  Hablar también del estudio de las posibles soluciones, aunque después se elija otra.
%
% Añadir trabajo relacionado, hablar sobre propuestas de otras personas.
%
% Incorporar comparación de las tecnologías planteadas.
% 1. **Inmediato**: Completar análisis de tecnologías móviles
% 2. **Corto plazo**: Añadir métricas cuantitativas y trabajo relacionado
% 3. **Medio plazo**: Desarrollar análisis de arquitecturas y contribución


En este apartado se presenta un análisis del estado del arte en el ámbito de las bibliotecas de archivos multimedia de código abierto (\acrshort{foss}).
Se examinan las principales soluciones disponibles, sus características técnicas, fortalezas y limitaciones, así como las tendencias actuales en el sector.
Dado que uno de los objetivos del proyecto es desarrollar un producto que sea de código abierto, la comparación se centra en soluciones FOSS que ya están en el mercado y que han sido ampliamente adoptadas por la comunidad, lo cual nos va a permitir desarrollar una comparación mas extensa sobre cómo están organizados los proyectos para facilitar su mantenimiento y escalabilidad, así como las tecnologías que utilizan para ofrecer sus servicios.

Además, se realiza un estudio sobre las tecnologías que vamos a utilizar en el proyecto en comparación con las alternativas y las que ya se utilizan en los proyectos existentes que se analizan.

En el panorama actual de las bibliotecas de archivos multimedia de código abierto, existe una amplia variedad de soluciones que buscan ofrecer alternativas libres y gratuitas a los servicios propietarios como Google Photos o iCloud. Este análisis del estado del arte se centra en las tres soluciones más populares según el número de estrellas en GitHub: Immich, PhotoPrism y Ente.

% Fuentes:
% Google Photos: https://photos.google.com/
% Apple Photos: https://www.apple.com/ios/photos/
% Amazon Photos: https://www.amazon.com/photos
% Microsoft OneDrive Photos: https://www.microsoft.com/en-us/microsoft-365/onedrive/online-cloud-storage
\subsection{Soluciones Propietarias Relevantes}

En el ámbito de la gestión y almacenamiento de fotografías, las soluciones propietarias han marcado el estándar en cuanto a experiencia de usuario, integración de servicios y capacidades avanzadas de inteligencia artificial. Entre las plataformas más destacadas se encuentran Google Photos, Apple Photos, Amazon Photos y Microsoft OneDrive Photos, cada una con un enfoque particular y funcionalidades diferenciadoras.

Google Photos sobresale por su motor de búsqueda semántica basado en inteligencia artificial, que permite localizar imágenes mediante descripciones textuales, reconocimiento automático de objetos, lugares y personas, así como la agrupación inteligente de rostros. Además, ofrece funciones como la creación automática de álbumes, recuerdos personalizados, sugerencias de edición y generación de vídeos y animaciones a partir de colecciones de fotos. La integración con Google Assistant permite búsquedas por voz y automatización de tareas relacionadas con la gestión de imágenes.

Apple Photos, por su parte, se integra de forma nativa en el ecosistema de dispositivos Apple, ofreciendo sincronización automática y segura a través de iCloud. Destaca por sus potentes herramientas de edición no destructiva, la organización automática mediante “Memories” y “People”, y la privacidad reforzada mediante el procesamiento local de datos sensibles, como el reconocimiento facial. La integración con Siri permite búsquedas contextuales y sugerencias inteligentes.

Amazon Photos ofrece almacenamiento ilimitado de fotografías en alta resolución para suscriptores de Amazon Prime, así como detección automática de duplicados y organización por personas, lugares y objetos. Su enfoque está orientado a la simplicidad y la capacidad de compartir álbumes de forma privada o pública, además de la integración con dispositivos Amazon Echo Show para visualización mediante comandos de voz.

Microsoft OneDrive Photos integra la gestión de imágenes con el resto de servicios de productividad de Microsoft 365, facilitando la colaboración y el acceso multiplataforma. Incluye funciones de etiquetado automático, búsqueda por contenido visual y organización cronológica, así como integración con herramientas de edición en línea y sincronización con dispositivos Windows y móviles.

Entre las características avanzadas que suelen estar más desarrolladas en estas soluciones propietarias, y que pueden servir de inspiración para el desarrollo de alternativas open-source, destacan:
\begin{itemize}
    \item \textbf{Búsqueda semántica avanzada}: Localización de imágenes mediante descripciones naturales, reconocimiento de escenas, objetos y personas.
    \item \textbf{Agrupación y etiquetado inteligente}: Detección y agrupación automática de rostros, lugares y eventos.
    \item \textbf{Generación automática de recuerdos y contenido}: Creación de álbumes, vídeos y animaciones personalizadas a partir de colecciones de fotos.
    \item \textbf{Integración con asistentes virtuales}: Búsqueda y gestión de imágenes mediante comandos de voz.
    \item \textbf{Sincronización y acceso multiplataforma}: Integración transparente con diferentes dispositivos y sistemas operativos.
    \item \textbf{Herramientas avanzadas de edición}: Edición no destructiva, sugerencias automáticas y filtros inteligentes.
    \item \textbf{Privacidad y control de datos}: Procesamiento local de información sensible y opciones avanzadas de control de acceso.
\end{itemize}

Si bien algunas de estas funcionalidades comienzan a estar presentes en proyectos de código abierto, la madurez, precisión y facilidad de uso de las implementaciones propietarias sigue siendo, en muchos casos, superior debido a la inversión en inteligencia artificial, recursos computacionales y la integración profunda con sus respectivos ecosistemas. La incorporación de estas capacidades en soluciones FOSS representa un reto y una oportunidad para cerrar la brecha funcional existente.

\subsection{Panorama General de Soluciones FOSS}

El ecosistema de bibliotecas de archivos multimedia de código abierto ha experimentado un crecimiento significativo en los últimos años, impulsado por las crecientes preocupaciones sobre la privacidad de los datos y la dependencia de servicios en la nube propietarios. Según el análisis comparativo realizado por Meichthys \parencite{meichthys2024}, existen más de 16 proyectos activos que ofrecen diferentes enfoques y características.

Las soluciones analizadas se pueden clasificar en tres categorías principales:
\begin{itemize}
    \item \textbf{Soluciones escalables}: Enfocadas en escalabilidad y características avanzadas
    \item \textbf{Soluciones centradas en privacidad}: Priorizan la seguridad y el cifrado
    \item \textbf{Soluciones ligeras}: Optimizadas para recursos limitados
\end{itemize}

\subsection{Análisis de las Tres Soluciones Principales}

\subsubsection{Immich}

(\cite{immich-documentation}) Immich es una solución de gestión de fotos de código abierto orientada a usuarios que buscan una alternativa privada y autoalojada a servicios comerciales como Google Photos. Su desarrollo comenzó en 2022 y ha experimentado un rápido crecimiento gracias a una comunidad activa y a la adopción de tecnologías modernas.

\textbf{Propósito y público objetivo:} Immich está diseñado para usuarios particulares, familias y pequeños equipos que desean mantener el control sobre sus fotos y vídeos, evitando la dependencia de servicios en la nube de terceros. Es especialmente atractivo para entusiastas de la tecnología y defensores de la privacidad.

\textbf{Historia y contexto:} El proyecto nació como respuesta a la falta de alternativas libres y modernas a los grandes servicios comerciales, con un enfoque en la experiencia de usuario y la facilidad de despliegue.

\textbf{Modelo de desarrollo:} Immich es mantenido principalmente por una comunidad de desarrolladores en GitHub, con contribuciones frecuentes y una hoja de ruta pública.

\textbf{Características funcionales:}
\begin{itemize}
    \item \textbf{Facilidad de uso:} Interfaz web moderna, intuitiva y responsiva, con aplicaciones móviles para Android e iOS.
    \item \textbf{Instalación y despliegue:} Ofrece imágenes Docker y documentación clara, facilitando la instalación en servidores personales o NAS.
    \item \textbf{Soporte multiplataforma:} Disponible en web y dispositivos móviles, con sincronización automática desde el móvil.
    \item \textbf{Idiomas disponibles:} Traducción a varios idiomas, incluyendo español, gracias a la colaboración comunitaria.
    \item \textbf{Tecnologías:} Backend en NestJS/Node.js, frontend en SvelteKit, móvil en Flutter, base de datos PostgreSQL y Redis, IA con TensorFlow.
\end{itemize}

\textbf{Comunidad y ecosistema:}
\begin{itemize}
    \item \textbf{Comunidad activa:} Foros, Discord y GitHub con alta participación.
    \item \textbf{Documentación:} Completa y en constante actualización.
    \item \textbf{Extensibilidad:} API pública y soporte para integraciones futuras.
\end{itemize}

\textbf{Seguridad y privacidad:}
\begin{itemize}
    \item \textbf{Gestión de datos personales:} Los datos permanecen en el servidor del usuario, sin envíos a terceros.
    \item \textbf{Cifrado:} Actualmente no implementa cifrado de extremo a extremo, pero sí buenas prácticas de seguridad en el almacenamiento y acceso.
    \item \textbf{Actualizaciones:} Lanzamientos frecuentes y respuesta rápida a vulnerabilidades.
\end{itemize}

\textbf{Casos de uso y ejemplos reales:}
\begin{itemize}
    \item Utilizado por usuarios domésticos y pequeñas organizaciones para gestionar colecciones fotográficas privadas.
    \item Referencias y testimonios positivos en foros de autoalojamiento y privacidad.
\end{itemize}

\textbf{Limitaciones actuales:}
\begin{itemize}
    \item Alto consumo de recursos debido a Node.js.
    \item API en constante evolución (su última versión aún no se considera estable).
    \item Compatibilidad limitada con carpetas existentes.
    \item Algunas funciones avanzadas (como la edición colaborativa o el reconocimiento facial avanzado) están en desarrollo.
    \item La integración con otros servicios y dispositivos aún es limitada en comparación con soluciones comerciales.
\end{itemize}

\subsubsection{PhotoPrism}

(\cite{photoprism-documentation}) PhotoPrism es una de las soluciones FOSS más maduras y populares para la gestión de fotos, con una comunidad consolidada y un enfoque en la organización eficiente y el respeto a la privacidad.

\textbf{Propósito y público objetivo:} Orientado a usuarios que buscan una alternativa autoalojada, robusta y fácil de usar para organizar grandes colecciones de fotos, con especial atención a la preservación de metadatos y la integración con sistemas existentes.

\textbf{Historia y contexto:} Lanzado en 2017, PhotoPrism ha evolucionado para convertirse en una referencia dentro del software libre de gestión fotográfica, con un desarrollo sostenido y una base de usuarios creciente.

\textbf{Modelo de desarrollo:} Proyecto comunitario con liderazgo claro, financiado parcialmente por donaciones y patrocinios.

\textbf{Características funcionales:}
\begin{itemize}
    \item \textbf{Facilidad de uso:} Interfaz web clara y funcional, con enfoque en la organización y búsqueda eficiente.
    \item \textbf{Instalación y despliegue:} Instalación sencilla mediante Docker o binarios, compatible con múltiples sistemas operativos.
    \item \textbf{Soporte multiplataforma:} Acceso desde cualquier navegador y soporte PWA para móviles.
    \item \textbf{Idiomas disponibles:} Traducción a numerosos idiomas.
    \item \textbf{Tecnologías:} Backend en Go, frontend en Vue.js, base de datos SQLite/MariaDB, almacenamiento local con archivos sidecar.
\end{itemize}

\textbf{Comunidad y ecosistema:}
\begin{itemize}
    \item \textbf{Comunidad consolidada:} Foros, GitHub y canales de soporte activos.
    \item \textbf{Documentación:} Muy completa y detallada.
    \item \textbf{Extensibilidad:} Limitada, pero con integración básica mediante API y archivos sidecar.
\end{itemize}

\textbf{Seguridad y privacidad:}
\begin{itemize}
    \item \textbf{Gestión de datos personales:} Los datos permanecen bajo control del usuario.
    \item \textbf{Cifrado:} No implementa cifrado de extremo a extremo, pero permite despliegues seguros en redes privadas.
    \item \textbf{Actualizaciones:} Ciclo de lanzamientos estable y respuesta adecuada a incidencias.
\end{itemize}

\textbf{Casos de uso y ejemplos reales:}
\begin{itemize}
    \item Utilizado por fotógrafos, familias y pequeñas empresas para organizar y buscar fotos de forma eficiente.
    \item Referencias en comunidades de software libre y autoalojamiento.
\end{itemize}

\textbf{Limitaciones actuales:}
\begin{itemize}
    \item Soporte limitado para múltiples usuarios.
    \item Escalabilidad horizontal restringida.
    \item Aplicaciones móviles limitadas a PWA, sin apps nativas.
    \item No utiliza una estructura estandarizada para la organización, dificultando la contribución de terceros.
\end{itemize}

\subsubsection{Ente}

Ente destaca por su enfoque radical en la privacidad y la seguridad, ofreciendo cifrado de extremo a extremo y una experiencia multiplataforma coherente gracias a Flutter.

\textbf{Propósito y público objetivo:} Dirigido a usuarios que priorizan la privacidad y la seguridad de sus fotos, como periodistas, activistas o cualquier persona preocupada por la confidencialidad de sus datos.

\textbf{Historia y contexto:} Proyecto joven pero con rápido crecimiento, impulsado por la demanda de soluciones seguras y privadas en el ámbito de la gestión fotográfica.

\textbf{Modelo de desarrollo:} Comunidad activa y transparente, con desarrollo abierto y enfoque en la seguridad.

\textbf{Características funcionales:}
\begin{itemize}
    \item \textbf{Facilidad de uso:} Interfaz moderna y coherente en todas las plataformas.
    \item \textbf{Instalación y despliegue:} Sencillo para usuarios finales, con apps en tiendas oficiales y opción de autoalojamiento.
    \item \textbf{Soporte multiplataforma:} Apps nativas para móvil y escritorio, y versión web.
    \item \textbf{Idiomas disponibles:} Traducción en expansión gracias a la comunidad.
    \item \textbf{Tecnologías:} Backend en Go, frontend y apps en Flutter, base de datos PostgreSQL, cifrado client-side.
\end{itemize}

\textbf{Comunidad y ecosistema:}
\begin{itemize}
    \item \textbf{Comunidad creciente:} Participación activa en GitHub y foros.
    \item \textbf{Documentación:} Clara y orientada a la seguridad.
    \item \textbf{Extensibilidad:} Limitada por el enfoque en la privacidad.
\end{itemize}

\textbf{Seguridad y privacidad:}
\begin{itemize}
    \item \textbf{Gestión de datos personales:} Cifrado de extremo a extremo, arquitectura de conocimiento cero.
    \item \textbf{Cifrado:} Todos los datos se cifran en el cliente antes de ser almacenados.
    \item \textbf{Actualizaciones:} Rápida respuesta a vulnerabilidades y mejoras continuas en seguridad.
\end{itemize}

\textbf{Casos de uso y ejemplos reales:}
\begin{itemize}
    \item Adoptado por usuarios preocupados por la privacidad y organizaciones que manejan información sensible.
    \item Recomendado en foros de privacidad y seguridad digital.
\end{itemize}

\textbf{Limitaciones actuales:}
\begin{itemize}
    \item Overhead del cifrado afecta el rendimiento.
    \item Características de IA limitadas por la privacidad.
    \item No se utiliza ninguna estructura de repositorio, dificultando la aportación al proyecto.
\end{itemize}

\subsection{Comparación Técnica Detallada}

\begin{table}[H]
\centering
\begin{tabular}{|l|c|c|c|}
\hline
\textbf{Característica} & \textbf{Immich} & \textbf{PhotoPrism} & \textbf{Ente} \\
\hline
Estrellas GitHub & 66,988 & 37,499 & 19,431 \\
Lenguaje Principal & TypeScript & Go & \gls{dart} \\
Arquitectura & Microservicios & Monolítica & Cliente-Servidor \\
Base de Datos & PostgreSQL & SQLite/MySQL & PostgreSQL \\
Aplicación Móvil & Flutter (8/10) & PWA (4/10) & Flutter (8/10) \\
Reconocimiento IA & 9/10 & 9/10 & En desarrollo \\
Múltiples Usuarios & 8/10 & No soportado & 9/10 \\
Búsqueda & 9/10 & 8/10 & 6/10 \\
Privacidad & 7/10 & 8/10 & 10/10 \\
Facilidad de Contribución & Alta & Baja & Baja-Media \\
\hline
\end{tabular}
\caption{Comparación técnica de las principales soluciones FOSS}
\label{tab:tech_comparison}
\end{table}

A continuación se explican brevemente las características comparadas en la tabla anterior:

\textbf{Estrellas GitHub:} Indica la popularidad y el nivel de interés de la comunidad en cada proyecto, reflejando tanto la visibilidad como la actividad de usuarios y desarrolladores.

\textbf{Lenguaje Principal:} Se refiere al lenguaje de programación predominante en el desarrollo del proyecto, lo que puede influir en la facilidad de contribución, el rendimiento y la compatibilidad con otras tecnologías.

\textbf{Arquitectura:} Describe el enfoque estructural del software. Una arquitectura de microservicios facilita la escalabilidad y el mantenimiento, mientras que una arquitectura monolítica puede simplificar el despliegue. El modelo cliente-servidor, por su parte, suele estar orientado a la seguridad y la separación de responsabilidades.

\textbf{Base de Datos:} Indica el sistema de gestión de bases de datos utilizado para almacenar la información. La elección de la base de datos afecta el rendimiento, la escalabilidad y la facilidad de integración con otros sistemas.

\textbf{Aplicación Móvil:} Evalúa la calidad y el tipo de experiencia móvil ofrecida, diferenciando entre aplicaciones nativas (mayor rendimiento y funcionalidad) y aplicaciones web progresivas (PWA), que suelen ser más limitadas.

\textbf{Reconocimiento IA:} Hace referencia a las capacidades de inteligencia artificial para el reconocimiento de imágenes, como la detección de rostros, objetos o escenas, y su grado de madurez en cada solución.

\textbf{Múltiples Usuarios:} Indica el soporte para la gestión de varios usuarios en la misma instancia, lo que es relevante para entornos familiares, de equipos o empresas.

\textbf{Búsqueda:} Evalúa la potencia y precisión de los mecanismos de búsqueda y filtrado de imágenes, incluyendo la búsqueda por metadatos, texto o reconocimiento automático.

\textbf{Privacidad:} Analiza el nivel de protección de los datos del usuario, considerando aspectos como el cifrado, la arquitectura de conocimiento cero y la gestión local de la información.

\textbf{Facilidad de Contribución:} Se refiere a la facilidad con la que nuevos desarrolladores pueden colaborar en el proyecto, influida por la calidad de la documentación y la modularidad del código.

\subsection{Análisis de Rendimiento}

El análisis de rendimiento revela diferentes enfoques optimizados para casos de uso específicos:

\textbf{Immich} sobresale en escalabilidad y características avanzadas, pero requiere más recursos del servidor. Su arquitectura de microservicios permite el escalado horizontal y la distribución de carga de trabajo, especialmente beneficial para el procesamiento de IA.

\textbf{PhotoPrism} ofrece la mejor eficiencia de recursos gracias a Go y su arquitectura monolítica, siendo ideal para instalaciones en hardware limitado. Su procesamiento de archivos es especialmente eficiente para fotógrafos profesionales.

\textbf{Ente} equilibra el rendimiento con la seguridad, trasladando el procesamiento al cliente para mantener la privacidad, aunque esto introduce latencia en las operaciones de cifrado/descifrado.

\subsection{Tendencias y Evolución del Sector}

El análisis del estado del arte revela varias tendencias importantes:

\begin{enumerate}
    \item \textbf{Migración hacia arquitecturas modernas}: Immich lidera esta tendencia con TypeScript y microservicios
    \item \textbf{Enfoque en la privacidad}: Ente representa la vanguardia en soluciones privacy-first
    \item \textbf{Madurez y estabilidad}: PhotoPrism ejemplifica la importancia de la estabilidad a largo plazo
    \item \textbf{Convergencia móvil}: Flutter se está estableciendo como la tecnología preferida para aplicaciones móviles
\end{enumerate}

\subsection{Conclusiones}

El ecosistema de bibliotecas de archivos multimedia FOSS presenta tres paradigmas distintos que abordan diferentes necesidades del mercado. Immich representa la innovación y escalabilidad, PhotoPrism la madurez y eficiencia, mientras que Ente pionerea en privacidad y seguridad.

Esta diversidad de enfoques indica un mercado en maduración donde no existe una solución única, sino que cada proyecto optimiza para casos de uso específicos. La elección entre estas soluciones depende fundamentalmente de los requisitos de escalabilidad, privacidad, recursos disponibles y experiencia técnica del usuario final.

El análisis sugiere que el futuro del sector se dirigirá hacia la convergencia de estas características, buscando soluciones que combinen la escalabilidad de Immich, la eficiencia de PhotoPrism y la privacidad de Ente.

Además, las nuevas tecnologías como pueden ser \gls{react-native} o en este caso \gls{lynxjs}, pueden aportar un mejor desarrollo de una aplicación móvil nativa, que permita una mejor experiencia de usuario y un mejor rendimiento en dispositivos móviles sin perder características nativas de las plataformas.

No se ve en ningún proyecto entre los más populares que haga uso de \gls{rust}. Aunque hay una tendencia a utilizar Go, el cual es un lenguaje que es más sencillo y proporciona un rendimiento muy bueno, Rust ofrece ventajas significativas en términos de seguridad y rendimiento, especialmente para aplicaciones que requieren un alto grado de concurrencia y eficiencia en el manejo de memoria. Esto sugiere una oportunidad para explorar Rust como una alternativa viable para el desarrollo de bibliotecas de archivos multimedia FOSS en el futuro.

Por lo general, se suele hacer uso de un almacenamiento de objetos ya implementado, como puede ser \gls{minio}, que es compatible con \gls{s3}. Esto permite una mayor escalabilidad y flexibilidad en el almacenamiento de grandes volúmenes de datos, lo cual es esencial para aplicaciones que manejan grandes bibliotecas de archivos multimedia.
Aún así, sería interesante explorar la posibilidad de implementar un almacenamiento de objetos propio, ya que esto podría ofrecer ventajas en términos de personalización y optimización para casos de uso específicos como puede ser el caso en el que ya se tiene un almacenamiento de archivos en el servidor y se quiere aprovechar para almacenar las fotos.


\subsection{Análisis de Costes}

El análisis de costes es un aspecto fundamental a la hora de evaluar las diferentes soluciones de bibliotecas de archivos multimedia FOSS. Este análisis abarca varios aspectos que impactan directamente en el coste total de implementación y mantenimiento.

\subsubsection{Requisitos de Hardware}

Los requisitos de hardware varían significativamente entre las diferentes soluciones:

\begin{itemize}
    \item \textbf{Immich}:
    \begin{itemize}
        \item Rendimiento óptimo con procesador moderno (por ejemplo, Intel NUC de 12ª generación)
        \item Mínimo 16GB de RAM recomendado para bibliotecas grandes
        \item Almacenamiento SSD para la base de datos y caché
        \item Almacenamiento NAS/HDD para los archivos originales
    \end{itemize}

    \item \textbf{PhotoPrism}:
    \begin{itemize}
        \item Requiere hardware más potente para un rendimiento similar
        \item Mínimo 16GB de RAM, preferiblemente 32GB para bibliotecas grandes
        \item GPU integrada recomendada para aceleración por hardware
        \item SSD local para miniaturas y caché
    \end{itemize}

    \item \textbf{Ente}:
    \begin{itemize}
        \item Requisitos moderados debido al cifrado client-side
        \item 8-16GB de RAM suficientes para la mayoría de casos
        \item Menor dependencia de GPU
    \end{itemize}
\end{itemize}

\subsubsection{Consumo de Recursos}

El consumo de recursos afecta directamente a los costes operativos:

\begin{itemize}
    \item \textbf{Almacenamiento}:
    \begin{itemize}
        \item Immich: Genera dos miniaturas por imagen, altamente configurable
        \item PhotoPrism: Genera 8 miniaturas JPEG con diferentes relaciones de aspecto
        \item Ente: Uso eficiente del almacenamiento debido al cifrado
    \end{itemize}

    \item \textbf{Procesamiento}:
    \begin{itemize}
        \item Immich: Procesamiento paralelo eficiente, menor impacto en CPU
        \item PhotoPrism: Mayor uso de CPU durante la indexación y procesamiento
        \item Ente: Sobrecarga adicional debido al cifrado
    \end{itemize}
\end{itemize}

\subsubsection{Costes de Mantenimiento}

Los costes de mantenimiento incluyen:

\begin{itemize}
    \item \textbf{Actualizaciones y Parches}:
    \begin{itemize}
        \item Immich: Actualizaciones frecuentes, proceso automatizado
        \item PhotoPrism: Desarrollo más lento, actualizaciones menos frecuentes
        \item Ente: Ciclo de desarrollo moderado
    \end{itemize}

    \item \textbf{Backup y Redundancia}:
    \begin{itemize}
        \item Todas las soluciones requieren estrategias de backup
        \item Costes adicionales para almacenamiento redundante
        \item Necesidad de monitorización y mantenimiento regular
    \end{itemize}
\end{itemize}

\subsubsection{Costes de Escalabilidad}

La escalabilidad impacta en los costes a largo plazo:

\begin{itemize}
    \item \textbf{Vertical}:
    \begin{itemize}
        \item Immich: Mejor rendimiento con hardware modesto
        \item PhotoPrism: Puede requerir actualizaciones de hardware más frecuentes
        \item Ente: Escalado limitado por el cifrado
    \end{itemize}

    \item \textbf{Horizontal}:
    \begin{itemize}
        \item Immich: Arquitectura de microservicios facilita la escalabilidad
        \item PhotoPrism: Limitaciones en escalabilidad horizontal
        \item Ente: Diseño centrado en privacidad limita opciones de escalado
    \end{itemize}
\end{itemize}

\subsubsection{Modelos de Monetización}

Es importante considerar los diferentes modelos de monetización:

\begin{itemize}
    \item \textbf{Immich}:
    \begin{itemize}
        \item Completamente gratuito y open source
        \item Sin funciones premium o de pago
        \item Financiado por donaciones y comunidad
    \end{itemize}

    \item \textbf{PhotoPrism}:
    \begin{itemize}
        \item Modelo freemium
        \item Algunas características avanzadas requieren suscripción
        \item Membresías Plus y Pro disponibles
    \end{itemize}

    \item \textbf{Ente}:
    \begin{itemize}
        \item Enfoque en privacidad y seguridad
        \item Modelo de negocio basado en donaciones
    \end{itemize}
\end{itemize}

\subsection{Análisis Comparativo de Métricas}

A partir de pruebas realizadas por la comunidad y documentación oficial, se presenta un análisis detallado de las métricas de rendimiento y capacidades de las tres soluciones principales.

\subsubsection{Rendimiento de Procesamiento}

\begin{table}[H]
\centering
\begin{tabular}{|l|c|c|c|}
\hline
\textbf{Métrica} & \textbf{Immich} & \textbf{PhotoPrism} & \textbf{Ente} \\
\hline
Velocidad de Indexación & Alta & Media-Baja & Media \\
Tiempo de Carga UI & $<$1s & 1-2s & $<$1s \\
Procesamiento Paralelo & Sí & No & Parcial \\
Uso de CPU & Moderado & Alto & Bajo \\
\hline
\end{tabular}
\caption{Comparación de métricas de rendimiento}
\label{tab:performance_metrics}
\end{table}

\subsubsection{Gestión de Almacenamiento}

\begin{itemize}
    \item \textbf{Miniaturas por Imagen}:
    \begin{itemize}
        \item Immich: 2 miniaturas configurables
        \item PhotoPrism: 8 miniaturas JPEG (diferentes relaciones de aspecto)
        \item Ente: Optimizado para cifrado, número variable
    \end{itemize}

    \item \textbf{Espacio de Caché}:
    \begin{itemize}
        \item Immich: Aproximadamente 1.5x el tamaño original
        \item PhotoPrism: 2-3x el tamaño original
        \item Ente: 1-1.5x el tamaño original
    \end{itemize}
\end{itemize}

\subsubsection{Precisión de Reconocimiento}

\begin{table}[H]
\centering
\begin{tabular}{|l|c|c|c|}
\hline
\textbf{Característica} & \textbf{Immich} & \textbf{PhotoPrism} & \textbf{Ente} \\
\hline
Detección Facial & 90\% & 60\% & 85\% \\
Reconocimiento de Objetos & 85\% & 70\% & No disponible \\
Agrupación de Fotos & Excelente & Buena & Limitada \\
Precisión de Búsqueda & Alta & Media & Media \\
\hline
\end{tabular}
\caption{Comparación de precisión en características de IA}
\label{tab:ai_accuracy}
\end{table}

\subsubsection{Tiempos de Procesamiento}

Para una biblioteca de prueba de 40,000 imágenes (aproximadamente 1.6TB):

\begin{itemize}
    \item \textbf{Immich}:
    \begin{itemize}
        \item Importación inicial: 2-3 horas
        \item Indexación completa: 4-6 horas
        \item Generación de miniaturas: 3-4 horas
        \item Procesamiento de IA: 6-8 horas
    \end{itemize}

    \item \textbf{PhotoPrism}:
    \begin{itemize}
        \item Importación inicial: 4-5 horas
        \item Indexación completa: 24-48 horas
        \item Generación de miniaturas: 8-10 horas
        \item Procesamiento de IA: 12-24 horas
    \end{itemize}

    \item \textbf{Ente}:
    \begin{itemize}
        \item Importación inicial: 3-4 horas
        \item Indexación completa: 8-10 horas
        \item Generación de miniaturas: 4-5 horas
        \item Procesamiento de IA: No aplicable
    \end{itemize}
\end{itemize}

\subsubsection{Métricas de Escalabilidad}

\begin{itemize}
    \item \textbf{Límites de Biblioteca}:
    \begin{itemize}
        \item Immich: Probado con $>$1 millón de fotos
        \item PhotoPrism: Recomendado hasta 500,000 fotos
        \item Ente: Probado con $>$250,000 fotos
    \end{itemize}

    \item \textbf{Rendimiento con Carga}:
    \begin{itemize}
        \item Immich: Mantiene rendimiento con múltiples usuarios
        \item PhotoPrism: Degradación notable con múltiples usuarios
        \item Ente: Rendimiento constante pero limitado
    \end{itemize}
\end{itemize}

\subsubsection{Soporte de Formatos}

\begin{table}[H]
\centering
\begin{tabular}{|l|c|c|c|}
\hline
\textbf{Formato} & \textbf{Immich} & \textbf{PhotoPrism} & \textbf{Ente} \\
\hline
JPEG & Sí & Sí & Sí \\
RAW & Parcial & Completo & No \\
HEIC & Sí & Sí & Parcial \\
Videos & Sí & Sí & Limitado \\
Live Photos & Sí & No & No \\
\hline
\end{tabular}
\caption{Comparación de soporte de formatos}
\label{tab:format_support}
\end{table}

\subsection{Interoperabilidad y Estándares Abiertos}

La interoperabilidad y el uso de estándares abiertos son factores clave para la adopción y sostenibilidad de las bibliotecas de archivos multimedia FOSS. Permiten a los usuarios migrar fácilmente sus datos, integrar la aplicación con otros servicios y evitar el "vendor lock-in". A continuación se analiza cómo las principales soluciones abordan estos aspectos:

\begin{itemize}
    \item \textbf{Integración con servicios externos}: Immich y PhotoPrism ofrecen soporte para almacenamiento externo mediante protocolos como \gls{webdav} y S3 (Amazon Simple Storage Service o compatibles como MinIO), facilitando la integración con soluciones de almacenamiento en la nube o NAS. Ente, por su enfoque en privacidad, limita la integración a servicios que no comprometan la seguridad de los datos.
    \item \textbf{Soporte de estándares de metadatos}: PhotoPrism destaca por su soporte completo de estándares como EXIF, \acrshort{iptc} y \acrshort{xmp} para metadatos, permitiendo una mejor interoperabilidad con otras aplicaciones de gestión de fotos. Immich y Ente ofrecen soporte parcial, centrado principalmente en EXIF.
    \item \textbf{Exportación e importación}: Todas las soluciones permiten la exportación e importación de fotos y metadatos, aunque el grado de automatización y compatibilidad varía. PhotoPrism facilita la migración mediante herramientas específicas y documentación detallada. Immich permite la importación desde carpetas existentes, pero la exportación masiva puede requerir herramientas adicionales. Ente prioriza la exportación cifrada para mantener la privacidad.
    \item \textbf{Federación y APIs}: Ninguna de las soluciones principales implementa actualmente protocolos de federación como \gls{activity-pub}, aunque existen discusiones en la comunidad sobre su posible adopción futura. Todas ofrecen APIs REST para integración con aplicaciones de terceros, aunque el grado de estabilidad y documentación varía (Immich tiene una API en evolución, PhotoPrism una API más estable).
\end{itemize}

La adopción de estándares abiertos y la interoperabilidad son aspectos en los que aún existe margen de mejora en el sector. Fomentar la compatibilidad con protocolos y formatos ampliamente aceptados facilitará la integración con otros sistemas y la migración de datos, beneficiando tanto a usuarios finales como a desarrolladores.


\subsection{Conclusiones}
\subsubsection{Conclusiones del Análisis de Métricas}

El análisis de métricas revela que Immich destaca en varios aspectos clave:

\begin{itemize}
    \item Mayor eficiencia en procesamiento paralelo
    \item Mejor equilibrio entre almacenamiento y funcionalidad
    \item Superior precisión en reconocimiento facial y de objetos
    \item Mejor escalabilidad para bibliotecas grandes
\end{itemize}

PhotoPrism, aunque más maduro, muestra limitaciones en rendimiento y escalabilidad, mientras que Ente destaca en privacidad pero con funcionalidades más limitadas. Estas métricas refuerzan la elección de tecnologías y arquitectura para nuestro proyecto, que busca combinar la eficiencia de Immich con mejoras adicionales en rendimiento y funcionalidad.

\subsubsection{Conclusiones del Análisis de Costes}

El análisis de costes revela que Immich ofrece la mejor relación coste-beneficio para la mayoría de los casos de uso:

\begin{itemize}
    \item Requisitos de hardware más moderados
    \item Mejor eficiencia en el uso de recursos
    \item Arquitectura que facilita la escalabilidad
    \item Modelo completamente gratuito sin costes ocultos
\end{itemize}

Sin embargo, la elección final dependerá de factores específicos como:
\begin{itemize}
    \item Tamaño de la biblioteca de fotos
    \item Recursos de hardware disponibles
    \item Necesidades de escalabilidad
    \item Presupuesto para mantenimiento
\end{itemize}

Para nuestro proyecto, el análisis de costes refuerza la decisión de utilizar tecnologías eficientes como Rust y Lynx.js, que permitirán minimizar los requisitos de hardware y los costes operativos mientras se mantiene un alto nivel de rendimiento.

\subsubsection{Conclusiones Interoperabilidad y Estándares Abiertos}
El análisis de interoperabilidad y estándares abiertos muestra que Immich y PhotoPrism son las soluciones más avanzadas en este aspecto, con un enfoque claro en la integración con servicios externos y el uso de metadatos estandarizados. Ente, aunque prioriza la privacidad, limita la interoperabilidad al restringir la integración con servicios externos.

En nuestra propuesta buscamos fomentar la interoperabilidad mediante el uso de estándares abiertos y APIs bien documentadas, lo que permitirá a los usuarios migrar fácilmente sus datos y contribuir al proyecto. Esto es esencial para garantizar la sostenibilidad a largo plazo y la adopción por parte de la comunidad.
Gracias a esto se busca que el proyecto no solo sea una solución de gestión de fotos, sino también un ecosistema abierto y colaborativo que permita a los usuarios y desarrolladores contribuir y beneficiarse mutuamente, además de poder integrar la aplicación con otros servicios y herramientas existentes.

\subsection{Aportaciones de nuestro proyecto al estado del arte}
Una vez realizado un estudio de las soluciones existentes y las tecnologías más utilizadas en el desarrollo de aplicaciones móviles y web, se ha identificado una clara oportunidad para mejorar la experiencia del usuario en la gestión de bibliotecas de archivos multimedia.

Nuestro proyecto busca abordar las limitaciones actuales de las aplicaciones de fotos, especialmente en términos de rendimiento, usabilidad, facilidad de aportación al proyecto y características avanzadas.
\subsubsection{Rendimiento}
El rendimiento es un aspecto crítico en las aplicaciones de fotos, especialmente cuando se manejan grandes colecciones. Muchas aplicaciones existentes sufren de lentitud en la carga y visualización de imágenes, lo que afecta negativamente la experiencia del usuario.
No solo se busca optimizar la carga de imágenes en la aplicación, sino mejorar la velocidad de respuesta que ofrece el servidor a la hora de recibir, procesar y responder a las peticiones de los usuarios.

Todo esto lo conseguiremos gracias a:
\begin{itemize}
    \item Aplicación móvil programada en Lynx.js, que permite un rendimiento optimizado mediante el uso de componentes específicos para la carga de datos e imágenes, así como un acceso más directo a las APIs nativas del dispositivo. Aunque es una tecnología emergente, utiliza JavaScript/TypeScript con su propia variante de React, lo que facilita la curva de aprendizaje para desarrolladores familiarizados con estas tecnologías.
    \item Servidor programado en Rust, que ofrece alto rendimiento, eficiencia en la gestión de recursos y acceso a optimizaciones a bajo nivel, lo que se traduce en una respuesta más rápida a las peticiones de los usuarios. Usar Rust nos va a proporcionar un código más seguro y eficiente, evitando errores comunes como las condiciones de carrera o los punteros nulos sin necesidad de un recolector de basura (GC).
\end{itemize}
\subsubsection{Contribución al proyecto}
Nuestro proyecto no solo se centrará en ofrecer una aplicación de fotos, sino que también se diseñará como un proyecto FOSS, lo que permitirá a la comunidad contribuir y mejorar la aplicación de manera continua.

Para ello se hará uso de las mejoras prácticas de desarrollo de software:
\begin{itemize}
    \item Uso de control de versiones (\Gls{git}), documentación clara y accesible, y un proceso de revisión de código que fomente la colaboración y la calidad del código.
    \item Se implementará una estructura de proyecto que facilite la incorporación de nuevos desarrolladores, con guías claras sobre cómo contribuir y estándares de codificación.
    \item El uso de Rust también aporta valor a nuestro proyecto al ofrecer un lenguaje moderno y seguro que minimiza los errores comunes de programación, lo que facilita la colaboración y la contribución de nuevos desarrolladores, eliminando posibles errores en la aplicación incluso antes de que se ejecute.
    \item Utilizaremos TypeScript en la aplicación móvil, lo que permitirá a los desarrolladores familiarizados con JavaScript contribuir fácilmente al proyecto, lo que facilitará la incorporación de nuevos colaboradores y fomentará una comunidad activa.
    \item Implementación de pruebas unitarias y de integración para asegurar la calidad del código y la funcionalidad de la aplicación.
\end{itemize}


~
\chapter{Análisis de Tecnologías}
\label{sec:tecnologias}


Tal como se ha mencionado en el capítulo anterior, hay una tendencia general a usar TypeScript o Go para el desarrollo de el servidor de sincronización de multimedia y Flutter o PWA para el desarrollo de aplicaciones móviles debido a su popularidad y facilidad de uso.
Sin embargo, existen otras tecnologías que pueden ofrecer ventajas significativas en términos de rendimiento, seguridad y facilidad de desarrollo.

Se estudiarán los diferentes lenguajes y tecnologías como posibles candidatos para el desarrollo del servidor, entre ellos Rust, Go, C, C++, Python, Java y Ruby on Rails. Cada uno de estos lenguajes tiene sus propias ventajas y desventajas, y la elección del lenguaje adecuado dependerá de los requisitos específicos del proyecto, como la necesidad de alto rendimiento, seguridad en la gestión de memoria y facilidad de desarrollo.

También se estudiarán las tecnologías disponibles para el desarrollo de la aplicación móvil, entre ellas React Native, Flutter, PWA y Lynx.js. Cada una de estas tecnologías tiene sus propias ventajas y desventajas, y la elección de la tecnología adecuada dependerá de los requisitos específicos del proyecto, como la necesidad de una experiencia de usuario nativa, el rendimiento en dispositivos móviles y la facilidad de desarrollo.

\section{Almacenamiento de archivos}

Para el almacenamiento de los archivos multimedia tenemos dos opciones: un almacenamiento local o un almacenamiento en la nube.

\subsection{Almacenamiento local en servidor}
Esta opción implica alojar físicamente los archivos en los discos duros del servidor que ejecuta la aplicación.

\begin{itemize}
    \item \textbf{Ventajas:}
    \begin{itemize}
        \item \textbf{Control total}: Se tiene el control absoluto sobre los datos, la seguridad y la infraestructura.
        \item \textbf{Latencia baja}: La velocidad de acceso a los archivos es extremadamente alta, ya que no hay dependencias de la red externa.
        \item \textbf{Coste predecible}: Una vez adquirido el hardware de almacenamiento, los costes operativos son fijos. No hay gastos variables por el uso de ancho de banda o por la cantidad de datos almacenados.
    \end{itemize}

    \item \textbf{Desventajas:}
    \begin{itemize}
        \item \textbf{Escalabilidad limitada}: La capacidad de almacenamiento está restringida por el hardware físico. Escalar requiere la compra e instalación de más discos duros.
        \item \textbf{Coste inicial alto}: Requiere una inversión inicial significativa en hardware (servidores y almacenamiento).
        \item \textbf{Mantenimiento y seguridad}: Se es responsable de todo el mantenimiento, las copias de seguridad, la redundancia de datos (\texttt{RAID}) y la protección contra fallos.
        \item \textbf{Implementación de módulo de almacenamiento}: Hay que implementar el módulo que se va a encargar del almacenamiento de los archivos de forma local.
    \end{itemize}
\end{itemize}

\subsection{Almacenamiento en la nube}
Esta opción utiliza servicios de terceros como \textbf{Amazon S3}, \textbf{Google Cloud Storage} o \textbf{Microsoft Azure Blob Storage} para almacenar los archivos. Estos servicios se basan en la arquitectura de almacenamiento de objetos, optimizada para grandes volúmenes de datos no estructurados utilizando el protocolo de comunicación para transferencia de archivos S3.

\begin{itemize}
\item \textbf{Ventajas:}
    \begin{itemize}
        \item \textbf{Escalabilidad ilimitada}: Se puede almacenar prácticamente cualquier cantidad de datos sin preocuparse por la capacidad del hardware.
        \item \textbf{Coste flexible}: Se paga por el almacenamiento y el ancho de banda consumido, lo que lo hace ideal para proyectos con un crecimiento variable.
        \item \textbf{Fiabilidad y durabilidad}: Estos servicios ofrecen alta disponibilidad y durabilidad de los datos, con redundancia automática y copias de seguridad integradas.
        \item \textbf{Funcionalidades adicionales}: Suelen incluir características avanzadas como gestión de versiones, seguridad (acceso controlado) y replicación global.
        \item \textbf{Mismo protocolo de comunicación para varias opciones}: Hay varias opciones para cada protocolo de almacenamiento, lo que permitirá al usuario no tener que depender únicamente de un proveedor. Por ejemplo, el protocolo que utiliza Amazon (\gls{s3}) tiene varias opciones que son totalmente compatibles con ese mismo protocolo, como pueden ser \gls{minio}, \href{https://www.digitalocean.com/products/spaces}{DigitalOcean Spaces} o \href{https://wasabi.com/es}{Wasabi}, por lo que el usuario tan solo necesitaría cambiar los datos de conexión con los servicios en la configuración de la aplicación para cambiar de proveedor.
    \end{itemize}
\item \textbf{Desventajas:}
\begin{itemize}
\item \textbf{Coste recurrente}: Los gastos se acumulan con el uso, lo que puede ser un inconveniente a largo plazo para un proyecto con un gran volumen de usuarios.
\item \textbf{Dependencia de terceros}: Se depende de la disponibilidad y políticas del proveedor de servicios en la nube.
\item \textbf{Latencia variable}: El acceso a los archivos depende de la red, lo que puede introducir una latencia mayor que en el almacenamiento local.
\end{itemize}
\end{itemize}

Aunque hablamos de almacenamiento en la nube, existe la opción de utilizar una solución que imite el almacenamiento en la nube pero de forma local, como puede ser \gls{minio}, que es una solución de almacenamiento de objetos compatible con el protocolo S3 que se puede instalar en un servidor propio. De esta forma, se pueden aprovechar las ventajas del almacenamiento en la nube (escalabilidad, fiabilidad, funcionalidades adicionales) sin depender de un proveedor externo.

\section{Servidor}
El servidor de sincronización de multimedia es el componente central de la aplicación, encargado de gestionar la comunicación entre el cliente y el almacenamiento de fotos.
Disponemos de una amplia variedad a la hora de elegir un lenguaje/\gls{framework} para el desarrollo del servidor, cada uno con sus propias ventajas y desventajas.

Las principales soluciones serían:
\begin{itemize}
    \item \textbf{Node.js con TypeScript}: Muy popular, especialmente para aplicaciones web. Ofrece un ecosistema rico y una gran comunidad, siendo menos eficiente en términos de rendimiento y consumo de recursos con respecto a su competencia pero con una curva de aprendizaje más suave y una gran cantidad de bibliotecas disponibles.
    \item \textbf{Go}: Conocido por su eficiencia y facilidad de uso en aplicaciones concurrentes. Es una opción sólida para aplicaciones que requieren alto rendimiento y escalabilidad.
    \item \textbf{Rust}: Ofrece un alto rendimiento y seguridad en la gestión de memoria. Aunque tiene una curva de aprendizaje más pronunciada, es ideal para aplicaciones que requieren alta concurrencia y eficiencia.
    \item \textbf{C}: Lenguaje de bajo nivel con alto rendimiento, pero su complejidad y la gestión manual de memoria lo hacen menos adecuado para aplicaciones web modernas.
    \item \textbf{C++}: Similar a C, pero con características de programación orientada a objetos. Ofrece un alto rendimiento, pero su complejidad y la gestión manual de memoria lo hacen menos adecuado para aplicaciones web modernas.
    \item \textbf{Python}: Muy utilizado en el ámbito de la ciencia de datos y aprendizaje automático, pero menos eficiente en términos de rendimiento y escalabilidad para aplicaciones web, además de contar con un \gls{tipado-dinamico} que puede llevar a errores en tiempo de ejecución.
    \item \textbf{Java}: Aunque es robusto y escalable, su complejidad y consumo de recursos lo hacen menos atractivo para aplicaciones ligeras.
    \item \textbf{Ruby on Rails}: Muy popular para aplicaciones web, pero menos eficiente en términos de rendimiento y escalabilidad. No se ha trabajado mucho con esta tecnología, por lo que no se tiene una experiencia directa con ella.
\end{itemize}

A continuación se muestra una comparativa de rendimiento de los lenguajes anteriormente mencionados, en la cual se han medido el tiempo de CPU y el uso de memoria en tres algoritmos diferentes: Bubble Sort, Monte Carlo Pi y Monte Carlo Pi con un generador de números aleatorios simple (\texttt{SimpleRNG}).

\begin{figure}[H]
  \centering
  \includegraphics[width=0.8\textwidth]{assets/rust-cpu-comparison.png}
  \caption{Comparativa de uso de CPU entre C, C++, Go, Java, Python y Rust (menos es mejor) \parencite{rust-for-safety-and-performance}}
  \label{fig:rust-cpu-comparison}
\end{figure}


\begin{figure}[H]
  \centering
  \includegraphics[width=0.8\textwidth]{assets/rust-memory-comparison.png}
  \caption{Comparativa de uso de memoria entre C, C++, Go, Java, Python y Rust (menos es mejor) \parencite{rust-for-safety-and-performance}}
  \label{fig:rust-memory-comparison}
\end{figure}

Como se puede ver en las Figuras \ref{fig:rust-cpu-comparison} y \ref{fig:rust-memory-comparison}, Rust tiene un rendimiento muy bueno en comparación con otros lenguajes de programación, tanto en uso de CPU como en uso de memoria.

A partir de las Figuras, podemos definir dos fórmulas para calcular el porcentaje de mejora con respecto a los otros lenguajes de programación:
\begin{equation}
    \text{Mejora}_{\text{CPU}} (\%) = \left( \frac{T_{\text{\textit{lenguaje}}} - T_{\text{Rust}}}{T_{\text{\textit{lenguaje}}}} \right) \times 100
\end{equation}

\begin{equation}
    \text{Mejora}_{\text{Memoria}} (\%) = \left( \frac{M_{\text{\textit{lenguaje}}} - M_{\text{Rust}}}{M_{\text{\textit{lenguaje}}}} \right) \times 100
\end{equation}


Cogiendo Rust como referencia, puesto que es el que mejor resultados tiene en ambas métricas, obtenemos las siguientes tablas de comparación:

\begin{table}[H]
\centering
\begin{tabular}{|c|c|c|c|}
\hline
\textbf{Lenguaje} & \textbf{Bubble Sort} & \textbf{Monte Carlo Pi} & \textbf{Monte Carlo Pi (SimpleRNG)} \\
\hline
C      & 38.2\%  & 37.9\%  & -0.8\% \\
C++    & 38.2\%  & 55.5\%  & -0.8\% \\
Go     & 6.5\%   & 64.1\%  & 0.4\%           \\
Java   & 17.0\%  & 59.6\%  & 19.4\%          \\
Python & 98.0\%  & 99.3\%  & 99.4\%          \\
\hline
\end{tabular}
\caption{Porcentaje de mejora de Rust con respecto a otros lenguajes de programación en términos de tiempo de CPU. (Mayor es mejor)}
\end{table}

\begin{table}[H]
\centering
\begin{tabular}{|c|c|c|c|}
\hline
\textbf{Lenguaje} & \textbf{Bubble Sort} & \textbf{Monte Carlo Pi} & \textbf{Monte Carlo Pi (SimpleRNG)} \\
\hline
C      & 89.8\%  & 20.4\%  & 25.6\%  \\
C++    & 45.1\%  & 45.3\%  & 45.9\%  \\
Go     & 65.4\%  & 43.5\%  & 44.0\%  \\
Java   & 97.8\%  & 95.4\%  & 95.4\%  \\
Python & 89.7\%  & 78.4\%  & 78.2\%  \\
\hline
\end{tabular}
\caption{Porcentaje de mejora de Rust con respecto a otros lenguajes de programación en términos de uso de memoria. (Mayor es mejor)}
\end{table}

Dado que para nuestro proyecto se busca la solución más eficiente (tanto en términos de velocidad como recursos) y segura, una vez vista la comparación entre las tecnologías candidatas, el análisis se puede reducir a una comparación entre Go y Rust, lenguajes que ofrecen un alto rendimiento y seguridad en la gestión de memoria.
Lenguajes como C, C++ han sido descartados puesto que, aunque ofrecen un alto rendimiento, su gestión de memoria es propensa a errores y fugas de memoria, lo que puede ser problemático en un desarrollo de código abierto y colaborativo.

Rust y Go son dos lenguajes de programación modernos que han ganado una popularidad considerable en los últimos años, especialmente en el desarrollo de sistemas y aplicaciones de alto rendimiento. Aunque ambos comparten objetivos como la eficiencia y la concurrencia, sus filosofías de diseño y enfoques para resolver problemas difieren significativamente.

\textbf{Rust} es un lenguaje de programación de sistemas enfocado en la seguridad de memoria y la concurrencia. Su principal objetivo es ofrecer el rendimiento de C/C++ sin los problemas comunes de gestión de memoria, como los punteros nulos o las \glspl{condicion-carrera}, gracias a su sistema de propiedad (\gls{rust-ownership}) y préstamos (\gls{rust-borrowing}).

\textbf{Go} (también conocido como Golang) es un lenguaje desarrollado por Google, diseñado para ser simple, eficiente y productivo, especialmente para la programación concurrente y de redes. Prioriza la simplicidad en su sintaxis y herramientas, facilitando una curva de aprendizaje suave.

\subsection{Gestión de memoria}
\subsubsection{Rust}

Rust implementa un sistema de gestión de memoria único basado en los conceptos de \textit{ownership}, \textit{borrowing} y \textit{\glspl{rust-lifetimes}}. Este sistema garantiza la seguridad de memoria en tiempo de compilación sin necesidad de un \acrfull{gc}.

El sistema de gestión de memoria de Rust se basa en las siguientes reglas (\cite{rustbook2024} Capítulo 4):
\begin{itemize}
    \item Cada valor en Rust tiene un único propietario (owner).
    \item Cuando el propietario sale del ámbito, el valor se libera automáticamente.
    \item Los valores pueden ser prestados (borrowed) de forma mutable o inmutable, pero no ambos al mismo tiempo.
    \item Solamente puede haber un préstamo mutable o múltiples préstamos inmutables a un valor al mismo tiempo. Un préstamo mutable no puede existir a la vez que un préstamo inmutable, dado que podría llevar a condiciones de carrera.
    \item Los préstamos tienen una duración (lifetime) que garantiza que los datos referenciados sigan siendo válidos durante su uso. El tiempo de vida de los préstamos se determina en tiempo de compilación, lo que evita errores comunes de punteros nulos o dangling pointers.
\end{itemize}

Gracias a ello, tenemos control preciso sobre la memoria, ausencia de pausas por GC, prevención de fugas de memoria y carreras de datos de forma estática. El compilador será muy estricto, lo que hará que prevengamos errores en tiempo de ejecución.

Todo este paradigma de programación es totalmente distinto a lo que estamos acostumbrados en otros lenguajes de programación como puede ser Java o c++/c, lo que puede llevar a una curva de aprendizaje más pronunciada, pero a largo plazo nos va a permitir desarrollar aplicaciones más seguras y eficientes.

\subsubsection{Go}

Go utiliza un recolector de basura para la gestión automática de la memoria. Este GC está optimizado para baja latencia, aunque introduce ciertas pausas (\cite{go-documentation}, The Go Memory Model).

Go está diseñado para ser un lenguaje de programación lo más eficiente posible para concurrencia, por lo que su modelo de memoria también está optimizado para facilitar la programación concurrente. Utiliza un modelo de memoria basado en \glspl{csp} (Communicating Sequential Processes \cite{communicating-sequential-processes}), donde las goroutines (hilos ligeros) se comunican a través de canales, evitando la necesidad de compartir memoria directamente.
\begin{itemize}
    \item \textbf{Ventajas:} Simplifica la gestión de memoria para el desarrollador, haciendo más sencillo el desarrollo de aplicaciones concurrentes. El lenguaje es más fácil de aprender y usar, especialmente para principiantes.
    \item \textbf{Desventajas:} Introduce una sobrecarga (overhead) y posibles pausas no determinísticas debido al GC, lo que puede ser un inconveniente en aplicaciones de tiempo real estricto.
\end{itemize}

\subsection{Rendimiento}
\subsubsection{Rust}
Rust está diseñado para ofrecer un rendimiento comparable al de C y C++. Sus abstracciones de ``coste cero'' aseguran que las características de alto nivel no impongan una penalización en tiempo de ejecución. La ausencia de GC también contribuye a un rendimiento predecible.

Rust ofrece varias abstracciones de coste cero como \texttt{iterators}, \texttt{closures} y \texttt{async/await} que permiten escribir código limpio y expresivo sin sacrificar el rendimiento. Además, su sistema de tipos y el modelo de propiedad permiten al compilador realizar optimizaciones agresivas que en otro lenguaje son imposibles.

En el rendimiento podemos distinguir entre dos aspectos:
\begin{itemize}
    \item \textbf{Compilación:} Los tiempos de compilación pueden ser más largos debido a las exhaustivas verificaciones que realiza el compilador.
    \item \textbf{Ejecución:} Muy alta velocidad de ejecución y uso eficiente de los recursos gracias a su modelo de propiedad.
\end{itemize}

\subsubsection{Go}

Go ofrece un buen rendimiento, aunque generalmente no alcanza el nivel de Rust o C++ en tareas que requieren máxima optimización a bajo nivel. Su compilador es notablemente rápido en comparación con el de Rust, dado que está enfocado a solucionar errores en ejecución y no en compilación.

Go utiliza un modelo de concurrencia basado en \glspl{goroutine} y canales (\cite{go-documentation}, The Go Memory Model), lo que permite un alto grado de paralelismo sin complicaciones adicionales. Esto lo hace ideal para aplicaciones que requieren manejar múltiples tareas simultáneamente, como servidores web o servicios de red.

\begin{itemize}
    \item \textbf{Compilación:} Tiempos de compilación muy rápidos, lo que agiliza el ciclo de desarrollo.
    \item \textbf{Ejecución:} Buen rendimiento para la mayoría de las aplicaciones, especialmente en \acrshort{i-o} y concurrencia. El GC puede impactar el rendimiento en ciertos escenarios.
\end{itemize}

\subsection{Concurrencia}
\subsubsection{Rust}

Rust aborda la concurrencia con un enfoque en la seguridad (``\gls{fearless-concurrency}'', \cite{rustbook2024}, Capítulo 16), el cual permite realizar operaciones concurrentes sin preocuparse por problemas usuales de la concurrencia como pueden ser las condiciones de carrera.
Su sistema de tipos y el modelo de propiedad previenen las carreras de datos en tiempo de compilación.
Utiliza primitivas como \texttt{async/await} para la programación asíncrona, además de hilos de sistema operativo, consiguiendo paralelismo para la programación asíncrona.

Rust permite la creación de hilos seguros y eficientes, y su modelo de propiedad garantiza que no haya condiciones de carrera. El compilador verifica en tiempo de compilación que no se acceda a datos compartidos de forma insegura gracias a su modelo de propiedad de variables, lo que reduce significativamente los errores comunes en la programación concurrente.
\begin{itemize}
    \item \textbf{Ventajas:} Concurrencia segura sin condiciones de carrera garantizada por el compilador. Buen soporte para paralelismo.
    \item \textbf{Desventajas:} La programación concurrente puede ser más verbosa o compleja de configurar inicialmente en comparación con Go.
\end{itemize}

\subsubsection{Go}

La concurrencia es una de las características estrella de Go. Se basa en \textit{goroutines} y \textit{canales} (channels) para la comunicación entre goroutines, siguiendo el paradigma de \acrfull{csp} \parencite{communicating-sequential-processes}.
Éste se basa en la idea de que las goroutines se comunican entre sí a través de canales sin acceder a las mismas posiciones de memoria (cada goroutine tiene su propia copia de el mensaje) y reduce el riesgo de condiciones de carrera.
Si se quiere tener una comunicación bidireccional, se tendría que utilizar un enfoque más tradicional mediante \glspl{mutex} o \glspl{semaforo} junto con \glspl{lock}, lo cual puede ser más complejo y propenso a errores.
\begin{itemize}
    \item \textbf{Ventajas:} Modelo de concurrencia muy simple y potente. Facilidad para escribir software concurrente y paralelo.
    \item \textbf{Desventajas:} Aunque las goroutines son ligeras, una mala gestión puede llevar a problemas de rendimiento o fugas de goroutines. Las condiciones de carrera son posibles y deben ser manejadas por el desarrollador.
\end{itemize}

\subsection{Sintaxis y curva de aprendizaje}
\subsubsection{Rust}
La sintaxis de Rust es moderna y expresiva, pero su sistema de tipos y el modelo de gestión de memoria (ownership y borrowing) introducen una curva de aprendizaje considerablemente más pronunciada que la de Go.
Ya se ha trabajado anteriormente con Rust, lo cual facilita el aprendizaje de este lenguaje.
Además, la documentación oficial de Rust es muy completa y está bien estructurada, contando con el libro oficial de rust \parencite{rustbook2024}, que ayuda a los nuevos usuarios a familiarizarse con el lenguaje.
\begin{itemize}
    \item \textbf{Ventajas:} Sintaxis potente que permite un control muy granular.
    \item \textbf{Desventajas:} Dificultad inicial alta para dominar los conceptos clave.
\end{itemize}

\subsubsection{Go}

Go fue diseñado con la simplicidad como uno de sus principios fundamentales. Su sintaxis es minimalista y fácil de aprender, especialmente para programadores con experiencia en lenguajes tipo C.
\begin{itemize}
    \item \textbf{Ventajas:} Curva de aprendizaje suave, alta legibilidad y productividad rápida.
    \item \textbf{Desventajas:} La simplicidad puede llevar a cierta verbosidad en algunos casos (por ejemplo, el manejo de errores antes de la versión 1.13 con \texttt{wrap error}).
\end{itemize}

\subsection{Sistema de tipos y abstracciones}
\subsubsection{Rust}

Rust posee un sistema de tipos estático, fuerte y muy rico, inspirado en lenguajes como \gls{haskell}. Incluye \textit{\gls{rust-traits}}, genéricos avanzados, \acrlong{adt} \gls{adt-gls} como \texttt{enum} y \texttt{struct} que, gracias a el \gls{pattern-matching} y el coste cero, nos permite un desarrollo muy expresivo y seguro.
\begin{itemize}
    \item \textbf{Ventajas:} Gran expresividad, seguridad de tipos, permite abstracciones potentes y seguras.
    \item \textbf{Desventajas:} Puede resultar complejo para quienes vienen de lenguajes con sistemas de tipos más simples.
\end{itemize}

\subsubsection{Go}

Go tiene un sistema de tipos estático y simple. Utiliza interfaces para la polimorfismo de forma implícita (tipado estructural). Los genéricos fueron añadidos en la versión 1.18, lo que ha expandido sus capacidades de abstracción.
\begin{itemize}
    \item \textbf{Ventajas:} Simplicidad en el sistema de tipos, interfaces fáciles de usar. La adición de genéricos ha mejorado la reutilización de código.
    \item \textbf{Desventajas:} Menos expresivo que el sistema de tipos de Rust. Antes de los genéricos, la falta de ellos era una limitación importante.
\end{itemize}

\subsection{Ecosistema y herramientas}
\subsubsection{Rust}

Rust cuenta con \textbf{Cargo}, una herramienta de gestión de dependencias y construcción de proyectos muy elogiada. El repositorio oficial de paquetes es \textbf{crates.io}, que alberga una cantidad creciente de bibliotecas.

Cargo además ofrece herramientas integradas para pruebas, documentación y gestión de versiones.
\begin{itemize}
    \item \textbf{Ventajas:} Herramientas robustas y unificadas. Comunidad activa y creciente.
    \item \textbf{Desventajas:} Aunque el ecosistema está creciendo rápidamente, puede no ser tan maduro como el de Go en ciertas áreas específicas (ej. algunas bibliotecas para servicios web muy específicos).
\end{itemize}

\subsubsection{Go}

Go posee una excelente librería estándar que cubre muchas necesidades comunes, especialmente en networking y servicios web. Sus herramientas de desarrollo (formateo, testing, profiling) están integradas en la distribución del lenguaje. Utiliza módulos de Go para la gestión de dependencias.
\begin{itemize}
    \item \textbf{Ventajas:} Librería estándar muy completa. Herramientas simples y efectivas. Compilación cruzada sencilla.
    \item \textbf{Desventajas:} Menor cantidad de bibliotecas de terceros para ciertos dominios muy especializados en comparación con lenguajes más antiguos, aunque el ecosistema es maduro para sus casos de uso principales.
\end{itemize}

\subsection{Manejo de errores}
\subsubsection{Rust}
Rust no utiliza excepciones. El manejo de errores se realiza principalmente a través de los tipos \texttt{Result<T, E>} y \texttt{Option<T>} y el pattern matching exhaustivo, que obligan al programador a considerar los casos de éxito y error explícitamente, sin posibilidad de compilar si no se manejan los errores correctamente.
\begin{itemize}
    \item \textbf{Ventajas:} Manejo de errores robusto y explícito, que previene errores no gestionados.
    \item \textbf{Desventajas:} Puede resultar verboso en comparación con las excepciones, aunque el operador \texttt{?}\footnote{El operador ? devuelve el error si ocurre sin necesidad de introducir un \textit{match}} ayuda a mitigar esto.
\end{itemize}

\subsubsection{Go}
Go maneja los errores retornándolos como el último valor de una función. Por convención, un error es un valor que satisface la interfaz \texttt{error}. Esto requiere comprobaciones explícitas \texttt{if err != nil}.
El manejo de errores en este caso no es exhaustivo.
\begin{itemize}
    \item \textbf{Ventajas:} Simple y explícito.
    \item \textbf{Desventajas:} Puede llevar a código repetitivo y verboso con múltiples comprobaciones de error.
\end{itemize}

\subsection{Casos de uso principales}
\subsubsection{Rust}

\begin{itemize}
    \item Programación de sistemas (sistemas operativos, navegadores web).
    \item Motores de videojuegos.
    \item \acrfull{cli}.
    \item Desarrollo en \gls{webassembly} (WASM).
    \item Sistemas embebidos.
    \item Aplicaciones que requieren alto rendimiento y seguridad de memoria.
\end{itemize}

\subsubsection{Go}

\begin{itemize}
    \item Servicios de backend y microservicios.
    \item Herramientas de red y servidores.
    \item Herramientas de \gls{devops} y CLI.
    \item Bases de datos distribuidas.
    \item Aplicaciones concurrentes.
\end{itemize}

\subsection{Resumen de ventajas y desventajas}

\subsubsection{Rust}
Ventajas:
\begin{itemize}
    \item Seguridad de memoria sin recolector de basura.
    \item Alto rendimiento, comparable a C/C++.
    \item Concurrencia segura ("fearless concurrency").
    \item Sistema de tipos rico y expresivo.
    \item Excelente gestor de paquetes y herramientas (Cargo).
    \item Creciente popularidad en dominios críticos.
\end{itemize}
Desventajas:
\begin{itemize}
    \item Curva de aprendizaje pronunciada.
    \item Tiempos de compilación más lentos.
    \item Mayor verbosidad en algunos aspectos debido a la gestión de memoria y errores.
\end{itemize}

\subsubsection{Go}
Ventajas:

\begin{itemize}
    \item Simplicidad y facilidad de aprendizaje.
    \item Modelo de concurrencia simple y potente (goroutines, channels).
    \item Tiempos de compilación muy rápidos.
    \item Librería estándar robusta.
    \item Buena productividad para el desarrollo de servicios de red.
    \item Respaldo de Google y una comunidad madura.
\end{itemize}
Desventajas:
\begin{itemize}
    \item Rendimiento generalmente inferior a Rust en cómputo intensivo.
    \item El recolector de basura puede introducir latencias.
    \item El sistema de tipos es menos expresivo que el de Rust (aunque mejorado con genéricos).
    \item El manejo de errores puede ser verboso.
\end{itemize}


\section{Aplicación móvil}

Para el desarrollo de aplicaciones móviles en el contexto de bibliotecas de archivos multimedia, existen varias tecnologías que ofrecen diferentes enfoques y características. A continuación se presenta un análisis detallado de las principales opciones disponibles: React Native \parencite{danielsson2016reactnative}, Flutter, Progressive Web Apps (PWA) \parencite{tandel2018impact} y Lynx.js \parencite{danielsson2016reactnative} \parencite{lynx-documentation}.

\subsection{Arquitectura y funcionamiento}

\paragraph{React Native}
(\cite{react-native-documentation}, \href{https://reactnative.dev/architecture/overview}{Arquitectura}) Desarrollado por Meta, permite crear aplicaciones nativas usando JavaScript y React. Su arquitectura se basa en un \gls{bridge} que comunica el código JavaScript con los componentes nativos de la plataforma, renderizando una interfaz de usuario verdaderamente nativa. Permite la integración de código nativo si es necesario y ofrece \textit{hot reload} para agilizar el desarrollo.

\paragraph{Flutter}
(\cite{flutter-documentation}, \href{https://docs.flutter.dev/resources/architectural-overview}{Arquitectura}) Creado por Google, utiliza el lenguaje Dart y renderiza la interfaz de usuario desde cero mediante su propio motor gráfico, \gls{skia}. Esto garantiza una apariencia y comportamiento consistentes en todas las plataformas. Su arquitectura se compone de un \textit{engine} en C++ y un \textit{framework} en Dart que gestiona los widgets. La compilación es \gls{ahead-of-time} (AOT) para producción, lo que optimiza el rendimiento.

\paragraph{PWA}
(\cite{pwa-documentation}) Las \acrfull{pwa} son aplicaciones web que utilizan tecnologías estándar para ofrecer una experiencia similar a la nativa. Su arquitectura se apoya en \glspl{service-worker} para la funcionalidad offline y el cacheo de recursos, un \textit{Web App Manifest} para la instalación en el dispositivo, y un diseño \textit{responsive} para adaptarse a diferentes tamaños de pantalla.

\paragraph{Lynx.js}
(\cite{lynx-documentation}, \href{https://lynxjs.org/react/lifecycle.html#dual-thread-architecture-design}{Arquitectura doble hilo}) Es una tecnología emergente que ejecuta JavaScript en un \textit{runtime} nativo sin necesidad de un WebView, similar a React Native pero con un enfoque en el rendimiento a través de una arquitectura de doble hilo. Promete una compilación \textit{just-in-time} (JIT) optimizada para móviles y un \textit{bridge} de comunicación más eficiente.

\subsection{Rendimiento}

\paragraph{React Native}
Ofrece un buen rendimiento al utilizar componentes nativos, pero la comunicación a través del \textit{bridge} puede introducir latencia, especialmente en interacciones complejas o animaciones intensivas. Para tareas de alto rendimiento como el procesamiento de imágenes, a menudo se requiere el uso de bibliotecas de terceros que implementan la lógica en código nativo.

\paragraph{Flutter}
Proporciona un rendimiento alto y consistente, con animaciones fluidas a 60 \acrshort{fps} en la mayoría de los dispositivos. Al no depender de un \textit{bridge} y compilar a código nativo, la comunicación con las APIs del sistema es directa. Su motor Skia es especialmente eficiente para tareas gráficas intensivas.

\paragraph{PWA}
El rendimiento es generalmente inferior al de las soluciones nativas, ya que está limitado por el motor del navegador. Aunque las tecnologías como \gls{webassembly} y \gls{webgl} han mejorado las capacidades, las tareas intensivas pueden sufrir en dispositivos de gama baja o media.

\paragraph{Lynx.js}
Promete un rendimiento superior al de React Native gracias a su arquitectura de doble hilo, que separa la lógica de la aplicación de la renderización de la interfaz. Sin embargo, al ser una tecnología nueva, hay pocos datos empíricos que respalden estas afirmaciones en aplicaciones complejas del mundo real.

\subsection{Ecosistema y comunidad}

\paragraph{React Native}
Cuenta con el ecosistema más maduro y una de las comunidades más grandes en el desarrollo móvil multiplataforma. Existe una vasta cantidad de bibliotecas, herramientas y tutoriales disponibles, lo que facilita la resolución de problemas y la integración de funcionalidades.

\paragraph{Flutter}
Su ecosistema ha crecido rápidamente y es muy activo, con un fuerte respaldo de Google. Aunque tiene menos paquetes que React Native, la calidad y el mantenimiento de las bibliotecas principales son excelentes. La comunidad es conocida por ser colaborativa y acogedora.

\paragraph{PWA}
Se beneficia del ecosistema web en su totalidad, que es el más grande y diverso de todos. Sin embargo, encontrar paquetes específicos para funcionalidades móviles avanzadas puede ser más complicado, y la compatibilidad entre navegadores sigue siendo un desafío.

\paragraph{Lynx.js}
Su ecosistema es muy nuevo y la comunidad es muy pequeña. La documentación es escasa y encontrar soluciones a problemas específicos puede ser difícil. Su desarrollo está impulsado principalmente por TikTok, lo que plantea dudas sobre su viabilidad a largo plazo como proyecto de código abierto.

\subsection{Curva de aprendizaje y desarrollo}

\paragraph{React Native}
La curva de aprendizaje es relativamente suave para los desarrolladores con experiencia en React y JavaScript. La reutilización de código entre iOS y Android es alta (70-80\%), lo que acelera el desarrollo. El \textit{hot reload} es una característica muy apreciada.

\paragraph{Flutter}
Requiere aprender el lenguaje Dart y el paradigma de \textit{widgets} de Flutter, lo que puede suponer una curva de aprendizaje inicial más pronunciada. Sin embargo, una vez superada, el desarrollo es muy productivo gracias a herramientas como el \textit{hot reload} y una documentación excelente.

\paragraph{PWA}
El desarrollo es accesible para cualquier desarrollador web. Se utiliza una única base de código para la web y el móvil, y la distribución es tan simple como desplegar un sitio web, sin necesidad de pasar por las tiendas de aplicaciones.

\paragraph{Lynx.js}
Utiliza JavaScript/TypeScript, lo que lo hace familiar para los desarrolladores web. Sin embargo, la falta de documentación y ejemplos hace que la curva de aprendizaje sea artificialmente alta debido a la necesidad de experimentar y descubrir cómo funcionan las cosas.

\subsection{Análisis específico para bibliotecas de fotos}

\subsubsection{Procesamiento de Imágenes}
Para una aplicación de gestión de fotos, el procesamiento eficiente de imágenes es crucial.
\begin{itemize}
    \item \textbf{Flutter}: Sobresale gracias a su motor gráfico Skia, ideal para la manipulación de imágenes y la creación de interfaces personalizadas.
    \item \textbf{React Native}: Depende de bibliotecas nativas para el procesamiento intensivo, lo que puede añadir complejidad.
    \item \textbf{PWA}: Está limitado por la API Canvas y WebGL del navegador, cuyo rendimiento puede ser variable.
    \item \textbf{Lynx.js}: Promete buen rendimiento, pero carece de pruebas concluyentes.
\end{itemize}

\subsubsection{Gestión de Memoria}
La gestión de grandes colecciones de fotos requiere un manejo de memoria eficiente.
\begin{itemize}
    \item \textbf{Flutter}: Ofrece un control granular sobre la memoria y un \textit{garbage collector} optimizado para Dart.
    \item \textbf{React Native}: Depende del \textit{garbage collector} de JavaScript, lo que puede llevar a fugas de memoria si no se gestiona con cuidado.
    \item \textbf{PWA}: La gestión de memoria es automática y depende del navegador, con limitaciones en dispositivos más antiguos.
    \item \textbf{Lynx.js}: Promete optimizaciones, pero no están verificadas en la práctica.
\end{itemize}

\subsubsection{Carga y Visualización}
La carga perezosa (\textit{lazy loading}) y el cacheo son fundamentales para una experiencia de usuario fluida.
\begin{itemize}
    \item \textbf{Flutter}: Proporciona excelentes herramientas nativas para \textit{lazy loading} y cacheo eficiente.
    \item \textbf{React Native}: Existen bibliotecas muy populares y eficientes como \texttt{react-native-fast-image}.
    \item \textbf{PWA}: Los \textit{Service Workers} permiten un cacheo robusto, pero el almacenamiento local tiene limitaciones.
    \item \textbf{Lynx.js}: Ofrece un componente optimizado para la carga de imágenes y datos.
\end{itemize}

\subsection{Conclusiones}

Dado que el proyecto busca crear una biblioteca de fotos de código abierto competitiva, la elección tecnológica debe priorizar el rendimiento, la experiencia de usuario y la facilidad de contribución.

\textbf{Flutter} se perfila como una opción muy sólida debido a su rendimiento gráfico superior, la consistencia multiplataforma que facilita el mantenimiento por parte de una comunidad FOSS, y un ecosistema maduro con el respaldo de Google.

\textbf{React Native} es una alternativa viable, especialmente si el equipo de desarrollo tiene una fuerte experiencia en JavaScript. Su principal ventaja es el vasto ecosistema de bibliotecas y una mayor flexibilidad para integraciones nativas específicas.

\textbf{PWA} se descarta por su rendimiento inferior en tareas intensivas y las limitaciones en el acceso a las APIs nativas del dispositivo, lo que comprometería la calidad de la experiencia de usuario en una aplicación de gestión de fotos.

\textbf{Lynx.js} es una opción demasiado arriesgada en este momento. Su inmadurez, la falta de comunidad y la incertidumbre sobre su futuro lo hacen una opción arriesgada para un proyecto que busca sostenibilidad a largo plazo.
Se incluye también el factor de que es una tecnología muy nueva y con algunas funcionalidades fundamentales aún en desarrollo (como puede ser el envío de archivos mediante el cliente HTTP incorporado, una implementación de navegación entre pantallas, etc.).

\subsubsection{Tecnologías no nombradas}
Otras opciones como el desarrollo nativo (Kotlin/Swift) o Kotlin Multiplatform fueron consideradas. Se descartaron principalmente por la mayor curva de aprendizaje y, en el caso de KMP, por una madurez del ecosistema aún insuficiente para un proyecto de esta envergadura, a pesar de sus prometedoras capacidades para compartir código de forma nativa.


\section{Propuesta}
Ya que es un tfg de desarrollo describimos lo realizado y cómo, así como los resultados obtenidos.

\subsection{Metodología}
Para el desarrollo de este proyecto se va a hacer uso de la metodología ágil Scrum. Esta metodología se basa en el desarrollo iterativo e incremental, lo que permite una mayor flexibilidad y adaptación a los cambios durante el proceso de desarrollo.

Separaremos el desarrollo en distintos sprints, cada uno de ellos con una duración de dos semanas.

Durante cada sprint se seleccionarán las historias de usuario\footnote{Las historias de usuario son descripciones concisas y sencillas de una funcionalidad, escritas desde la perspectiva del usuario} al principio del sprint, se realizarán las tareas necesarias para completarlas y al final del sprint se realizará una revisión y una retrospectiva del mismo, donde se evaluará lo que se ha hecho, cómo se puede mejorar y se planificará el siguiente sprint dependiendo del estado del recién terminado.

Gracias a esta metodología conseguimos tener una organización muy clara de lo que se va a hacer, cómo se va a hacer y cuándo se va a hacer.

\subsection{Tecnologías}

\subsection{Historias de usuario}
En esta sección se detallan las historias de usuario de la aplicación, separadas en dos grupos: las de la aplicación de servidor y las de móvil.

Se ha considerado esta separación ya que la aplicación de servidor tiene un objetivo diferente al de la aplicación móvil, de esta manera conseguimos una mejor organización de las historias de usuario.

Durante los primeros sprints se trabajará de manera principalmente separada, enfocándose en la parte correspondiente que se defina de la aplicación y en una fase más avanzada se trabajará de manera conjunta, integrando ambas aplicaciones.

Para la planificación del desarrollo se han utilizado puntos de historia, los cuales representan una estimación de lo que se considera que se tardará en implementar las historias de usuario. Esta estimación es relativa, es decir, no representa un tiempo real sino una estimación con respecto a todas las demás historias de usuario, siendo 1 punto de historia la historia de usuario más sencilla de implementar o que menos tiempo requiere.

Este es un listado inicial de historias de usuario, durante los sprints se irá especificando si alguna historia de usuario ha cambiado, añadido o eliminado del product backlog\footnote{El product backlog es una lista priorizada de requisitos o tareas pendientes en un proyecto ágil.}.

Cada historia de usuario tiene un identificador único, una descripción de la historia de usuario y una estimación en puntos de historia. Ésta es después desglosada en historias de usuario más pequeñas de las cuales se definen tareas que tienen que ser realizadas para completar la historia de usuario con su estimación en horas.

\subsubsection{Servidor}

\subsubsection{Móvil} 

\subsection{Planificación inicial}

\subsection{Presupuesto}



\section{Conclusiones}
Reflejamos el problema, solución alcanzada y resultados.
Leemos los objetivos y vemos si se han cumplido y el por qué no en el caso de no haberlos alcanzado. 
Maximizamos información, no queremos algo extenso, queremos algo resumido


\printbibliography[title=Bibliografía]

\end{document}
